\documentclass[00_complete]{subfiles}
%\documentclass[12pt]{report}
\usepackage[utf8]{inputenc}
\usepackage{amsmath,amssymb,amsthm,gensymb,parskip,graphicx,footmisc,csquotes,enumerate,datetime2}
\usepackage[]{libertinus}
\usepackage[breaklinks]{hyperref}
\hypersetup{
  pdfauthor={Moshe Krumbein},
  colorlinks=true,
  linkcolor={black},
  filecolor={black},
  citecolor={black}, %blue
  urlcolor={black}, %blue
}
\usepackage[top=30mm,bottom=30mm,left=30mm,right=30mm]{geometry}
%\setlength{\emergencystretch}{2em} % prevent overfull lines
\providecommand{\tightlist}{%
\setlength{\itemsep}{0pt}\setlength{\parskip}{0pt}}

\renewcommand\qedsymbol{$\blacksquare$}

\theoremstyle{definition}
\newtheorem*{definition}{Definition}
\newtheorem*{theorem}{Theorem}
\newtheorem*{axiom}{Axiom}
\newtheorem*{lemma}{Lemma}

\theoremstyle{remark}
\newtheorem*{note}{Note}
\newtheorem*{symbols}{Symbol}
\newtheorem{example}{Example}[section]
\newtheorem*{claim}{Claim}
\newtheorem*{conclusion}{Conclusion}
\newtheorem*{reminder}{Reminder}

\usepackage{fancyhdr}
\usepackage[italicdiff]{physics}
\MakeOuterQuote{"}

\renewcommand{\chaptermark}[1]{\markboth{#1}{}}

\pagestyle{fancy}

\setlength{\headheight}{14.5pt}
\addtolength{\topmargin}{-2.5pt}

\fancyhf{}
\rhead{Moshe Krumbein}
\lhead{\chaptermark}
\cfoot{\thepage}
\fancyhead[R]{\chaptername~\thechapter}
\fancyhead[L]{\mbox{\leftmark}}

\usepackage[Rejne]{fncychap}
\usepackage{titling}

\makeatletter
\renewcommand{\@chapapp}{\vspace*{-100pt}\huge\thetitle}
\makeatother

\makeatletter
\newcommand{\subtitle}[1]{%
  {\center\vspace*{-60pt}%
  \linespread{1.1}\Large\scshape#1%
  \par\nobreak\vspace*{35pt}}
}
\makeatother

\newcommand{\Chapter}[2]{
    \def\n{#2}
    \setcounter{chapter}{\the\numexpr\n-1}
    \chapter{#1}
    \subtitle{\theauthor~- \thedate}
}

\DeclareMathOperator{\Ima}{Im}
\DeclareMathOperator{\Id}{Id}
\DeclareMathOperator{\cis}{cis}

\newcommand{\Mod}[1]{\ (\mathrm{mod}\ #1)}
\newcommand{\st}[0]{\;\mathrm{s.t.}\;}

\title{Intro to Probability and Statistics}
\author{Moshe Krumbein}
\date{Fall 2022}

\begin{document}
\Chapter{Introduction}{1}

\section{What is Statistics?}

The quantitative measure of the amount of uncertainty in regards to an outcome
within a well defined space. (How "surprised" I would be at a given outcome)

By doing a test a large number of times and counting the number of "random"
outcomes, we are able to predict with certainty.

For example, we are find the boiling point of a given liquid, even though we
don't know the specific details of a given particle within the liquid.

\section{Trials}
\begin{definition}[Sample Space]
   A \emph{sample space} is non-empty set $\Omega$. (That describes the results
   of a sampling [and potentially additional results])
\end{definition}
\subsection{Examples}
\begin{table}[ht!]
\centering
{\renewcommand{\arraystretch}{1.2}% for the vertical padding
\begin{tabular}{ c|c|c }
    Sampling & $\Omega$ & $p$\\
    \hline
    Fair coin toss &  $\{H,T\}$ & $p_H=p_T=\frac{1}{2}$\\
    Unfair coin toss &  $\{H,T\}$ & $p_H=p=1-p_T$\\
    3 coin tosses & $\{H,T\}^3$ & $\forall a,b,c \in \{H,T\}:p_{abc}=\frac{1}{8}$\\
    Coin tosses until heads &$\{H,T\}^{\aleph_0}\;\slash\;\mathbb{N}\cup\{\infty\}$& $p_i=2^{-i}$\\
    Picking an angle on a circle &  $[0,2\pi)$&hard\\
    Rolling a D6 and flipping a coin &  $[6]\times\{H,T\}$&$p_{dc}=\frac{1}{12}$\\
\end{tabular}}
\end{table}
\begin{definition}[Probability Mass Function]
    $p:\Omega\to[0,1]$ such that $\sum\limits_{\omega\in\Omega} p_\omega=1$.

    $\{\omega: p_\omega>0\}$ is called the \emph{support} of $p$.
\end{definition}
\begin{definition}[Event]
    $\mathcal F = 2^\Omega$. An \emph{event} in $F$ is the set $E \in \mathcal F$

    Practically, an \emph{event} is a "subset" of $\Omega$.
\end{definition}
\begin{definition}[Probability Function]
    $\mathbb{P}:\mathcal F \to [0,1]$ such that $\mathbb{P}(\Omega)=1$.

    $$\forall(E_i)_{i \in \mathbb{N}}E_i \in \mathcal F: E_i\cap E_j = \emptyset
    \implies \mathbb{P}\left(\bigcup_{i=1}^\infty E_i\right) =
    \sum_{n=1}^{\infty}\mathbb{P}(E_i)$$
\end{definition}
\section{Characteristics of the Probability Function}
\begin{itemize}
    \item Additivity- $\mathbb{P}(E_1 \cupdot E_2) = \mathbb{P}(E_1) +
        \mathbb{P}(E_2)$
    \item Monotonocity - $E_1 \subset E_1 \implies \mathbb{P}(E_2) \leq
        \mathbb{P}(E_2)$
    \item Completeness - $\mathbb{P}(A^c) = 1-\mathbb{P}(A)$ (where $A^c=\Omega
        \setminus A$)
\end{itemize}
The trio $(\Omega, \mathcal F, \mathbb{P})$ where $\Omega$ is a \emph{sample
space}, $\mathcal F$ is a \emph{event space} and $\mathbb{P}$ is a
\emph{probability function} on $\mathcal F$ is called a \emph{probability
space}.
\begin{definition}[Discrete Probability Space]
    For $p$ we can say that $\mathbb{P}$ fits $p$ if $\mathbb{P}(A)=
    \sum\limits_{a\in A}p_a$. If $\mathbb{P}$ has a discrete function, it will
    be $p_\omega=\mathbb{P}(\{\omega\})$.
\end{definition}
\begin{claim}
    The following are equivalent:
    \begin{itemize}
        \item $\mathbb{P}$ has a discrete $p$
        \item $\mathbb{P}$ is supported by a countable set
        \item $\mathbb{P}(A)=\sum\limits_{a \in A}\mathbb{P}(\{a\})$
    \end{itemize}
    $(\Omega, \mathcal F, \mathbb{P})$ such that $\mathbb{P}$ satisfies this is
    called a \emph{discrete probability space}.
\end{claim}
\section{Examples}

\begin{example}
In a pot there are six balls, on four of them are written $A$ and on two are
written $B$. One ball is removed and its letter is written down.

Description A:
\begin{gather*}
    \Omega =\{A,B\} \quad p_A = \frac{4}{6}=\frac{2}{3},
    p_B=\frac{2}{6}=\frac{1}{3} \\
    \mathbb{P}(\emptyset)=1-\mathbb{P}(\Omega) \\
    \mathbb{P}(\{A\})=1-\mathbb{P}(\{B\})=\frac{2}{3}
\end{gather*}
Description B:
\begin{gather*}
    \Omega =\{1,\dots,6\}=[6] \quad \forall i\neq j: p_i = p_j \quad \forall
    i:p_i=\frac{1}{6}\\
    \mathbb{P}(``A")=\mathbb{P}([4])=\frac{4}{6}=\frac{2}{3} \\
\end{gather*}
\end{example}
\begin{definition}[Bernoulli trial]
    \emph{Probability space} such that:
    \[
        \left(\{0,1\},2^{\{0,1\}},\mathbb{P}_p\right)
    \]
    where $p_0=1-p, p_1=p$.

    In other words where the outcomes are either $0$ or $1$ and the odds of $1$
    is $p$.
\end{definition}
\begin{definition}[Uniform Probability Space]
   $$(\Omega,2^{\Omega},\mathbb{P}_p)$$
   where $\forall \omega \in \Omega: p_\omega=\frac{1}{|\Omega|}$.

   A probability space is uniform \emph{over A} if for all spaces $A
   \subseteq \Omega$:
   \[
       p_a=
       \begin{cases}
           \frac{1}{|A|} & a \in A \\
           0 & a \notin A
       \end{cases}
   \]
\end{definition}
\begin{example}
    Sum of two D6s
\end{example}
\begin{claim}
    Over a \emph{uniform probability space}:
    \[
        \mathbb{P}(A)=\frac{|A|}{|\Omega|}
    \]
\end{claim}
\begin{definition}[Multiplying Probability Spaces]
    Let there be $\Omega_1, \Omega_2, p^1, p^2$:
    \[
        p:\Omega_1\times\Omega_2\to [0,1] \\
    \]
    such that:
    \[
        p_{(a,b)}=p^1_a\cdot p^2_b
    \]
    Then the probability space
    \[
        (\Omega=\Omega_1\times\Omega_2,2^\Omega,\mathbb{P}_p)
    \]
    is the \emph{multiplication} of:
    \[
        (\Omega_1,2^{\Omega_1},\mathbb{P}_{p^1}) \times
        (\Omega_2,2^{\Omega_2},\mathbb{P}_{p^2})
    \]
\end{definition}
\begin{definition}[Multiplication of Events]
    Let $\Omega_1\times \Omega_2$ be a multiplication of sample spaces. The
    event $A\times B: A\in \Omega_1,\; B \in \Omega_2$ is called a
    multiplication of events.
\end{definition}
\section{Two-step Experiments}
\begin{example}
We flip 2 coins and we pick a number between $1$ and $5h+5$ such that $h$ is
the number of heads flipped. What are the odds the number will be $3$ or less?

There are two sample spaces:
\begin{itemize}
    \item $\Omega_1$: number of heads
    \item $\Omega_2$: number we pick
\end{itemize}
Let $(\Omega_1, \mathcal F, \mathbb{P})$ (the first step of the experiment),
$\Omega_2$ a sample space and for all $\omega \in \Omega_1$ the probability
space $(\Omega_2, \mathcal F, \mathbb{P}_{2,\omega})$.

Then define $(\Omega, \mathcal F, \mathbb{P}): \Omega: \Omega_1 \times
\Omega_2$ where $p$ is defined as
$$p(\omega_1,
\omega_2)=\mathbb{P}_1(\{\omega_1\}) \cdot
\mathbb{P}_{2,\omega_1}(\{\omega_2\})$$
In other words it's the odds of each case in the first step times the odds of
the second step, given each potential first step.

The odds that the number that is picked is less than $3$ is expressed by:
$$\{(\cdot,b): b \leq 3\}$$
so we define:
\begin{gather*}
    A_0=\{(0,b):b\leq3\}\\
    A_1=\{(1,b):b\leq3\}\\
    A_2=\{(2,b):b\leq3\}\\
    E=A_1 \cupdot A_2 \cupdot A_3
\end{gather*}
From "additivity":
\[
    \mathbb{P}(E)=\mathbb{P}(A_0)+\mathbb{P}(A_1)+\mathbb{P}(A_2) \\
\]
where:
\begin{gather*}
    \mathbb{P}(A_0)=\mathbb{P}(\{(0,b):b\leq 3\}) =
    \mathbb{P}(\{(0,1),(0,2),(0,3)\})=3\left(\frac{1}{4}\cdot
        \frac{1}{5}\right) \\
    \mathbb{P}(A_1)=3\left(\frac{1}{2}\cdot \frac{1}{10}\right) \\
    \mathbb{P}(A_2)=3\left(\frac{1}{4}\cdot \frac{1}{15}\right) \\
    \mathbb{P}(E)=3 \left( \frac{3}{60}+ \frac{1}{60}+\frac{1}{60}\right) =
    \frac{21}{60}
\end{gather*}

\end{example}
\begin{example}
\begin{gather*}
    (\bar{\Omega}, \bar{\mathcal F}, \bar{\mathbb{P}}) \\
    p(\omega_1, \omega_2,\omega_3, \cdots) =
    \mathbb{P}(\{\omega_1\})\cdot\mathbb{P}_{\omega_1}(\{\omega_2\})\cdot\mathbb{P}_{\omega_1,\omega_2}(\{\omega_2\})\cdot
    \cdots
\end{gather*}
\end{example}
\section{The Birthday Paradox}
\begin{gather*}
    \Omega=[265]^2 \\
    P(A^c)=\frac{|A^c|}{|\Omega|}=\frac{365\cdot365\cdot363\cdot
    (265-n+1)}{365^n}
\end{gather*}

\begin{table}[ht!]
\centering
{\renewcommand{\arraystretch}{1.2}% for the vertical padding
\begin{tabular}{ c|c}
    $n$ & $\mathbb{P}(A)$\\
    \hline
    $23$ & $0.507$ \\
    $50$ & $0.970$ \\
    $70$ & $0.999$
\end{tabular}}
\end{table}
\section{Law of Total Probability}
\[
    \mathbb{P}(B)=\sum_{n \in \mathbb{N}} \mathbb{P}(B\cap A_n)
\]



\end{document}
