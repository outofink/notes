\documentclass[00_complete]{subfiles}

%\documentclass[12pt]{report}
\usepackage[utf8]{inputenc}
\usepackage{amsmath,amssymb,amsthm,gensymb,parskip,graphicx,footmisc,csquotes,enumerate,datetime2}
\usepackage[]{libertinus}
\usepackage[breaklinks]{hyperref}
\hypersetup{
  pdfauthor={Moshe Krumbein},
  colorlinks=true,
  linkcolor={black},
  filecolor={black},
  citecolor={black}, %blue
  urlcolor={black}, %blue
}
\usepackage[top=30mm,bottom=30mm,left=30mm,right=30mm]{geometry}
%\setlength{\emergencystretch}{2em} % prevent overfull lines
\providecommand{\tightlist}{%
\setlength{\itemsep}{0pt}\setlength{\parskip}{0pt}}

\renewcommand\qedsymbol{$\blacksquare$}

\theoremstyle{definition}
\newtheorem*{definition}{Definition}
\newtheorem*{theorem}{Theorem}
\newtheorem*{axiom}{Axiom}
\newtheorem*{lemma}{Lemma}

\theoremstyle{remark}
\newtheorem*{note}{Note}
\newtheorem*{symbols}{Symbol}
\newtheorem{example}{Example}[section]
\newtheorem*{claim}{Claim}
\newtheorem*{conclusion}{Conclusion}
\newtheorem*{reminder}{Reminder}

\usepackage{fancyhdr}
\usepackage[italicdiff]{physics}
\MakeOuterQuote{"}

\renewcommand{\chaptermark}[1]{\markboth{#1}{}}

\pagestyle{fancy}

\setlength{\headheight}{14.5pt}
\addtolength{\topmargin}{-2.5pt}

\fancyhf{}
\rhead{Moshe Krumbein}
\lhead{\chaptermark}
\cfoot{\thepage}
\fancyhead[R]{\chaptername~\thechapter}
\fancyhead[L]{\mbox{\leftmark}}

\usepackage[Rejne]{fncychap}
\usepackage{titling}

\makeatletter
\renewcommand{\@chapapp}{\vspace*{-100pt}\huge\thetitle}
\makeatother

\makeatletter
\newcommand{\subtitle}[1]{%
  {\center\vspace*{-60pt}%
  \linespread{1.1}\Large\scshape#1%
  \par\nobreak\vspace*{35pt}}
}
\makeatother

\newcommand{\Chapter}[2]{
    \def\n{#2}
    \setcounter{chapter}{\the\numexpr\n-1}
    \chapter{#1}
    \subtitle{\theauthor~- \thedate}
}

\DeclareMathOperator{\Ima}{Im}
\DeclareMathOperator{\Id}{Id}
\DeclareMathOperator{\cis}{cis}

\newcommand{\Mod}[1]{\ (\mathrm{mod}\ #1)}
\newcommand{\st}[0]{\;\mathrm{s.t.}\;}

\title{Introduction to Electrical Engineering}
\author{Moshe Krumbein}
\date{Fall 2022}

\begin{document}
\Chapter{Lumped Elements (Circuits)}{1}

For example, if we attach a 1.5V with 2 wires to a light bulb, what is the
current (amps) are running through the wires?
$$I = \frac{V}{R}$$
This is a major abstraction and is not always true!

We do know that Maxwell's Laws are always true:
\begin{enumerate}
    \item $\vec \grad \cdot \vec E = 4 \pi \rho$
    \item $\vec \grad \cdot \vec B = 0$
    \item $\vec \grad \times \vec E = -\frac{1}{c}\pdv{\vec B}{t}$
    \item $\vec \grad \times \vec B = \frac{1}{c}\left(4 \pi \vec j
        + \pdv{\vec E}{t}\right)$
\end{enumerate}

When we can we use abstraction?

We can change an component and use its Volt-Current Relation (V-I).
This can only be done if the voltage and current to be defined (injective).

\begin{reminder}
    Review from electromagnetism:
   \begin{enumerate}
       \item \underline{Current:} $I$ which passes through area $A$ which is
           defined to be the amount of charge that passes through $A$ per time
           unit.
           $$I \equiv \frac{dq}{dt}$$
   \end{enumerate}
\end{reminder}
Let's relate these equations:
\[
    I = \iint_A \vec j (\vec  r ) \dd{s}
\]
\begin{reminder}
    Review from electromagnetism:
   \begin{enumerate}
       \item[2.] \underline{Voltage:} Difference in electric potentials between
           between two points (symbolized by $V$, $V_{ab}$, $V_{12}$) \\
           Voltage is a measure of the required energy/work needed in order to
           to move a charge from point $a$ to point $b$. \\
           (Voltage is a amount of energy accumulated/flowing) in a component.)
           \[
               \text{(Volt)} =\frac{\text{(Joule)}}{\text{(Col)}}
               \left(\frac{\text{(Energy)}}{\text{(Charge)}}\right)
           \]
   \end{enumerate}
\end{reminder}
\[
    W = \int_{r_1}^{r_2} \vec F (\vec r) \dd{\vec r} = \int_{r_1}^{r_2} q \vec
    E (\vec r) \dd{\vec r}
\]
Only if $\vec \grad \times \vec E = 0$:
\[
    W = q(\varphi(\vec r_1) - \varphi (\vec r_2)) = q(V_2-V_1)=qV_{12}
\]
In other words, only when there is no change in the magnetic field (Maxwell's
4th law), or at the very least \emph{quasistatic} (changes at an incredibly
slow rate).

Requirements for the \emph{lumped elements model}:
\begin{enumerate}
    \item $\pdv{B}{t} \approx 0$ outside of the components (required to define
        electric potential/voltage)
\end{enumerate}
\begin{reminder}
    Review from electromagnetism:
   \begin{enumerate}
       \item[3.] \underline{Power:} Energy per unit of time (Watt):
           \[
               P(t) \equiv \frac{\dd{W}}{\dd{t}}
               = \frac{\dd{W}}{\dd{q}} \cdot \frac{\dd{q}}{\dd{t}}
               = V(t) \cdot I(t)
           \]
           We "agree":
           \begin{itemize}
                \item $P > 0$ - power is used in a component (resistor)
                \item $P \leq 0$ - power is provided by a component
           \end{itemize}
        \item[4.] \underline{Direction and Signs:}
            \[
                B \to A \implies V = V_A-V_B
            \]
            $I > 0$ - The positive particles "move" from left to right

            The current is \emph{positive} if the current is in the direction
            of the arrow.
            \begin{note}
            It's easiest to define the direction of the positive current from
            the head of the electrical potential arrow to its tail.
            \end{note}
   \end{enumerate}
\end{reminder}
Back to our requirements:
\begin{enumerate}
    \item[2.] Current over components in series do not change:
        \begin{gather*}
            I_A=I_B \\ I_A=I_B=0 \\
            \frac{\dd{Q_{in}}}{\dd{t}}-\frac{\dd{Q_{out}}}{\dd{t}} = 0 \\
            \frac{\dd}{\dd{t}}(Q_{in}-Q_{out}) = 0 \\
        \frac{\dd}{\dd{t}}\left(\sum
        _{\substack{\text{between} \\ A \text{ and } B}} Q\right) = 0
        \end{gather*}
        In other words charge is not created.
    \item[3.] $V_{AB}=V_{CD} \leftarrow l \ll ct$ \\
        In other words, we must only work with components/circuits that there
        size is are small enough in relation to measurement time.
\end{enumerate}
\begin{gather*}
    2\pi f_o\Delta t \ll 2\pi \\
    \Delta t \ll \frac{1}{f_0} = T \\\\
    l \ll \frac{c}{f_0}
\end{gather*}
\section{Tricks to Solve Circuits}
Ways to solve circuits:
\begin{enumerate}
    \item By hand
    \item Linear Algebra
    \item "current-loop"
    \item Simplification
\end{enumerate}
\subsection{Node Analysis}
\begin{enumerate}
    \item Select one node as the ground reference. The choice does not affect
        the element voltages (but it does affect the nodal voltages) and is
        just a matter of convention. Choosing the node with the most
        connections can simplify the analysis. For a circuit of N nodes the
        number of nodal equations is $N-1$.
    \item Assign a variable for each node whose voltage is unknown. If the
        voltage is already known, it is not necessary to assign a variable.
    \item For each unknown voltage, form an equation based on Kirchhoff's
        Current Law (i.e. add together all currents leaving from the node and
        mark the sum equal to zero). The current between two nodes is equal to
        the voltage of the node where the current exits minus the voltage of
        the node where the current enters the node, both divided by the
        resistance between the two nodes.
    \item If there are voltage sources between two unknown voltages, join the
        two nodes as a supernode. The currents of the two nodes are combined in
        a single equation, and a new equation for the voltages is formed.
    \item Solve the system of simultaneous equations for each unknown voltage.
\end{enumerate}
If there are only sources of current and resistors, the circuit can be plugged
in directly into the matrix.
\end{document}
