\documentclass[00_complete]{subfiles}

%\documentclass[12pt]{report}
\usepackage[utf8]{inputenc}
\usepackage{amsmath,amssymb,amsthm,gensymb,parskip,graphicx,footmisc,csquotes,enumerate,datetime2}
\usepackage[]{libertinus}
\usepackage[breaklinks]{hyperref}
\hypersetup{
  pdfauthor={Moshe Krumbein},
  colorlinks=true,
  linkcolor={black},
  filecolor={black},
  citecolor={black}, %blue
  urlcolor={black}, %blue
}
\usepackage[top=30mm,bottom=30mm,left=30mm,right=30mm]{geometry}
%\setlength{\emergencystretch}{2em} % prevent overfull lines
\providecommand{\tightlist}{%
\setlength{\itemsep}{0pt}\setlength{\parskip}{0pt}}

\renewcommand\qedsymbol{$\blacksquare$}

\theoremstyle{definition}
\newtheorem*{definition}{Definition}
\newtheorem*{theorem}{Theorem}
\newtheorem*{axiom}{Axiom}
\newtheorem*{lemma}{Lemma}

\theoremstyle{remark}
\newtheorem*{note}{Note}
\newtheorem*{symbols}{Symbol}
\newtheorem{example}{Example}[section]
\newtheorem*{claim}{Claim}
\newtheorem*{conclusion}{Conclusion}
\newtheorem*{reminder}{Reminder}

\usepackage{fancyhdr}
\usepackage[italicdiff]{physics}
\MakeOuterQuote{"}

\renewcommand{\chaptermark}[1]{\markboth{#1}{}}

\pagestyle{fancy}

\setlength{\headheight}{14.5pt}
\addtolength{\topmargin}{-2.5pt}

\fancyhf{}
\rhead{Moshe Krumbein}
\lhead{\chaptermark}
\cfoot{\thepage}
\fancyhead[R]{\chaptername~\thechapter}
\fancyhead[L]{\mbox{\leftmark}}

\usepackage[Rejne]{fncychap}
\usepackage{titling}

\makeatletter
\renewcommand{\@chapapp}{\vspace*{-100pt}\huge\thetitle}
\makeatother

\makeatletter
\newcommand{\subtitle}[1]{%
  {\center\vspace*{-60pt}%
  \linespread{1.1}\Large\scshape#1%
  \par\nobreak\vspace*{35pt}}
}
\makeatother

\newcommand{\Chapter}[2]{
    \def\n{#2}
    \setcounter{chapter}{\the\numexpr\n-1}
    \chapter{#1}
    \subtitle{\theauthor~- \thedate}
}

\DeclareMathOperator{\Ima}{Im}
\DeclareMathOperator{\Id}{Id}
\DeclareMathOperator{\cis}{cis}

\newcommand{\Mod}[1]{\ (\mathrm{mod}\ #1)}
\newcommand{\st}[0]{\;\mathrm{s.t.}\;}

\title{Algorithms}
\author{Moshe Krumbein}
\date{Spring 2023}

\begin{document}
\Chapter{Introduction}{1}
\section{What is an Algorithm?}
An \textit{algorithm} is a formal description of an effective method of solving
a well-defined set of problems, while using limited resources (namely, time and
space).

Specifically, an algorithm matches any possible input to an output that solves a
given problem for that input.

Most of the problems that we will be focusing on are combinitorical
optimization, where the input is defined as a large finite set of potential
solutions and we have to select the minimal (or maximal) solution.

We will focus on asymptotic efficiency of the algorithm, as a function of the
size of the input.

\section{History}
One of the first descriptions of algorithms was found in ancient Babylonia. In
ancient Greece \textit{Euclid's Algorithm}, \textit{Sieve of Eratosthenes}, and
the \textit{Babylonian Square-Root Method}, were created from the third century
BCE until the first century CE.

The \textit{Hindu-Arabic numeral system} was developed by Indian mathematician
between the first and fourth centuries and was later adapted by Arabic
mathematicians by the ninth century. Most notably the digit $0$ was first used
by the turn of the seventh century by Brahmagupta.

Although first found to be used in the first century in China, \textit{Gauss'
Elimination Method} was monumental development for algorithmically solving
matrices.

In 1928, Hilbert and Ackermann proposed the \textit{Entscheidungsproblem}
(decision problem), which asks if a problem can, by way of an algorithm, can be
answered "Yes" or "No" in according to whether the input, a statement, is
universally valid. 
\section{Matroids}

Given a finite non-directed connected graph $ G=(V,E) $ and a weight
function $ w $ on the edges $ w: E \to \mathbb{N} $. We are searching for a
\textit{spanning tree} such that that sum of the weights of the edges on the
tree are \textit{minimal}.

A \textit{tree} is a connected acyclic graph. A tree \textit{spans}
$ G $ if and only if it contains all on the nodes of $ G $.

\begin{definition}[Kruskal's Algorithm]
    We sort the edges in non-decreasing order of weight. We then traverse all
    the edges in order and add each edge that does not close a cycle with
    edges that we have already added.
\end{definition}

In general we symbolize $ |V|=n, |E|=m $.

Instead of $ w(E) $ we will use $ W- w(e) $ where $ W> \max_e w(e) $. In this
way we transform Kruskal's algorithm to find a maximum spanning tree instead
of a minimum spanning tree.
\begin{definition}[Matroid]
    Given a finite set $ E $ and a collection $ I $ of subsets of $ E $, the
    ordered pair $ (E,I) $ is called a \textit{matroid} if and only if:
    \begin{enumerate}
        \item \textbf{Inheritance}: If $ A \in I $ and $ B \subseteq I $, then
            $ B \in I $
        \item \textbf{Extension}: If $ A,B \in I $ and $ |B|<|A| $, then there
            exists $ c \in A \setminus B $ such that $ B \cup \{e\} \in I $
    \end{enumerate}
\end{definition}
The sets in $ I $ are called \textit{independent}. Any set in $ I $ that is
maximal in relation to inclusion is called a \textit{basis}.
\begin{claim}
    In a matroid, all bases are the same size.
\end{claim}
\begin{claim}[Exchange]
    If $ A,B\in I $ are two bases, then for all $ e \in A \setminus B $ there
    exists $ f \in B \setminus A $ such that $ A \setminus \{e\} \cup \{f\} $ is
    also a basis.
\end{claim}
If $ I $ is the collection of all of the subsets of $ E $ or an empty set, these
are the trivial matroids.

Given $ n $ vectors over some vector space (for example $ \mathbb{R}^d $) -
$ E $. The subsets can be in $ I $ if and only if it is linearly independent.

Let $ E $ be a set of non-directed edges of a finite graph $ A \subseteq E $
that's in $ I $ if and only if $ A $ is acyclic.
\begin{claim}
    $ (E,I) $ is a matroid (cyclic matroid of $ G $).
\end{claim}

Given a finite non-directed graph $ G=(V,E) $, the pair $ (E,I) $ such that
$ A \in I $ if and only if for every connected component of $ A $ there is at
most one cycle that is a matroid.

Given matroid $ (E,I) $ and a weight function $ w: E \to \mathbb{N} $, we want
to find $ A \in I $ such that $ w(A) = \sum_{e \in A}w(e) $ is maximal.

\begin{definition}[Greedy Algorithm]
    We sort the elements of $ E $ in non-increasing order and we traverse over
    the elements in that order and add to the solution that keeps it in $ I $.
\end{definition}

\begin{theorem}[Rado--Edmonds Theorem]
    Let the system of sets $ (E,I) $ (where $ E $ is finite) that maintains
    \textit{inheritance} and let $ w: E \to \mathbb{N}$ be a weight function
    over the elements of $ E $. The \textit{greedy algorithm} finds $ A \in I $
    with a maximal $ w(A) $ if and only if $ (E, I) $ is a matroid.
\end{theorem}
\begin{proof}
    We with prove both directions:
    \begin{itemize}
        \item[$\Rightarrow$] Clearly the result is a basis because every element
            that we skipped over cannot be added to the to the final result
            (implied by inheritance).

            To prove that the result is optimal (in our case, maximal), we will
            prove it using induction over the size of the partial solution which
            contains a optimal basis.

            Let $ B $ be some optimal basis and $ A $ be $ A $ be the set that
            the algorithm is using. At the beginning $ A = \emptyset $ so
            obviously $ A \subseteq B $.

            We will assume that $ A \subseteq B $ and we will consider $ e \in
            E $. If $ A \cup \{e\} \notin I $ and $ B \cup \{e\} \notin i $,
            then we're left with $ A \subseteq B $.

            If so, we assume that $ A \cup \{e\} \in I $. If $ e \in B $ then
            clearly $ A \cup \{e\} \subseteq B$. If so, we're left with the
            problem that $ e \notin B $. But then, we will extend $ A \cup \{e\}
            $ to basis $ B' $ such that $ e \in B' \setminus B$, and therefore,
            there is a $ f \in B \setminus B' $ such that $ B \setminus \{f\}
            \cup \{e\} $ is a basis, but $ w(f) \leq w(e) $ and therefore $ B
            \setminus \{f\} \cup \{e\} $ is also an optimal basis. 




    \end{itemize}
\end{proof}
\end{document}
