\documentclass[00_complete]{subfiles}

%\documentclass[12pt]{report}
\usepackage[utf8]{inputenc}
\usepackage{amsmath,amssymb,amsthm,gensymb,parskip,graphicx,footmisc,csquotes,enumerate,datetime2}
\usepackage[]{libertinus}
\usepackage[breaklinks]{hyperref}
\hypersetup{
  pdfauthor={Moshe Krumbein},
  colorlinks=true,
  linkcolor={black},
  filecolor={black},
  citecolor={black}, %blue
  urlcolor={black}, %blue
}
\usepackage[top=30mm,bottom=30mm,left=30mm,right=30mm]{geometry}
%\setlength{\emergencystretch}{2em} % prevent overfull lines
\providecommand{\tightlist}{%
\setlength{\itemsep}{0pt}\setlength{\parskip}{0pt}}

\renewcommand\qedsymbol{$\blacksquare$}

\theoremstyle{definition}
\newtheorem*{definition}{Definition}
\newtheorem*{theorem}{Theorem}
\newtheorem*{axiom}{Axiom}
\newtheorem*{lemma}{Lemma}

\theoremstyle{remark}
\newtheorem*{note}{Note}
\newtheorem*{symbols}{Symbol}
\newtheorem{example}{Example}[section]
\newtheorem*{claim}{Claim}
\newtheorem*{conclusion}{Conclusion}
\newtheorem*{reminder}{Reminder}

\usepackage{fancyhdr}
\usepackage[italicdiff]{physics}
\MakeOuterQuote{"}

\renewcommand{\chaptermark}[1]{\markboth{#1}{}}

\pagestyle{fancy}

\setlength{\headheight}{14.5pt}
\addtolength{\topmargin}{-2.5pt}

\fancyhf{}
\rhead{Moshe Krumbein}
\lhead{\chaptermark}
\cfoot{\thepage}
\fancyhead[R]{\chaptername~\thechapter}
\fancyhead[L]{\mbox{\leftmark}}

\usepackage[Rejne]{fncychap}
\usepackage{titling}

\makeatletter
\renewcommand{\@chapapp}{\vspace*{-100pt}\huge\thetitle}
\makeatother

\makeatletter
\newcommand{\subtitle}[1]{%
  {\center\vspace*{-60pt}%
  \linespread{1.1}\Large\scshape#1%
  \par\nobreak\vspace*{35pt}}
}
\makeatother

\newcommand{\Chapter}[2]{
    \def\n{#2}
    \setcounter{chapter}{\the\numexpr\n-1}
    \chapter{#1}
    \subtitle{\theauthor~- \thedate}
}

\DeclareMathOperator{\Ima}{Im}
\DeclareMathOperator{\Id}{Id}
\DeclareMathOperator{\cis}{cis}

\newcommand{\Mod}[1]{\ (\mathrm{mod}\ #1)}
\newcommand{\st}[0]{\;\mathrm{s.t.}\;}

\title{Algorithms}
\author{Moshe Krumbein}
\date{Spring 2023}

\begin{document}
\Chapter{Introduction}{1}
\section{What's is an Algorithm?}
An \textit{algorithm} is a formal description of an effective method of solving
a well-defined set of problems, while using limited resources (namely, time and
space).

Specifically, an algorithm matches any possible input to an output that solves a
given problem for that input.

Most of the problems that we will be focusing on are combinitorical
optimization, where the input is defined as a large finite set of potential
solutions and we have to select the minimal (or maximal) solution.

We will focus on asymptotic efficiency of the algorithm, as a function of the
size of the input.

\section{History}
One of the first descriptions of algorithms was found in ancient Babylonia. In
ancient Greece \textit{Euclid's Algorithm}, \textit{Sieve of Eratosthenes}, and
the \textit{Babylonian Square-Root Method}, were created from the third century
BCE until the first century CE.

The \textit{Hindu-Arabic numeral system} was developed by Indian mathematician
between the first and fourth centuries and was later adapted by Arabic
mathematicians by the ninth century. Most notably the digit $0$ was first used
by the turn of the seventh century by Brahmagupta.

Although first found to be used in the first century in China, \textit{Gauss'
Elimination Method} was monumental development for algorithmically solving
matrices.

In 1928, Hilbert and Ackermann proposed the \textit{Entscheidungsproblem}
(decision problem), which asks if a problem can, by way of an algorithm, can be
answered "Yes" or "No" in according to whether the input, a statement, is
universally valid. 
\end{document}
