\documentclass[00_complete]{subfiles}

\title{Mathematical Methods}
\author{Moshe Krumbein}
\date{Fall 2021}

\begin{document}
\Chapter{Derivatives and Integrals}{5}

\section{Introduction}

The \emph{integral} is the area under the curve of a graph and the
\emph{derivative} is the slope of the function. A \emph{definite integral} is
also known as the \emph{antiderivative}.

\section{Derivative}

\begin{definition}[Derivative]
\[
    \frac{\Delta s}{\Delta t} = \frac{s(t_0+\Delta t)-s(t_0)}{\Delta t}
    \quad \Delta t \to 0
\]
$$\lim\limits_{h \to 0} \frac{f(x+h)-f(x)}{h}$$
\end{definition}


\begin{symbols}[Derivative]
    $$\frac{ds}{dt}, \dot s(t), s'(t)$$
\end{symbols}

\subsection{Geometric significance of derivatives}

A function is differential at the point $x_0$ when there are no "sharp edges"
or jumps at $x_0$.

\subsection{Linear approximation}

If $x$ is close to $x_0$, then:
$$f(x) \approx f(x_0)+f'(x_0)(x-x_0)$$

\section{Integral}

The area under the graph is the approximation: (\emph{Riemann Sum})
$$\sum_{i=1}^{n}f(t_i)(\overbrace{t_i-t_{i-1}}^{\Delta t_i})$$
$$\sum_{k=1}^{n}f\left(\frac{k}{n}\right)\cdot \frac{1}{n} \to
\int_{0}^{1}f(x)dx$$

If there does exist an approximation for this sum (that is not dependent on
specific division), then we say that $f$ is \emph{integratable} on $[a,b]$:

$$\int_{a}^{b}f(t)dt$$

\section{Fundamental Theorem of Calculus}

Integration and differentiation are inverse functions of each other (under
certain conditions)

\section{Operations between derivatives}

\begin{note}
        Differentiability $\implies$ continuity
        (continuity $\nRightarrow$ differentiability)
\end{note}

\begin{gather}
    (f \pm g)' = f' \pm g' \\
    (fg)' = f'g+fg' \\
    \left(\frac{f}{g}\right)' = \frac{f'g-fg'}{g^2} \\
    (g \circ f)' = f' \cdot g'(f)
\end{gather}

\section{Derivative of the inverse function}

Given function $f$ differentiable at the point $a$:
\[
    f^{-1}(x) \cdot f(x) = x
\]
Differentiate in terms of $x$:
\[
\begin{gathered}
    (f^{-1}(f(x)))'\cdot f'(x) = 1 \\
    (f^{-1}(f(x)))' = \frac{1}{f'(x)} \\
\end{gathered}
\]

\section{Differentiating implicit functions}

\begin{example}
    $x^2+y^2-1=0$
\end{example}

We differentiate both sides in terms of $x$ and attempt to isolate $\frac{dy}{dx}$.

\[
    \begin{gathered}
        x^y-\ln y = 2x \implies x^{f(x)} - \ln(f(x))=2x \\
        x^{f(x)}\left(\ln(x)\cdot f'(x)+\frac{1}{x}f(x)\right) - \frac{f'(x)}{f(x)}=2
    \end{gathered}
\]

\section{Linear Approximation}

\begin{reminder}
    If $f$ is differentiable at the point $a$, then near $a$ exists:
    $$f(x) \approx f(a)+f'(a)(x-a)$$
\end{reminder}

\begin{example}
    $$
    \begin{gathered}
        \tan^{-1}(1.05) \\ \\
        \tan^{-1}(1) = \frac{\pi}{4} \to \quad \begin{gathered}
            a = 1 \\ x = 1.05 \\ f(x) = \tan^{-1}x \\ f'(x) = \frac{1}{1+x^2}
        \end{gathered} \\
        f(1.05) \approx \overbrace{f(1)}^{\frac{\pi}{4}} +
        \overbrace{f'(1)}^{\frac{1}{2}}(0.05) \\
        \tan^{-1}(1.05) \approx \frac{\pi}{4} + 0.025
    \end{gathered}
    $$
\end{example}

\section{Differential}

\begin{definition}
    The \emph{differential} is the \emph{linear approximation} of the change of
    a function.
\end{definition}

If $y$ is a function of $x$:
\[
    \Delta y \approx \frac{dy}{dx} \Delta x \implies
    dy = \frac{dy}{dx}dx
\]

\begin{example}
    $$
    \begin{gathered}
        y = x^2 \quad z = \tan^{-1}(y) \\
        \text{Calculate the relation between $dx, dy, dz$ around the points:}
        \\
        x=1,\;y=1,\;z=\frac{\pi}{4} \\ \\
        \frac{dy}{dx} =2x \implies \frac{dy}{dx} = 2 \\
        dy = 2dx \quad dz = \frac{1}{2}dy
    \end{gathered}
    $$
\end{example}

\section{Lagrange's Mean Value Theorem}

\begin{definition}
    If $f$ is continuous on $[a,b]$ and differentiable on $(a,b)$, then there
    exists $a<c<b$ such that:
    \[
        f'(c) = \frac{f(b)-f(a)}{b-a}
    \]
\end{definition}

\end{document}
