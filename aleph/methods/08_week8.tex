\documentclass[00_complete]{subfiles}

%\documentclass[12pt]{report}
\usepackage[utf8]{inputenc}
\usepackage{amsmath,amssymb,amsthm,gensymb,parskip,graphicx,footmisc,csquotes,enumerate,datetime2}
\usepackage[]{libertinus}
\usepackage[breaklinks]{hyperref}
\hypersetup{
  pdfauthor={Moshe Krumbein},
  colorlinks=true,
  linkcolor={black},
  filecolor={black},
  citecolor={black}, %blue
  urlcolor={black}, %blue
}
\usepackage[top=30mm,bottom=30mm,left=30mm,right=30mm]{geometry}
%\setlength{\emergencystretch}{2em} % prevent overfull lines
\providecommand{\tightlist}{%
\setlength{\itemsep}{0pt}\setlength{\parskip}{0pt}}

\renewcommand\qedsymbol{$\blacksquare$}

\theoremstyle{definition}
\newtheorem*{definition}{Definition}
\newtheorem*{theorem}{Theorem}
\newtheorem*{axiom}{Axiom}
\newtheorem*{lemma}{Lemma}

\theoremstyle{remark}
\newtheorem*{note}{Note}
\newtheorem*{symbols}{Symbol}
\newtheorem{example}{Example}[section]
\newtheorem*{claim}{Claim}
\newtheorem*{conclusion}{Conclusion}
\newtheorem*{reminder}{Reminder}

\usepackage{fancyhdr}
\usepackage[italicdiff]{physics}
\MakeOuterQuote{"}

\renewcommand{\chaptermark}[1]{\markboth{#1}{}}

\pagestyle{fancy}

\setlength{\headheight}{14.5pt}
\addtolength{\topmargin}{-2.5pt}

\fancyhf{}
\rhead{Moshe Krumbein}
\lhead{\chaptermark}
\cfoot{\thepage}
\fancyhead[R]{\chaptername~\thechapter}
\fancyhead[L]{\mbox{\leftmark}}

\usepackage[Rejne]{fncychap}
\usepackage{titling}

\makeatletter
\renewcommand{\@chapapp}{\vspace*{-100pt}\huge\thetitle}
\makeatother

\makeatletter
\newcommand{\subtitle}[1]{%
  {\center\vspace*{-60pt}%
  \linespread{1.1}\Large\scshape#1%
  \par\nobreak\vspace*{35pt}}
}
\makeatother

\newcommand{\Chapter}[2]{
    \def\n{#2}
    \setcounter{chapter}{\the\numexpr\n-1}
    \chapter{#1}
    \subtitle{\theauthor~- \thedate}
}

\DeclareMathOperator{\Ima}{Im}
\DeclareMathOperator{\Id}{Id}
\DeclareMathOperator{\cis}{cis}

\newcommand{\Mod}[1]{\ (\mathrm{mod}\ #1)}
\newcommand{\st}[0]{\;\mathrm{s.t.}\;}


\title{Mathematical Methods}
\author{Moshe Krumbein}
\date{Fall 2021}

\begin{document}
\setcounter{chapter}{7}

\chapter{Definite Integrals}
\subtitle{\theauthor~- \thedate}

\section{Introduction}

We define $\int_{a}^{b}f(x)\dd{x}$ to be the area underneath the graph of
$f(x)$, above the $x$-axis on the interval $[a,b]$.

\section{The Fundamental Theorem of Calculus}
\begin{definition}[First Fundamental Theorem of Calculus]
    Under certain conditions:
    $$\dv{x}\underbrace{\left(\int_{a}^{x}f(t)\dd{t}\right)}_{F(x)} = f(x)$$
    \begin{proof}
        $$
        \begin{gathered}
            F'(x)= \lim\limits_{h \to 0} \frac{f(x+h)-F(x)}{h} \\
            = \lim\limits_{h \to 0} \frac{\int_{x}^{x+h}f(t)\dd{t}}{h} =
            \frac{h\cdot f(x) + \text{negligible}}{h} = f(x)
        \end{gathered}
        $$
    \end{proof}
\end{definition}
\begin{example}
    $$
        \dv{x}\left(\int_{1}^{x}\frac{\sin t}{t}\dd{t}\right) = \frac{\sin x}{x}
    $$
\end{example}
\begin{example}
    $$
        \dv{x}\left(\int_{1}^{x^2}\frac{\sin t}{t}\dd{t}\right) = \frac{\sin
        (x^2)}{x^2}\cdot 2x
    $$
\end{example}
\begin{example}
    $$
        \dv{x}\left(\int_{5x}^{e^x} e^{-t^2}\dd{t}\right) =
        \dv{x}\left(\int_{0}^{e^x}e^{-t^2}\dd{t}-\int_{0}^{5x}e^{-t^2}\dd{t}\right)
        = e^{-(e^x)^2}\cdot e^x - e^{-(5x)^2}\cdot 5
    $$
\end{example}
\begin{definition}[Upgraded Theorem]
    $$\dv{c}\left(\int_{g(x)}^{h(x)}f(t)\dd{t}\right)=f(h)\cdot h'-f(g)\cdot g'$$
\end{definition}
\begin{definition}[Second Fundamental Theorem of Calculus]
    Also known as the \emph{Newton-Leibniz axiom.}

    If $f$ is continuous and $F$ is an anti derivative of $f$, then:
    $$\int_{a}^{b}f(x)\dd{x}=F(b)-F(a)$$
    \begin{proof}
        According to the first theorem:
        $$\int_{a}^{x}f(t)\dd{t}=F(x)+c$$
        If we set $x=a$:
        $$0=\int_{a}^{a}f(t)\dd{t}=F(a)+c \implies C = -F(a)$$
    \end{proof}
\end{definition}
Essentially, there are a few things to keep in mind:
\begin{itemize}
    \item When doing substitution, it is important to also adjust the limits of
        the integral accordingly (this only works in the substitution is
        \emph{one-to-one}.)
    \item Often we can solve definite integrals simply by seeing if we
        substitute the limits of the integral are equal to each other, which
        means the integral is equal to zero, without having to work out the
        anti derivative at all.
\end{itemize}
\section{Improper Integrals}
There two types of improper integrals:
\begin{enumerate}
    \item Divergent improper integrals, e.g.:
        $$\int_{3}^{\infty}x\dd{x}=\infty \qquad
        \int_{0}^{1}\frac{1}{x}\dd{x}=\infty$$
    \item Convergent improper integrals, e.g.:
        $$\int_{3}^{\infty}\frac{1}{x^2}\dd{x}=\frac{1}{3}$$
\end{enumerate}
\begin{note}
    $$\int_{-1}^{1}\frac{\sin x}{x}\dd{x}$$
    This is a \emph{proper integral} because the function is \emph{bounded} on
    the interval even though there is an undefined point at $x=0$.
\end{note}
\begin{theorem}
$$\int_{1}^{\infty}\frac{1}{x^a}\dd{x}$$
If $a>1$, the integral \emph{converges}, and if $a\leq 1$, the integral
\emph{diverges}.

However:
$$\int_{0}^{1}\frac{1}{x^a}\dd{x}$$
If $a<1$, the integral \emph{converges}, and if $a\geq 1$, the integral
\emph{diverges}.
$$\int_{c}^{d}\frac{1}{(x-x_0)^a}\dd{x}, \quad c \leq x_0\leq d$$
If $a<1$, the integral \emph{converges}, and if $a\geq 1$, the integral
\emph{diverges}.
\end{theorem}
\end{document}
