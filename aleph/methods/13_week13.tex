\documentclass[00_complete]{sub files}

%\documentclass[12pt]{report}
\usepackage[utf8]{inputenc}
\usepackage{amsmath,amssymb,amsthm,gensymb,parskip,graphicx,footmisc,csquotes,enumerate,datetime2}
\usepackage[]{libertinus}
\usepackage[breaklinks]{hyperref}
\hypersetup{
  pdfauthor={Moshe Krumbein},
  colorlinks=true,
  linkcolor={black},
  filecolor={black},
  citecolor={black}, %blue
  urlcolor={black}, %blue
}
\usepackage[top=30mm,bottom=30mm,left=30mm,right=30mm]{geometry}
%\setlength{\emergencystretch}{2em} % prevent overfull lines
\providecommand{\tightlist}{%
\setlength{\itemsep}{0pt}\setlength{\parskip}{0pt}}

\renewcommand\qedsymbol{$\blacksquare$}

\theoremstyle{definition}
\newtheorem*{definition}{Definition}
\newtheorem*{theorem}{Theorem}
\newtheorem*{axiom}{Axiom}
\newtheorem*{lemma}{Lemma}

\theoremstyle{remark}
\newtheorem*{note}{Note}
\newtheorem*{symbols}{Symbol}
\newtheorem{example}{Example}[section]
\newtheorem*{claim}{Claim}
\newtheorem*{conclusion}{Conclusion}
\newtheorem*{reminder}{Reminder}

\usepackage{fancyhdr}
\usepackage[italicdiff]{physics}
\MakeOuterQuote{"}

\renewcommand{\chaptermark}[1]{\markboth{#1}{}}

\pagestyle{fancy}

\setlength{\headheight}{14.5pt}
\addtolength{\topmargin}{-2.5pt}

\fancyhf{}
\rhead{Moshe Krumbein}
\lhead{\chaptermark}
\cfoot{\thepage}
\fancyhead[R]{\chaptername~\thechapter}
\fancyhead[L]{\mbox{\leftmark}}

\usepackage[Rejne]{fncychap}
\usepackage{titling}

\makeatletter
\renewcommand{\@chapapp}{\vspace*{-100pt}\huge\thetitle}
\makeatother

\makeatletter
\newcommand{\subtitle}[1]{%
  {\center\vspace*{-60pt}%
  \linespread{1.1}\Large\scshape#1%
  \par\nobreak\vspace*{35pt}}
}
\makeatother

\newcommand{\Chapter}[2]{
    \def\n{#2}
    \setcounter{chapter}{\the\numexpr\n-1}
    \chapter{#1}
    \subtitle{\theauthor~- \thedate}
}

\DeclareMathOperator{\Ima}{Im}
\DeclareMathOperator{\Id}{Id}
\DeclareMathOperator{\cis}{cis}

\newcommand{\Mod}[1]{\ (\mathrm{mod}\ #1)}
\newcommand{\st}[0]{\;\mathrm{s.t.}\;}


\title{Mathematical Methods}
\author{Moshe Krumbein}
\date{Fall 2021}

\begin{document}
\Chapter{Integration on Surfaces}{13}
\section{Introduction}

\begin{reminder}
    A \emph{surface} on $\mathbb{R}^3$ is the \emph{image} of a function:
        $$\underline r: D\to \mathbb{R}^3 \qquad r(u,v)=\begin{pmatrix}
            x(u,v) \\ y(u,v) \\ z(u,v)
        \end{pmatrix}$$
    The symbolize the surface as $\Sigma$ ($\Sigma = \Im \underline r$).
\end{reminder}
\begin{example}
    A sphere with a radius of $3$:
    $$\underline r (u,v) = \begin{pmatrix}
        3 \sin v \cos u \\ 3 \sin v \sin u \\ 3 \cos v
    \end{pmatrix} \quad D: 0 \leq u \leq 2\pi, 0 \leq v \leq \pi$$
    A small rectangle whose area is $\delta_u \cdot \delta_v$ becomes similar
    to a parallelogram on $\Sigma$:
    $$\underbrace{\|\underline r_u \times \underline r_v\|}_{\|\underline N\|}\delta_u\delta_v$$
\end{example}
\begin{symbols}
    \begin{align*}
        \dd{S}&=\|\underline r_u \times \underline r_v\|\dd{u}\dd{v} \\
        \dd{\underline S} &= (\underline r_u \times \underline r_v)\dd{u}\dd{v}
        \\
        \underline N &= \underline r_u \times \underline r_v \\
        \underline{\hat N} &= \frac{\underline r_u \times \underline
        r_v}{\|\underline r_u \times \underline r_v\|} \\
    \end{align*}
\end{symbols}
\begin{example}[Half-Torus]
    \begin{gather*}
        r(u,v)=\begin{pmatrix}
            (2+\cos u) \cos v \\
            (2+\cos u) \sin v \\
            \sin u
        \end{pmatrix} \quad D: 0 \leq u \leq \pi, 0 \leq v \leq 2\pi \\
        \underline r_u=\begin{pmatrix}
            -\sin u \cos v \\ -\sin u \sin v \\ \cos u
        \end{pmatrix} \quad \underline r_v = \begin{pmatrix}
            -(2+\cos u)\sin v \\ (2+\cos u)\cos v \\ 0
        \end{pmatrix} \\
        \underline N = \underline r_u \times \underline r_v = (2+\cos u) \cdot \underline r_u
        \times \begin{pmatrix}
            -\sin v \\ \cos v \\ 0
        \end{pmatrix} = (2+\cos u )\begin{pmatrix}
            -\cos u \cos v \\ - \cos u \sin v \\ -\sin u
        \end{pmatrix} \\
        \|\underline N\|=\|\underline r_u \times \underline r_v\| = (2+\cos
        u)\sqrt{(-\cos u \cos v)^2 + (-\cos u \sin v)^2+(-\sin u)^2} \\
        =2+\cos u \\
        \underline{\hat N} = \frac{\underline N}{\|\underline N\|} =
        -\begin{pmatrix}
            \cos u \cos v \\ \cos u \sin v \\ \sin u
        \end{pmatrix} \\
        \dd{S} = \|\underline r_u \times\underline r_v\|\dd{u}\dd{v}=(2+\cos
        u)\dd{u}\dd{v} \\
        \dd{\underline S}= (\underline r_u \times \underline r_v)\dd{u}\dd{v} =
        -(2+\cos u)\begin{pmatrix}
            \cos u \cos v \\ \cos u \sin v \\ \sin u
        \end{pmatrix} \dd{u}\dd{v}
    \end{gather*}
\end{example}
\section{Integration on Surfaces}
\begin{reminder}
    We learned how to do integration on a curve:
    $$\underline r: \mathbb{R} \to \mathbb{R}^3$$
    and integration on a \emph{scalar field} $\phi$:
    \begin{gather*}
    \int\limits_C\phi\dd{s}=\int_{a}^{b}\phi(r(t))\|r'(t)\|\dd{t} \\
    \text{\small (which represents either \emph{mass} or \emph{area})}
    \end{gather*}
    and integration on a \emph{vector field} $\underline f$:
    \begin{gather*}
    \int\limits_Cf\cdot \dd{S}=\int_{a}^{b}f(\underline r(t))\cdot r'(t)\dd{t}
    \\
    \text{\small (which represents \emph{work})}
    \end{gather*}
\end{reminder}
Integration on a \emph{scalar field} $\phi$:
\begin{gather*}
\iint\limits_\Sigma \phi \dd{S} = \iint\limits_D \phi(r(u,v))\|r\underline
r_u \times \underline r_v\|\dd{u}\dd{v} \\
\text{\small (which represents \emph{mass})}
\end{gather*}
\begin{note}
    If the \emph{density} $\phi=1$, then the integral represents the
    \emph{surface area}.
\end{note}
Integration on a \emph{vector field} $\underline f$:
\begin{gather*}
\iint\limits_\Sigma f \cdot \dd{\underline S} = \iint\limits_D f(\underline
r(u,v))\cdot (\underline r_u \times \underline r_v)\dd{u}\dd{v} \\
\text{\small (which represents \emph{flux})}
\end{gather*}
\begin{example}
    Find the \emph{surface area} of our half-torus:

    We will take $\phi=1$ and calculate:
    $$\iint\limits_\Sigma 1 \cdot \dd{S}=\int_{0}^{2\pi}\int_{0}^{\pi}(2+\cos
    u)\dd{u}\dd{v}=2\pi[2u+\sin u]\Bigr|_0^\pi=4\pi^2$$
\end{example}
\begin{example}
    Integrate:
    \begin{gather*}
    \iint\limits_\Sigma \underline r \cdot \dd{\underline
    S}=\int_{0}^{2\pi}\int_{0}^{\pi}
    \underbrace{
    \begin{pmatrix}
        (2+\cos u) \cos v \\ (2+\cos u) \sin v \\ \sin u
    \end{pmatrix}}_{\underline r} \cdot \underbrace{-(2+\cos u)\begin{pmatrix}
        \cos u \cos v \\ \cos u \sin v \\ \sin u
    \end{pmatrix} \dd{u}\dd{v}}_{\dd{\underline S}} \\
    =\int_{0}^{2\pi}\int_{0}^{\pi}(2+\cos u)((2+\cos u)\cdot (-\cos u
    \cos^2v-\cos u \sin^2 v)-\sin^2 u) \dd{u}\dd{v} \\
    =\int_{0}^{2\pi}\int_{0}^{\pi}(2+\cos u)((2+\cos u)(-\cos u)-\sin^2
    u)\dd{u} \dd{v} \\
    =-\int_{0}^{2\pi}\int_{0}^{\pi}(2+\cos u)(2\cos u +1)
    \dd{u}\dd{v}=-2\pi\int_{0}^{\pi}(2\cos^2u +5\cos u + 2)\dd{u} \\
    =-6\pi^2
    \end{gather*}
\begin{note}
    Here, the \emph{normal} is facing "inwards," into the
    half-torus, and which mean the volume of liquid that
    \emph{enters} the half-torus is $6\pi^2$.
\end{note}
\end{example}
\begin{note}
    If the \emph{vector field} $\underline f$ is a \emph{constant unit vector},
    then:
    $$\iint\limits_\Sigma \underline f \cdot \dd{\underline S}$$
    represents the area of the \emph{projection} of $\Sigma$ on the plane
    perpendicular to $\underline f$.
\end{note}
\begin{example}
    Calculate the \emph{projection} of a half-torus onto the $xy$ plane:
    \begin{gather*}
    \iint\limits_\Sigma\begin{pmatrix}
        0 \\ 0 \\ 1
    \end{pmatrix} \cdot \dd{\underline S} = \int_{0}^{2\pi}\int_{0}^{\pi} \begin{pmatrix}
        0 \\ 0 \\ 1
    \end{pmatrix} \cdot -(2+\cos u)\begin{pmatrix}
        \cos u \cos v \\ \cos u \sin v \\ \sin u
    \end{pmatrix} \dd{u}\dd{v} \\
    =\int_{0}^{2\pi}\int_{0}^{\pi}(2+\cos u)(-\sin u)\dd{u}\dd{v} =
    2\pi\int_{0}^{\pi} (-2\sin u -\sin u \cos u)\dd{u} = |-8\pi| \\=8\pi
    \end{gather*}
\end{example}
\end{document}
