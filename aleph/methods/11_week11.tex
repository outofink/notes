\documentclass[00_complete]{subfiles}

%\documentclass[12pt]{report}
\usepackage[utf8]{inputenc}
\usepackage{amsmath,amssymb,amsthm,gensymb,parskip,graphicx,footmisc,csquotes,enumerate,datetime2}
\usepackage[]{libertinus}
\usepackage[breaklinks]{hyperref}
\hypersetup{
  pdfauthor={Moshe Krumbein},
  colorlinks=true,
  linkcolor={black},
  filecolor={black},
  citecolor={black}, %blue
  urlcolor={black}, %blue
}
\usepackage[top=30mm,bottom=30mm,left=30mm,right=30mm]{geometry}
%\setlength{\emergencystretch}{2em} % prevent overfull lines
\providecommand{\tightlist}{%
\setlength{\itemsep}{0pt}\setlength{\parskip}{0pt}}

\renewcommand\qedsymbol{$\blacksquare$}

\theoremstyle{definition}
\newtheorem*{definition}{Definition}
\newtheorem*{theorem}{Theorem}
\newtheorem*{axiom}{Axiom}
\newtheorem*{lemma}{Lemma}

\theoremstyle{remark}
\newtheorem*{note}{Note}
\newtheorem*{symbols}{Symbol}
\newtheorem{example}{Example}[section]
\newtheorem*{claim}{Claim}
\newtheorem*{conclusion}{Conclusion}
\newtheorem*{reminder}{Reminder}

\usepackage{fancyhdr}
\usepackage[italicdiff]{physics}
\MakeOuterQuote{"}

\renewcommand{\chaptermark}[1]{\markboth{#1}{}}

\pagestyle{fancy}

\setlength{\headheight}{14.5pt}
\addtolength{\topmargin}{-2.5pt}

\fancyhf{}
\rhead{Moshe Krumbein}
\lhead{\chaptermark}
\cfoot{\thepage}
\fancyhead[R]{\chaptername~\thechapter}
\fancyhead[L]{\mbox{\leftmark}}

\usepackage[Rejne]{fncychap}
\usepackage{titling}

\makeatletter
\renewcommand{\@chapapp}{\vspace*{-100pt}\huge\thetitle}
\makeatother

\makeatletter
\newcommand{\subtitle}[1]{%
  {\center\vspace*{-60pt}%
  \linespread{1.1}\Large\scshape#1%
  \par\nobreak\vspace*{35pt}}
}
\makeatother

\newcommand{\Chapter}[2]{
    \def\n{#2}
    \setcounter{chapter}{\the\numexpr\n-1}
    \chapter{#1}
    \subtitle{\theauthor~- \thedate}
}

\DeclareMathOperator{\Ima}{Im}
\DeclareMathOperator{\Id}{Id}
\DeclareMathOperator{\cis}{cis}

\newcommand{\Mod}[1]{\ (\mathrm{mod}\ #1)}
\newcommand{\st}[0]{\;\mathrm{s.t.}\;}


\title{Mathematical Methods}
\author{Moshe Krumbein}
\date{Fall 2021}

\begin{document}
\Chapter{Multivariable Functions \texorpdfstring{$\mathbb{R}^{\lowercase{m}}
\to \mathbb{R}^{\lowercase{n}}$}{Rm to Rn}}{11}

\section{Introduction}
We will be examining functions of the form:
\begin{gather*}
    f(x_1,\dots,x_m)=\begin{pmatrix}
        f_1(x_1,\dots,x_m) \\ \vdots \\
        f_n(x_1\cdots,x_m)
    \end{pmatrix}
\end{gather*}
\begin{example}
    \begin{gather*}
        f(r,\theta)=\binom{r\cos \theta}{r \sin \theta}
        \quad \vline \quad
        f(r, \theta, \phi)=\begin{pmatrix}
            r \cos \theta \sin \phi \\
            r \sin \theta \sin \phi \\
            r \cos \phi
        \end{pmatrix}
    \end{gather*}
\end{example}
\begin{definition}[Derivative]
    For the partial derivative:
    \begin{gather*}
        \pdv{f}{x_i} = \begin{pmatrix}
            \pdv{f_1}{x_i} \\ \vdots \\ \pdv{f_n}{x_i}
        \end{pmatrix}
    \end{gather*}
    In general:
    \begin{gather*}
        \dd{\underline f}=\begin{pmatrix}
            \dd{f_1} \\ \vdots \\ \dd{f_n}
        \end{pmatrix} = \begin{pmatrix}
            \pdv{f_1}{x_1}\dd{x_1} +  \pdv{f_1}{x_2}\dd{x_2} +\dots+
            \pdv{f_1}{x_m}\dd{x_m} \\
            \vdots \\
            \pdv{f_n}{x_1}\dd{x_1} +  \pdv{f_n}{x_2}\dd{x_2} +\dots+
            \pdv{f_n}{x_m}\dd{x_m} \\
        \end{pmatrix} = \pdv{\underline f}{x_1}\dd{x_1}+\dots+\pdv{\underline
        f}{x_m}\dd{x_m} \\
    D\underline f = \begin{pmatrix}
    \pdv{f_1}{x_1}&\dots&\pdv{f_1}{x_m} \\
    \vdots && \vdots \\
    \pdv{f_n}{x_1}&\dots&\pdv{f_n}{x_m}
    \end{pmatrix} \quad
    D \underline f \cdot \begin{pmatrix}
    \dd{x_1} \\ \vdots \\ \dd{x_m}
    \end{pmatrix} = \dd{\underline f}
    \end{gather*}
\end{definition}
\begin{example}
    \begin{gather*}
        \underline f (r \theta)= \binom{r \cos \theta}{r \sin \theta} \quad
        \pdv{\underline f}{r} = \binom{\cos \theta}{\sin \theta} \quad
        \pdv{\underline f}{\theta}=\binom{-r\sin \theta}{r \cos \theta} \\
        \dd{\underline f} =\pdv{\underline f}{r}\dd{r}+\pdv{\underline
        f}{\theta}\dd{\theta} =
        \binom{\cos \theta}{\sin \theta}\dd{r}+
        \binom{-r\sin \theta}{r\cos \theta}\dd{\theta} \\
        D\underline f= \begin{pmatrix}
            \cos \theta & -r\sin \theta \\
            \sin \theta & r \cos \theta
        \end{pmatrix} \\
    \end{gather*}
\end{example}
\begin{example} We will consider the following cases
    \begin{enumerate}
        \item $m=1$
            $$f: \mathbb{R} \to \mathbb{R}^n$$
        Describes a path in $\mathbb{R}^n$ (Chapter 9).
        \item $n=1$
            $$f: \mathbb{R}^m \to \mathbb{R}$$
        What we did in Chapter 10.
        \item $m=n$ ($2$ or $3$)

            There are two different ways to examine these functions:
            \begin{enumerate}
                \item Substitution: $f(r,\theta)=\binom{r\cos\theta}{r\sin\theta}$
                \item Using \emph{vector fields}: each point in
                    $\mathbb{R}^n$ can be represented as an $n$-dimensional
                    vector.
            \end{enumerate}
        \item $m=2,n=3$

        Parameterization of a \emph{surface} in $\mathbb{R}^3$
        \begin{example}
            \begin{gather*}
           f:\mathbb{R}^2\to R: \qquad\qquad f:\mathbb{R}^2\to \mathbb{R}^3: \\
            \begin{gathered}
                f(x,y)=x^2+y^2
            \end{gathered} \quad \vline \quad \begin{gathered}
                f(x,y)=\begin{pmatrix}
                    x\\y\\x^2+y^2
                \end{pmatrix} \\
                f(\rho,\theta) =\begin{pmatrix}
                    \rho\cos\theta\\\rho\sin\theta\\\rho^2
                \end{pmatrix}
            \end{gathered}
            \end{gather*}
        \end{example}
    \end{enumerate}
\end{example}
\section{Surfaces in \texorpdfstring{$\mathbb{R}^3$}{R3}}
Or in other words, the \emph{image} of functions $f: \mathbb{R}^2\to
\mathbb{R}^3$.
\begin{example}[Paraboloid]
    \begin{gather*}
        f(\rho,\theta)=\begin{pmatrix}
            \rho\cos\theta\\\rho\sin\theta\\\rho^2
        \end{pmatrix} \quad
        f_{\rho}=\begin{pmatrix}
            \cos\theta\\\sin\theta\\ 2\rho
        \end{pmatrix} \quad f_{\theta}=\begin{pmatrix}
            -\rho\sin\theta\\\rho\cos\theta\\ 0
        \end{pmatrix} \\
        D \underline f = \begin{pmatrix}
            \cos\theta &-\rho\sin\theta \\
            \sin\theta &\rho\cos\theta \\
            2\rho & 0
        \end{pmatrix} \\
        \dd{\underline f} = f_{\rho}\dd{\rho} + f_{\theta}\dd{\theta} =
        \begin{pmatrix}
            \cos\theta\\\sin\theta\\ 2\rho
        \end{pmatrix}\dd{\rho} + \begin{pmatrix}
            -\rho\sin\theta \\ \rho\cos\theta\\ 0
        \end{pmatrix}\dd{\theta}
    \end{gather*}
    $\dd{\underline f}$ is how much $\underline f$ changes when there is a
    small change in $\rho$ or $\theta$. We want to find the \emph{area} of
    $\dd{\underline f}$:
    $$\|h f_\rho \times k f_\theta\|=hk\|f_\rho\times f_\theta\|$$
\end{example}
Basically we see for small unit on the plane $u,v$ that the area is surface is
approximately the area on the plane $uv$ times $\|f_u \times f_v\|$.
\begin{definition}
    A point such that $f_u \parallel f_v$ is called a \emph{singular point}.
\end{definition}
\section{Tangent plane on the surface \texorpdfstring{$f(u,v)$}{f(u,v)}}

It's easy to see that $f_u,f_v$ span the plane and therefore the \emph{normal}
of the plane is:
$$N=f_u \times f_v \quad \hat N = \frac{f_u \times f_v}{\|f_u \times f_v\|}$$
\begin{note}
    There are many parameterizations for each plane, for example:
    $$f(u,v)=\begin{pmatrix}
        u\\v\\u^2+v^2
    \end{pmatrix} \qquad g(u,v) = \begin{pmatrix}
        u\cos v \\ u \sin v \\ u^2
    \end{pmatrix}$$
\end{note}
\section{Change of Coordinates and Jacobian}
\begin{reminder}
    $f: \mathbb{R}^m \to \mathbb{R}n$:
    \begin{itemize} \tightlist
        \item Paths ($m=1$)
        \item $f: \mathbb{R}^2 \to R$ (all of Chapter 10)
        \item Surfaces ($m=2,n=3$)
        \item $m=n$

            Either changing coordinates or \emph{vector fields}
    \end{itemize}
\end{reminder}
Suppose $f(x,y)$ is differentiable and:
$$f(x,y)=\binom{u(x,y)}{v(x,y)}$$
To find the area of a small section from $x,y$, we see that we can approximate
with the area of the parallelogram of the vectors $u,v$, which we know can
calculated by the \emph{determinant}:
$$\begin{vmatrix}
    hu_x&ku_y \\ hv_x&kv_y
\end{vmatrix}$$
\begin{note}
    This matrix is the derivative $Df$ of $f$.
\end{note}
\begin{definition}[Jacobian]
    This determinant is called the \emph{Jacobian} symbolized by
    $\displaystyle \pdv{(u,v)}{(x,y)}$.
\end{definition}
We see that the area on the plane  $u,v$ is approximately the area on the plane
$x,y$ times $\displaystyle\left|\pdv{(u,v)}{(x,y)}\right|$.
\begin{example}
    \begin{gather*}
    f(r,\theta)=\binom{r\cos \theta}{r\sin\theta} \\ \\
    J = \left|\pdv{(x,y)}{(r,\theta)}\right|=\begin{vmatrix}
        \cos\theta &-r\sin\theta \\
        \sin\theta &r\cos\theta
    \end{vmatrix} = r(\cos^2\theta+\sin^2\theta)=r
    \end{gather*}
\end{example}
\begin{note}
    This is the same for working in three dimensions.
\end{note}
\begin{example}
   \begin{gather*}
       f(r,\theta,\phi) = \begin{pmatrix}
           r\cos\theta\sin\phi \\
           r\sin\theta\sin\phi \\
           r\cos\phi
       \end{pmatrix} \\ \\
       \left|\pdv{(x,y,z)}{(r,\theta,\phi)}\right|=\left|\begin{pmatrix}
           \vline & \vline & \vline \\
           f_r & f_\theta & f_\phi \\
           \vline & \vline & \vline
       \end{pmatrix} \right| = r^2\sin\phi
   \end{gather*}
\end{example}
\section{Operations on Vector and scalar Fields}
Suppose we have a \emph{scalar field} $\phi:\mathbb{R}^n \to \mathbb{R}$ and
a \emph{vector field} $f:\mathbb{R}^n\to \mathbb{R}^n$.
\begin{enumerate}
    \item Gradient:
        $$\grad \phi = \begin{pmatrix}
        \phi_x \\ \phi_y \\ \phi_z
    \end{pmatrix}$$
    \item Divergence:
        $$\Div f = \div f= \pdv{f_1}{x_1}+\pdv{f_2}{x_2} + \dots + \pdv{f_n}{x_n}$$
    \item Curl:

        If $f$ is a \emph{vector field} in $\mathbb{R}^3$ then:
        $$\Curl f = \curl f = \begin{vmatrix}
            i & \pdv{x} & f_x \\
            j & \pdv{y} & f_y \\
            k & \pdv{z} & f_z
        \end{vmatrix}$$
\end{enumerate}
\begin{example}
    $$\phi(x,y,z)=x^2e^{2x+3z}$$
    \begin{gather*}
        \Grad \phi = \grad \phi = \begin{pmatrix}
        2xy + 2e^{2x+3z} \\
        x^2 \\
        3e^{2x+3z}
        \end{pmatrix}
    \end{gather*}
\end{example}
\begin{example}
    $$f(x,y,z)=\begin{pmatrix}
        x^2y \\ yz \\ z
    \end{pmatrix}$$
    \begin{gather*}
        \Div f = \div f = 2xy + z + 1 \\
        \Curl f = \begin{vmatrix}
            i & \pdv{x} & x^2y \\
            j & \pdv{y} & yz \\
            k & \pdv{z} & z
        \end{vmatrix} = \begin{pmatrix}
            -y \\ 0 \\ -x^2
        \end{pmatrix}
    \end{gather*}
\end{example}
\subsection{Characteristics of \texorpdfstring{$\Div$, $\Grad$, and
$\Curl$}{div, grad, and curl}}
\begin{enumerate}
    \item $\grad \cdot (\curl A) = 0$

        $\Div(\Curl A)=0$
\end{enumerate}
\section{Integrating a Scalar Field Along a Curve}
Let $f: \mathbb{R}^2 \to \mathbb{R}$ be a \emph{vector field} and $\underline
r: [a,b] \to \mathbb{R}^2$ be the \emph{curve} $C$.
\begin{gather*}
\int\limits_cf(x,y)\underbrace{\dd{s}}_{\text{small section of } C} \qquad
\dd{s} = \|\underline{r}'(t)\|\dd{t} \\
\int\limits_cf\dd{s}=\int_a^b f(\underline{r}(t))\|\underline{r}'(t)\|\dd{t}
\end{gather*}
\begin{example}
    $$f(x,y)=y, \quad C= \text{ the upper half of the unit circle}$$
    \begin{gather*}
        r(t)=\binom{\cos t}{\sin t}_{0 \leq t \leq \pi} \quad
        r'(t) = \binom{-\sin t}{\cos t}_{0 \leq t \leq \pi} \quad \|r'(t)\|=1 \\
        \int\limits_cy\dd{s} = \int_0^\pi \sin t \cdot 1 \dd{t} = (-\cos
        t)\Bigr|_0^\pi = 2
    \end{gather*}
\end{example}
\section{Integrating a Vector Field Along a Curve}
Given a \emph{vector field} (power field) $\underline f$ and \emph{curve} $C$.
The work done by $F$ on $C$ is:
$$\int\underline f \cdot \dd{\underline S}$$
\begin{example}
    $$f(x,y)=\binom{x^2y}{y-2x}$$
    With the \emph{curve} $C$ on the section from $(1,1)$ to $(3,7)$.
    \begin{gather*}
        r(t)=\binom{1+2t}{1+6t} \\
        \int\limits_c \underline f \cdot \dd{\underline S} = \int_a^b f(r(t)) \cdot
        r'(t)\dd{t} \\
        = \int_0^1\binom{(1+2t)^2(1+6t)}{1+6t-2(t+2t)}\cdot\binom{2}{6}\dd{t}
        = \int_0^1 2(t+2t)^2(1+6t)+6(1+6t-2(1+2t))\dd{t}
    \end{gather*}
\end{example}
Another form:
\begin{gather*}
    f(x,y)=\binom{f_1(x,y)}{f_2(x,y)} \quad r(t)=\binom{x(t)}{y(t)} \\
    \int\limits_c \underline f \cdot \dd{\underline s} =
    \int\limits_c\binom{f_1}{f_2} \cdot \binom{x'(t)}{y'(t)}\dd{t} =
    \int\limits_c f_1 \underbrace{x'(t)\dd{t}}_{\dd{x}} + f_2\underbrace{y'(t)\dd{t}}_{\dd{y}}
\end{gather*}
\begin{theorem}
    If $\phi$ is a \emph{scalar vector}, then:
    $$\int\limits_c\grad \phi \cdot \dd{\underline S} = \phi(B)-\phi(A)$$
    Regardless of the path.
\end{theorem}
\section{Examples}

\begin{example}
    $$f(u,v)=\begin{pmatrix}
        uv \\ u + v \\ u^2 + v^2
    \end{pmatrix}$$
    For the point $(u,v) = (1,2)$
    \begin{enumerate}
        \item Normal:
            \begin{gather*}
                N=f_u \times f_v \\
                N = \begin{pmatrix}
                    v \\ 1 \\ 2u
                \end{pmatrix} \times \begin{pmatrix}
                    u \\ 1 \\ 2v
                \end{pmatrix} = \begin{pmatrix}
                    2v -2u \\ 2u^2-2v^2 \\ v-u
                \end{pmatrix}
            \end{gather*}
            Substituting in our point:
            \begin{gather*}
                N= \begin{pmatrix}
                    2\\1\\2
                \end{pmatrix} \times \begin{pmatrix}
                    1\\1\\4
                \end{pmatrix} = \begin{pmatrix}
                    2 \\-6\\1
                \end{pmatrix} \\
                \hat N = \frac{1}{\sqrt{41}}\begin{pmatrix}
                    2\\-6\\1
                \end{pmatrix}
            \end{gather*}
        \item Tangent plane:
            \begin{gather*}
                a(x-x_0)+b(y-y_0)+c(z-z_0)=0 \\
                2(x-2)-6(y-3)+1(z-5)=0 \\
            \end{gather*}
        \item Linear approximation:
            \begin{gather*}
                \dd{f}=f_u\dd{u}+ f_v\dd{v} = \begin{pmatrix}
                    2\\1\\2
                \end{pmatrix}(u-1) + \begin{pmatrix}
                    1\\1\\4
                \end{pmatrix}(v-2) \\
                f(u,v) \approx f_0 + \dd{f} = \begin{pmatrix}
                    2\\3\\5
                \end{pmatrix}+\begin{pmatrix}
                    2\\1\\2
                \end{pmatrix}(u-1)+\begin{pmatrix}
                    1\\1\\4
                \end{pmatrix}(v-2)
            \end{gather*}
    \end{enumerate}
\end{example}
\begin{example}
    Find the \emph{Jacobean} of:
    $$\binom{uv}{\frac{u}{v}}$$
    \begin{gather*}
        J=\begin{vmatrix}
            f_{1u} & f_{1v} \\ f_{2u} & f_{2v}
        \end{vmatrix} = \begin{vmatrix}
            v & u \\ \frac{1}{v} & -\frac{u}{v^2}
        \end{vmatrix} = -2\frac{u}{v}
    \end{gather*}
\end{example}
\begin{example}
    $\int\limits_cxy\dd{s}$, on the upper half a circle with the radius of $2$
    centered at the origin.
    \begin{gather*}
        r(t) = \binom{2\cos t}{2\sin t}_{0 \leq t \leq \pi} \\
        \int_{0}^{\pi}2\cos t \cdot 2 \sin t \cdot \underbrace{\left\|\binom{-2\sin
        t}{2\cos t}\right\|}_{2}\dd{t} = 8 \int_{0}^{\pi}\sin t \cos t \dd{t}=0
    \end{gather*}
\end{example}
\begin{example}
    $\int\limits_cx^2\dd{y}+y\dd{x}$, on the fourth quadrant on a circle of radius $3$
    centered at the origin, going clockwise.
    \begin{gather*}
        r(t)=\binom{3\sin t}{3\cos t}_{\frac{\pi}{2} \leq t \leq \pi} \left(\text{or }
        r(t)=\binom{3\cos(-t)}{3\sin(-t)}_{0\leq t\leq \frac{\pi}{2}}\right) \\
        \int_{\frac{\pi}{2}}^{\pi}(
            \underbrace{(3\sin t)^2}_{x^2}
            \underbrace{(-3\sin t)}_{y'(t)}+
            \underbrace{(3\cos t)}_{y}
            \underbrace{(3\cos t)}_{x'(t)}
        )\dd{t}
    \end{gather*}
\end{example}
\begin{example}
        $\int\limits_c\begin{pmatrix}
            \cos z \\ z \\y-x\sin z
        \end{pmatrix}\dd{\underline r}$ on the curve $(2\cos t,t^2,\cos(2t))$,
        $0 \leq t \leq \pi$.
        \begin{gather*}
            \int_{0}^{\pi}\begin{pmatrix}
                \cos(\cos 2t) \\ \cos 2t \\ t^2-2\cos \sin(\cos 2t)
            \end{pmatrix} \begin{pmatrix}
                -2\sin t \\ 2t \\-2\sin 2t
            \end{pmatrix}\dd{t}
        \end{gather*}
        This integral is difficult to do, so instead we will look for a
        \emph{potential} function $\phi(x,y,z)$ such that:
        \begin{align*}
            \phi_x= \cos z &\implies \phi=\int \cos z \dd{x} = x\cos z + c(y,z) \\
            \phi_y= z &\implies \phi = \int z \dd{y} = zx + c(x,z)\\
            \phi_z= y-x\sin z & \implies \boxed{\phi = \int (y-x\sin z)\dd{z} = yz +
            x \cos z + c(x,y)} \\
        \end{align*}
        $$\int\limits_c \grad\underbrace{(yz+x\cos z)}_{\phi} \dd{r} =
        \phi(B)-\phi(A) = \phi(-2,\pi^2,1)-\phi(2,0,1)$$
\end{example}
\end{document}

