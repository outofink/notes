%\documentclass[00_complete]{subfiles}

\documentclass[12pt]{report}
\usepackage[utf8]{inputenc}
\usepackage{amsmath,amssymb,amsthm,gensymb,parskip,graphicx,footmisc,csquotes,enumerate,datetime2}
\usepackage[]{libertinus}
\usepackage[breaklinks]{hyperref}
\hypersetup{
  pdfauthor={Moshe Krumbein},
  colorlinks=true,
  linkcolor={black},
  filecolor={black},
  citecolor={black}, %blue
  urlcolor={black}, %blue
}
\usepackage[top=30mm,bottom=30mm,left=30mm,right=30mm]{geometry}
%\setlength{\emergencystretch}{2em} % prevent overfull lines
\providecommand{\tightlist}{%
\setlength{\itemsep}{0pt}\setlength{\parskip}{0pt}}

\renewcommand\qedsymbol{$\blacksquare$}

\theoremstyle{definition}
\newtheorem*{definition}{Definition}
\newtheorem*{theorem}{Theorem}
\newtheorem*{axiom}{Axiom}
\newtheorem*{lemma}{Lemma}

\theoremstyle{remark}
\newtheorem*{note}{Note}
\newtheorem*{symbols}{Symbol}
\newtheorem{example}{Example}[section]
\newtheorem*{claim}{Claim}
\newtheorem*{conclusion}{Conclusion}
\newtheorem*{reminder}{Reminder}

\usepackage{fancyhdr}
\usepackage[italicdiff]{physics}
\MakeOuterQuote{"}

\renewcommand{\chaptermark}[1]{\markboth{#1}{}}

\pagestyle{fancy}

\setlength{\headheight}{14.5pt}
\addtolength{\topmargin}{-2.5pt}

\fancyhf{}
\rhead{Moshe Krumbein}
\lhead{\chaptermark}
\cfoot{\thepage}
\fancyhead[R]{\chaptername~\thechapter}
\fancyhead[L]{\mbox{\leftmark}}

\usepackage[Rejne]{fncychap}
\usepackage{titling}

\makeatletter
\renewcommand{\@chapapp}{\vspace*{-100pt}\huge\thetitle}
\makeatother

\makeatletter
\newcommand{\subtitle}[1]{%
  {\center\vspace*{-60pt}%
  \linespread{1.1}\Large\scshape#1%
  \par\nobreak\vspace*{35pt}}
}
\makeatother

\newcommand{\Chapter}[2]{
    \def\n{#2}
    \setcounter{chapter}{\the\numexpr\n-1}
    \chapter{#1}
    \subtitle{\theauthor~- \thedate}
}

\DeclareMathOperator{\Ima}{Im}
\DeclareMathOperator{\Id}{Id}
\DeclareMathOperator{\cis}{cis}

\newcommand{\Mod}[1]{\ (\mathrm{mod}\ #1)}
\newcommand{\st}[0]{\;\mathrm{s.t.}\;}


\title{Mathematical Methods}
\author{Moshe Krumbein}
\date{Fall 2021}

\begin{document}
\setcounter{chapter}{10}

\chapter{Multivariable Functions \texorpdfstring{$\mathbb{R}^{\lowercase{m}}
\to \mathbb{R}^{\lowercase{n}}$}{Rm to Rn}}
\subtitle{\theauthor~- \thedate}
\section{Introduction}
We will be examining functions of the form:
\begin{gather*}
    f(x_1,\dots,x_m)=\begin{pmatrix}
        f_1(x_1,\dots,x_m) \\ \vdots \\
        f_n(x_1\cdots,x_m)
    \end{pmatrix}
\end{gather*}
\begin{example}
    \begin{gather*}
        f(r,\theta)=\binom{r\cos \theta}{r \sin \theta}
        \quad \vline \quad
        f(r, \theta, \phi)=\begin{pmatrix}
            r \cos \theta \sin \phi \\
            r \sin \theta \sin \phi \\
            r \cos \phi
        \end{pmatrix}
    \end{gather*}
\end{example}
\begin{definition}[Derivative]
    For the partial derivative:
    \begin{gather*}
        \pdv{f}{x_i} = \begin{pmatrix}
            \pdv{f_1}{x_i} \\ \vdots \\ \pdv{f_n}{x_i}
        \end{pmatrix}
    \end{gather*}
    In general:
    \begin{gather*}
        \dd{\underline f}=\begin{pmatrix}
            \dd{f_1} \\ \vdots \\ \dd{f_n}
        \end{pmatrix} = \begin{pmatrix}
            \pdv{f_1}{x_1}\dd{x_1} +  \pdv{f_1}{x_2}\dd{x_2} +\dots+
            \pdv{f_1}{x_m}\dd{x_m} \\
            \vdots \\
            \pdv{f_n}{x_1}\dd{x_1} +  \pdv{f_n}{x_2}\dd{x_2} +\dots+
            \pdv{f_n}{x_m}\dd{x_m} \\
        \end{pmatrix} = \pdv{\underline f}{x_1}\dd{x_1}+\dots+\pdv{\underline
        f}{x_m}\dd{x_m} \\
    D\underline f = \begin{pmatrix}
    \pdv{f_1}{x_1}&\dots&\pdv{f_1}{x_m} \\
    \vdots && \vdots \\
    \pdv{f_n}{x_1}&\dots&\pdv{f_n}{x_m}
    \end{pmatrix} \quad
    D \underline f \cdot \begin{pmatrix}
    \dd{x_1} \\ \vdots \\ \dd{x_m}
    \end{pmatrix} = \dd{\underline f}
    \end{gather*}
\end{definition}
\begin{example}
    \begin{gather*}
        \underline f (r \theta)= \binom{r \cos \theta}{r \sin \theta} \quad
        \pdv{\underline f}{r} = \binom{\cos \theta}{\sin \theta} \quad
        \pdv{\underline f}{\theta}=\binom{-r\sin \theta}{r \cos \theta} \\
        \dd{\underline f} =\pdv{\underline f}{r}\dd{r}+\pdv{\underline
        f}{\theta}\dd{\theta} =
        \binom{\cos \theta}{\sin \theta}\dd{r}+
        \binom{-r\sin \theta}{r\cos \theta}\dd{\theta} \\
        D\underline f= \begin{pmatrix}
            \cos \theta & -r\sin \theta \\
            \sin \theta & r \cos \theta
        \end{pmatrix} \\
    \end{gather*}
\end{example}
\begin{example} We will consider the following cases
    \begin{enumerate}
        \item $m=1$
            $$f: \mathbb{R} \to \mathbb{R}^n$$
        Describes a path in $\mathbb{R}^n$ (Chapter 9).
        \item $n=1$
            $$f: \mathbb{R}^m \to \mathbb{R}$$
        What we did in Chapter 10.
        \item $m=n$ ($2$ or $3$)

            There are two different ways to examine these functions:
            \begin{enumerate}
                \item Substitution: $f(r,\theta)=\binom{r\cos\theta}{r\sin\theta}$
                \item Using \emph{vector fields}: each point in
                    $\mathbb{R}^n$ can be represented as an $n$-dimensional
                    vector.
            \end{enumerate}
        \item $m=2,n=3$

        Parameterization of a \emph{surface} in $\mathbb{R}^3$
        \begin{example}
            \begin{gather*}
           f:\mathbb{R}^2\to R: \qquad\qquad f:\mathbb{R}^2\to \mathbb{R}^3: \\
            \begin{gathered}
                f(x,y)=x^2+y^2
            \end{gathered} \quad \vline \quad \begin{gathered}
                f(x,y)=\begin{pmatrix}
                    x\\y\\x^2+y^2
                \end{pmatrix} \\
                f(\rho,\theta) =\begin{pmatrix}
                    \rho\cos\theta\\\rho\sin\theta\\\rho^2
                \end{pmatrix}
            \end{gathered}
            \end{gather*}
        \end{example}
    \end{enumerate}
\end{example}
\section{Surfaces in \texorpdfstring{$\mathbb{R}^3$}{R3}}
Or in other words, the \emph{image} of functions $f: \mathbb{R}^2\to
\mathbb{R}^3$.
\begin{example}[Paraboloid]
    \begin{gather*}
        f(\rho,\theta)=\begin{pmatrix}
            \rho\cos\theta\\\rho\sin\theta\\\rho^2
        \end{pmatrix} \quad
        f_{\rho}=\begin{pmatrix}
            \cos\theta\\\sin\theta\\ 2\rho
        \end{pmatrix} \quad f_{\theta}=\begin{pmatrix}
            -\rho\sin\theta\\\rho\cos\theta\\ 0
        \end{pmatrix} \\
        D \underline f = \begin{pmatrix}
            \cos\theta &-\rho\sin\theta \\
            \sin\theta &\rho\cos\theta \\
            2\rho & 0
        \end{pmatrix} \\
        \dd{\underline f} = f_{\rho}\dd{\rho} + f_{\theta}\dd{\theta} =
        \begin{pmatrix}
            \cos\theta\\\sin\theta\\ 2\rho
        \end{pmatrix}\dd{\rho} + \begin{pmatrix}
            -\rho\sin\theta \\ \rho\cos\theta\\ 0
        \end{pmatrix}\dd{\theta}
    \end{gather*}
    $\dd{\underline f}$ is how much $\underline f$ changes when there is a
    small change in $\rho$ or $\theta$. We want to find the \emph{area} of
    $\dd{\underline f}$:
    $$\|f_\rho\times f_\theta\|hk$$
\end{example}

\end{document}
