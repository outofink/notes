\documentclass[00_complete]{subfiles}

\title{Mathematical Methods}
\author{Moshe Krumbein}
\date{Fall 2021}

\begin{document}
\setcounter{chapter}{3}

\chapter{Single Variable Functions}
\subtitle{\theauthor~- \thedate}
\section{Functions}

\begin{definition}
For all inputs $x$ into a function $f(x)$, return a single output $y$.

\emph{Graph:} collection of all the points in the form $(x,f(x))$.

\emph{Image:} collection of all possible $y$ values.

\emph{Note:} A \emph{graph} of a function is not necessarily the \emph{image} of a function.
\end{definition}

\subsection{Hyperbolic Functions}

$$
\begin{gathered}
    \cosh(x)=\frac{1}{2}\left(e^x+e^{-x}\right) \text{ (even)} \\
    \sinh(x)=\frac{1}{2}\left(e^x-e^{-x}\right) \text{ (odd)} \\
    \sinh^2(x)-\cosh^2(x) = 1\\
    \tanh(x) = \frac{\sinh(x)}{\cosh(x)}
\end{gathered}
$$

The hyperbolic functions maintain many of the same identities of the
trigonometric functions.

The area between the \emph{ray} to $(\cosh(t), \sinh(t))$ and the hyperbolic
function $x^2-y^2 = 1$ is exactly equal to $\frac{t}{2}$.

$$
\begin{gathered}
    \sinh: \mathbb{R} \to \mathbb{R} \\
    \cosh: \mathbb{R} \to [1, \infty) \\
    \sinh^{-1}: \mathbb{R} \to \mathbb{R} \\
    \cosh^{-1}: [1, \infty) \to [0, \infty)
\end{gathered}
$$

Formula for $\cosh^{-1}x$:

$$
\begin{gathered}
    u = \cosh^{-1}x \\
    \Updownarrow \\
    \cosh(x) = x \\ \Updownarrow \\
    e^u + e^{-u} =2x \; \backslash : e^u\\
    e^{2u} + 1 - 2xe^u = 0 \\
    e^u = x \pm \sqrt{x^2-1} \\
    u = \ln(x\pm\sqrt{x^2-1}) \\
\end{gathered}
$$

We found 2 values for $u$ who are opposite from one either, so we define $u$ to
be the greater one:

$$\boxed{\cosh^{-1}x = \ln{x+\sqrt{x^2-1}}}$$

\section{Exponential Functions}

$$f(x) = x^n, \quad n \in \mathbb{R}$$

\subsection{With Natural Exponents}

If $n$ is odd then $f(x)=x^n$ is odd.

If $n$ is even then $f(x)=x^n$ is even.

\subsection{With Reciprocal Natural Exponents}

$$
\begin{gathered}
    f(x)= x^{\frac{1}{n}} = \sqrt[n]{x}
\end{gathered}
$$

This function is the inverse of $x^n$.

If $n$ is odd then $f(x)=x^n$ is odd.

If $n$ is even then $f(x)=x^n$ is even.

\subsection{With Rational Exponents}

$$
\begin{gathered}
    f(x) = x^{\frac{m}{n}} = \sqrt[n]{x^m}
\end{gathered}
$$

If $n$ is odd: $x^{\frac{m}{n}}$ is defined for all $x$.

If $m$ is even the function is even and if $m$ is odd then the function is odd.

If $\frac{m}{n} > 1$ the function is \emph{convex} (up) and if $\frac{m}{n} < 1$ the
function is \emph{concave} (down).

If $n$ is even (therefore $m$ is odd) $x^{\frac{m}{n}}$ is defined $x \in [0, \infty)$.

If $\frac{m}{n} > 1$ the function is \emph{convex} (up) and if $\frac{m}{n} < 1$ the
function is \emph{concave} (down).

\subsection{With Negative Rational Exponents}

$$
    f(x)=x^{-\frac{m}{n}} \\
$$

The right side of the function always looks like $\frac{1}{x}$

Just the right side of the function: $n$ is even.

Odd function that looks like $\frac{1}{x}$: $n,m$ are odd.

Even function that looks like $\frac{1}{x}$: $n$ is odd and $m$ is even.

\subsection{With Real Exponents}

$$\begin{gathered}
    x^\alpha : \alpha \in \mathbb{R} \backslash \mathbb{Q} \\
    x^\alpha := e^{\alpha \ln x} \quad (x > 0)
\end{gathered}
$$

\section{Composing Functions}

$$
\begin{gathered}
    g \circ f(x) = g(f(x)) \\
\end{gathered}
$$

\begin{note}
$$
\sin x  = \frac{1}{i}\sinh x
$$
\end{note}

\subsection{Graphing functions with Square Roots}

$$f(x)=\sqrt{(1-x)(2-x)(3-x)}$$

First we graph what's under the square root, then we see where the function is
defined with a square root and how it affects the slope of the function (if the
power is greater than or less than $1$).

\section{Continuous Functions}

$f$ is \emph{continuous} at point $a$ if $x \to a: f(x) \to f(a)$.

$f$ is a \emph{continuous function} if for all $a$ in the \emph{domain} of $f$ fulfills: $x \to a: f(x) \to f(a)$.

\subsection{Continuity of Functions on a Closed Interval}

\begin{theorem}[Intermediate value theorem]
If the function $f$ is continuous on $[a,b]$ and $c$ is between $f(b)$ and
$f(a)$ then there exists $x \in [a,b]$ such that $f(x) = c$.
\end{theorem}

\begin{theorem}[Intermediate value theorem]
If $f$ is continuous on $[a,b]$ then $f$ has a \emph{minimum} and
\emph{maximum} on $[a,b]$.
\end{theorem}
\end{document}
