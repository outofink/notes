\documentclass[00_complete]{subfiles}

%\documentclass[12pt]{report}
\usepackage[utf8]{inputenc}
\usepackage{amsmath,amssymb,amsthm,gensymb,parskip,graphicx,footmisc,csquotes,enumerate,datetime2}
\usepackage[]{libertinus}
\usepackage[breaklinks]{hyperref}
\hypersetup{
  pdfauthor={Moshe Krumbein},
  colorlinks=true,
  linkcolor={black},
  filecolor={black},
  citecolor={black}, %blue
  urlcolor={black}, %blue
}
\usepackage[top=30mm,bottom=30mm,left=30mm,right=30mm]{geometry}
%\setlength{\emergencystretch}{2em} % prevent overfull lines
\providecommand{\tightlist}{%
\setlength{\itemsep}{0pt}\setlength{\parskip}{0pt}}

\renewcommand\qedsymbol{$\blacksquare$}

\theoremstyle{definition}
\newtheorem*{definition}{Definition}
\newtheorem*{theorem}{Theorem}
\newtheorem*{axiom}{Axiom}
\newtheorem*{lemma}{Lemma}

\theoremstyle{remark}
\newtheorem*{note}{Note}
\newtheorem*{symbols}{Symbol}
\newtheorem{example}{Example}[section]
\newtheorem*{claim}{Claim}
\newtheorem*{conclusion}{Conclusion}
\newtheorem*{reminder}{Reminder}

\usepackage{fancyhdr}
\usepackage[italicdiff]{physics}
\MakeOuterQuote{"}

\renewcommand{\chaptermark}[1]{\markboth{#1}{}}

\pagestyle{fancy}

\setlength{\headheight}{14.5pt}
\addtolength{\topmargin}{-2.5pt}

\fancyhf{}
\rhead{Moshe Krumbein}
\lhead{\chaptermark}
\cfoot{\thepage}
\fancyhead[R]{\chaptername~\thechapter}
\fancyhead[L]{\mbox{\leftmark}}

\usepackage[Rejne]{fncychap}
\usepackage{titling}

\makeatletter
\renewcommand{\@chapapp}{\vspace*{-100pt}\huge\thetitle}
\makeatother

\makeatletter
\newcommand{\subtitle}[1]{%
  {\center\vspace*{-60pt}%
  \linespread{1.1}\Large\scshape#1%
  \par\nobreak\vspace*{35pt}}
}
\makeatother

\newcommand{\Chapter}[2]{
    \def\n{#2}
    \setcounter{chapter}{\the\numexpr\n-1}
    \chapter{#1}
    \subtitle{\theauthor~- \thedate}
}

\DeclareMathOperator{\Ima}{Im}
\DeclareMathOperator{\Id}{Id}
\DeclareMathOperator{\cis}{cis}

\newcommand{\Mod}[1]{\ (\mathrm{mod}\ #1)}
\newcommand{\st}[0]{\;\mathrm{s.t.}\;}

\title{Mathematical Methods}
\author{Moshe Krumbein}
\date{Fall 2021}

\begin{document}
\Chapter{Vectors and Euclidean Geometry}{1}

\section{Set Theory}

$$A \times B = \{(x,y) \;|\; x \in A, y \in B\}$$
$$\mathbb{R} \times \mathbb{R} = \mathbb{R}^2 = \{(x,y)\;|\;x,y \in \mathbb{R}\}$$

Circle:
$$\left\{(x,y) \in \mathbb{R}^2 \;|\; (x-x_0)^2 + (y-y_0)^2 = r^2\right\}$$

Graph:
$$G_f = \left\{(x,y) \in \mathbb{R}^2 \;|\; y=f(x)\right\}$$

\section{Vectors in \texorpdfstring{$\mathbb{R}^n$}{Rn}}

\subsection{What is a vector?}

A \emph{vector} is a is a arrow with a \emph{direction} and \emph{magnitude}.

It can express:
\begin{enumerate}
    \item Displacement
    \item Angular velocity
\end{enumerate}

Two vectors with the same magnitude and direction are \emph{equal}.
\subsection{Definitions and Symbols}

\subsubsection{What is \texorpdfstring{$\mathbb{R}^n$}{Rn}?}

Plane: $\mathbb{R}^2=\{(x,y)\;|\;x,y \in \mathbb{R}\}$

Space: $\mathbb{R}^3=\{(x,y,z)\;|\;x,y,z \in \mathbb{R}\}$

$$\mathbb{R}^n = \{(x_0,x_1,x_2,\dots)\;|\;\underset{1 \leq i \leq n}{x_i} \in \mathbb{R}\}$$

$$\mathbb{R}^n=\left\{
\begin{pmatrix}
   x_1 \\
   \vdots \\
   x_n \\
\end{pmatrix} \;|\;
\underset{1 \leq i \leq n}{x_i \in \mathbb{R}}
\right\}$$

Points: $A,B,C$

Parallel lines: $AB \parallel CD$

Perpendicular lines: $AB \perp CD$

Distance between points (on a plane):
$$AB = \sqrt{(a_1-b_1)^2+(a_2-b_2)^2}$$

Distance between points (in $\mathbb{R}^3$ space):
$$AB = \sqrt{(a_1-b_1)^2+(a_2-b_2)^2+(a_3-b_3)^2}$$

Distance (in $\mathbb{R}^n$):
$$AB= \sqrt{\displaystyle\sum_{i=1}^n(a_i-b_i)^2}$$

\subsection{Expressing vectors mathematically}

We place the \emph{tail} of the vector and put it at the \emph{origin} $(0,0)$ and the
\emph{head} at point $P(x,y)$.

If $\underline u = \vec{(x,y)}$ that means that $u$ is a vector that starts at $(0,0)$ and ends $x$ to the right and $y$ up.

\begin{symbols}
Vector: $\underline u$, $\vec{AB}$

Zero vector: $\underline 0 = \begin{pmatrix}
    0 \\
    \vdots \\
    0
\end{pmatrix}$

Magnitude: $\|\underline u\|$

Position vector at $(x,y,z)$: $\underline r =
\begin{pmatrix}
    x \\y\\z
\end{pmatrix}$

Unit vector (magnitude of 1): $\underline{\hat e}$

On $\mathbb{R}^2$: $\underline{\hat \imath} = \begin{pmatrix}
    1\\0
\end{pmatrix}
\quad
\underline{\hat \jmath} =  \begin{pmatrix}
    0\\1
\end{pmatrix}$

\end{symbols}
\subsection{Adding vectors}

$$\underline u + \underline v = \underline v + \underline u$$
$$\begin{pmatrix}
    a_1 \\a_2\\\vdots\\a_n
\end{pmatrix}+\begin{pmatrix}
    b_1 \\b_2\\\vdots\\b_n
\end{pmatrix}=\begin{pmatrix}
    a_1+b_1\\a_2+b_2\\\vdots\\a_n+b_n
\end{pmatrix}$$

\subsection{Multiplying a vector with a scalar}

$$\lambda \cdot \underline a  = \lambda \cdot \begin{pmatrix}
    a_1\\a_2\\\vdots\\a_n
\end{pmatrix} = \begin{pmatrix}
    \lambda \cdot a_1 \\
    \lambda \cdot a_2 \\
    \vdots \\
    \lambda \cdot a_n
\end{pmatrix}$$

$$- \underline a = \begin{pmatrix}
    -a_1\\\vdots\\-a_n
\end{pmatrix} = (-1) \cdot \underline a$$

\subsection{Finding getting a unit vector from a vector (normal)}

$$\underline{\hat a} = \frac{\underline a}{\|\underline a\|}$$

Sometimes we symbolize the length of $\underline r$ as $r$.

$$\underline r = r \cdot \underline{\hat r}$$

\subsection{Subtracting vectors}

$$\underline u + (-\underline v) = \underline u - \underline v$$

\subsection{Expressing vectors}

$$\begin{pmatrix}
    x\\y\\z
\end{pmatrix} = x\begin{pmatrix}
    1\\0\\0
\end{pmatrix} + y\begin{pmatrix}
    0\\1\\0
\end{pmatrix} + z\begin{pmatrix}
    0\\0\\1
\end{pmatrix}
= x\;\underline{\hat \imath} + y\;\underline{\hat \jmath} + z\;\underline{\hat k}$$

\subsection{Finding a line that passes through two position vectors
\texorpdfstring{$\underline a,\underline b$}{a, b}}

\subsubsection{Parametric Form}

$$\underline v = \underline b - \underline a$$

$$l = \{ \underline a + t \underline v \;|\; t \in \mathbb{R}\}$$
$$= \{(1-t)\underline a + t \underline b \;|\; t \in \mathbb{R}\}$$
$$= \{(1-t)\underline a + t \underline b \;|\; 0 \leq t \leq 1\}$$

If $t \in (0,1)$, the point is between $\underline a$ and $\underline b$ ($AB$).

\begin{example}
$$A=(1,0,6) \quad B=(2,1,3)$$


$$
    \left\{(1-t)\begin{pmatrix}
        1\\0\\6
    \end{pmatrix}+t \begin{pmatrix}
        2\\1\\-3
    \end{pmatrix}\;|\; t \in \mathbb{R}\right\}
$$
$$
    \left\{\begin{pmatrix}
        1+t\\t\\6-9t
    \end{pmatrix} \;|\; t \in \mathbb{R} \right\}
$$

$$
    AB=\underline r = \begin{pmatrix}
        1+t\\t\\6-9t
    \end{pmatrix}_{0 \leq t \leq 1}
$$
\end{example}
\subsection{Scalar multiplication (between two vectors)}

$\underline a, \underline b \in \mathbb{R}^n$:

$$\begin{pmatrix}
    a_1 \\ \vdots \\ a_n
\end{pmatrix} \cdot \begin{pmatrix}
    b_1 \\ \vdots \\ b_n
\end{pmatrix} = a_1 b_1 + \dots a_n b_n$$

$$\underline a \cdot \underline b = \sum_{i=1}^n a_i b_i$$

\begin{example}
$$\begin{pmatrix}
    2\\1\\7
\end{pmatrix} \cdot \begin{pmatrix}
    1\\-5\\2
\end{pmatrix} = 2 \cdot 1 + 1 \cdot (-5) + 7 \cdot 2 = 11$$
\end{example}

\subsubsection{Characteristics:}
\begin{enumerate}
    \item $\underline a \cdot \underline b = \underline b \cdot \underline a$
    \item $(\lambda \underline a) \cdot \underline b = \lambda (\underline a \cdot
   \underline b)$
    \item $\underline a \cdot (\underline b + \underline c) = \underline a \cdot
   \underline b + \underline a \cdot \underline c$
    \item $\underline a \cdot \underline a = \|a\|^2$
\end{enumerate}
\subsubsection{Geometric meaning:}

$\underline a, \underline b \in \mathbb{R}^n$:

$$\|\underline a - \underline b\|^2=(\underline a - \underline
b)\cdot(\underline a - \underline b)$$
$$=\|a\|^2+\|b\|^2 - 2(\underline a \cdot \underline b)$$
\begin{reminder}[Law of Cosines]
$$c^2=a^2+b^2+2ab\cos \theta$$
\end{reminder}
\begin{conclusion}
$$\underline a \cdot \underline b = \|a\| \cdot \|b\| \cos \theta$$
$$\cos \theta = \frac{\underline a \cdot \underline b}{\|a\|\cdot\|b\|}$$
$\underline a, \underline  b \neq \underline 0$:

$$
\begin{gathered}
    \underline a \cdot \underline b = 0 \iff \underline a \perp \underline b \\
    \underline a \cdot \underline b > 0 \iff \theta \text{ is acute } \left(0 \leq
    \theta < \frac{\pi}{2}\right)\\
    \underline a \cdot \underline b < 0 \iff \theta \text{ is obtuse } \left(\frac{\pi}{2} < \theta \leq \pi\right)\\
\end{gathered}
$$
\end{conclusion}
\section{Projection}
Let $\underline{\hat e}$ be a unit vector and $\underline a$ be some vector,
such that $\underline a \cdot \underline e = \|a\|\cos \theta$
(\emph{projection} of $\underline a$ on $\underline{\hat e}$).

\section{Planes in \texorpdfstring{$\mathbb{R}^3$}{R3}}

Consider a plane $\Pi \subseteq \mathbb{R}^3$, the \emph{normal vector} to $\Pi$,
$\underline n$ (perpendicular to the plane), and the \emph{unit vector} in the direction of the normal,
$\underline{\hat n}$.

$$\forall \; \underline r \in \Pi : \underline r \cdot
\underline{\hat n} = \mathbf 0$$

\begin{example}

$$
\begin{gathered}
    \underline n = \begin{pmatrix}
        1\\0\\6
    \end{pmatrix}, \|\underline n\| = \sqrt{37} \\
    \underline{\hat r} = \begin{pmatrix}
        \frac{1}{\sqrt{37}} \\
        0 \\
        \frac{6}{\sqrt{37}}
    \end{pmatrix} \\
    \underline r \cdot \begin{pmatrix}
        \frac{1}{\sqrt{37}} \\
        0 \\
        \frac{6}{\sqrt{37}}
    \end{pmatrix} = 3 \\
    \begin{pmatrix}
        x\\y\\z
    \end{pmatrix} \cdot \begin{pmatrix}
        \frac{1}{\sqrt{37}} \\
        0 \\
        \frac{6}{\sqrt{37}}
    \end{pmatrix} = 3
\end{gathered}
$$

$$\underline r \cdot \underline n = \alpha \|\underline n\|$$
$$\underline r \cdot \underline{\hat n} = \alpha$$

Given plane $3x+2y-z=10$, find the normal vector and the distance to the
origin.

Normal vector: $\underline n = \begin{pmatrix}
    3\\2\\-1
\end{pmatrix}$

Distance: $\frac{10}{\|\underline n\|} = \frac{10}{\sqrt{3^2+2^2+(-1)^2}} = \frac{10}{\sqrt{14}}$

\end{example}
In general, given the plane $ax+by+cz = d$, the distance from the origin is $\frac{d}{\sqrt{a^2+b^2+c^2}}$.

\section{Determinant}

A \emph{determinant} of a square matrix of the size $2\times2$ and $3\times3$:

$2\times2$:
$$
\begin{gathered}
    \begin{vmatrix}
        a & b \\
        c & d
    \end{vmatrix} = \det \begin{pmatrix}
        a & b \\
        c & d
    \end{pmatrix} = ad - bc
\end{gathered}
$$

It represents the \emph{signed} area of a parallelogram made from the vectors $\binom{a}{c}$,
$\binom{b}{d}$.

$3\times3$:
$$
\begin{gathered}
    \begin{vmatrix}
        a & b & c \\
        d & e & f \\
        g & h & i
    \end{vmatrix} = \det \begin{pmatrix}
        a & b & c \\
        d & e & f \\
        g & h & i
    \end{pmatrix} = a \begin{vmatrix}
        e & f \\ h & i
    \end{vmatrix} - d \begin{vmatrix}
        b & c \\ h & i
    \end{vmatrix} + g \begin{vmatrix}
        b & c \\ e & f
    \end{vmatrix}
\end{gathered}
$$

It represents the \emph{signed} area of a \emph{parallelepiped} (3D object made up of 6
parallelograms).

\section{Vector Multiplication}

\begin{symbols}
$\land$ or $\times$

A vector ($\in \mathbb{R}^3$) $\land$ a vector ($\in \mathbb{R}^3$) $=$ a vector ($\in \mathbb{R}^3$).

$$\begin{pmatrix}
    a_1 \\ a_2 \\a_3
\end{pmatrix} \land \begin{pmatrix}
    b_1 \\ b_2 \\ b_3
\end{pmatrix} = \begin{pmatrix}
    \phantom{-} \begin{vmatrix}
        a_2 & b_2 \\
        a_3 & b_3
    \end{vmatrix} \\
    \\
    - \begin{vmatrix}
        a_1 & b_1 \\
        a_3 & b_3
    \end{vmatrix} \\
    \\
    \phantom{-} \begin{vmatrix}
        a_1 & b_1 \\
        a_2 & b_2
    \end{vmatrix}
\end{pmatrix} = \begin{pmatrix}
    a_2b_3 - a_3b_2 \\
    a_3b_1 - a_1b_3 \\
    a_1b_2 - a_2b_1
\end{pmatrix}$$


$$\underline a \land \underline b = \underline{\hat \imath}
    \begin{vmatrix}
        a_2 & b_2 \\
        a_3 & c_3
    \end{vmatrix}
    - \underline{\hat \jmath} \begin{vmatrix}
        a_1 & b_1 \\
        a_3 & b_3
    \end{vmatrix}
    + \underline{\hat k} \begin{vmatrix}
        a_1 & b_1 \\
        a_2 & b_2
    \end{vmatrix}
$$
$$= \begin{vmatrix}
    \underline{\hat \imath} & a_1 & b_1 \\
    \underline{\hat \jmath} & a_2 & b_2 \\
    \underline{\hat k} & a_3 & b_3
\end{vmatrix}$$

\end{symbols}
\subsection{Characteristics}
\begin{enumerate}
    \item $\underline a \land \underline b$ is perpendicular to both $\underline a$
   and $\underline b$
    \item $\|\underline a \land \underline b\| = \|\underline a\| \cdot
        \|\underline b\| \sin \theta$

\end{enumerate}

\subsection{Triple scalar product \texorpdfstring{$\in \mathbb{R}^3$}{in R3}}

Given $\underline a, \underline b, \underline c \in \mathbb{R}^3$, their
\emph{triple scalar product} is:

$$
[\underline a, \underline b, \underline c] =
\begin{vmatrix}
    a_1 & b_1 & c_1 \\
    a_2 & b_2 & c_2 \\
    a_3 & b_3 & c_3
\end{vmatrix} = \underline a \cdot (\underline b \land \underline c)$$
\end{document}
