\documentclass[00_complete]{subfiles}

\title{Mathematical Methods}
\author{Moshe Krumbein}
\date{Fall 2021}

\begin{document}
\Chapter{Complex Numbers}{3}

\section{Introduction}

\[
\mathbb{C} = \{a +bi \mid a, b \in \mathbb{R}\}
\]

\emph{Inception:} To help solve 3rd degree polynomials.

\begin{definition}[Fundamental Theorem of Algebra]
For all polynomials:
$$a_nx_n+a_{a-1}x^{x-1}+\ldots+a_1x+a_0$$
that has \emph{complex coefficients} has \emph{complex roots}.
\end{definition}

\section{Operations}

\begin{gather}
    (a+bi)+(c+di) = (a+d)+(b+d)i \\
    (a+bi)-(c+di) = (a-d)+(b-d)i \\
    (a+bi)(c+di) = (ac - bd) + (ad + bc)i \\
    \frac{a+bi}{c+di} = \frac{(a+bi)(c-di)}{(c+di)(c-di)}
    = \frac{e+fi}{c^2+d^2} = \frac{e}{c^2+d^2}+\frac{f}{c^2+d^2}i
\end{gather}

\section{Complex Plane}

Every complex number \(z=a+bi\) can be represented on the complex plane
at the point \((a,b)\).

It can also be represented in the polar form: \[
\begin{gathered}
    r=|z| \quad \theta = \arg(z) \\
    z = r \cos \theta + r \sin \theta i \quad (\cis \theta)\\
\end{gathered}
\]

\subsection{Characteristics}

\begin{enumerate}
\item
  Properties of four algebraic operations of the real numbers also apply
  to the complex ones (i.e. associative, distributive, etc.)
\item
  \emph{Complex conjugate}: \[
  \begin{gathered}
   \overline{z_1 \pm z_2} = \overline z_1 \pm \overline z_2 \\
   \overline{z_1 \cdot z_2} = \overline z_1 \cdot \overline z_2 \\
   \overline{\frac{z_1}{z_2}} = \frac{\overline z_1}{\overline z_2} \\
   \frac{1}{z}=\frac{\overline z}{|z|^2}, \; z \cdot \overline z  = |z|^2 \\
  \end{gathered}
  \]
\end{enumerate}

\subsection{Analysis}

\[
\begin{gathered}
    z_1 = r_1(\cos \theta_1 + i \sin \theta_1) \\
    z_2 = r_2(\cos \theta_2 + i \sin \theta_2) \\
    z_1 z_2= r_1 r_2 [
        (\underbrace{\cos \theta_1 \cos \theta_2 - \sin \theta_1 \sin\theta_2}
            _{\cos (\theta_1 + \theta_2)})
        +i(\underbrace{\sin \theta_1 \cos \theta_2 + \sin \theta_2 \cos \theta_1}
            _{\sin(\theta_1 + \theta_2)})
    ] \\
\end{gathered}
\]

\begin{conclusion}
\[
\begin{gathered}
    |z_1 z_2| = r_1 r_2 \\
    \arg(z_1 z_2) = \theta_1 + \theta_2
\end{gathered}
\]
\end{conclusion}

\begin{definition}[De Moivre's Formula]
\[
(r \cis\theta)^n=r^n \cis(n \theta)
\]
\end{definition}

\begin{definition}[\texorpdfstring{\(n\)th-root of a complex
number}{nth-root of a complex number}]
\[
\begin{gathered}
    z^n = r \cis \theta \\
    z = \sqrt[n]{r} \cis\left(\frac{\theta + 2 \pi k}{n}\right),
    \quad k = 0, 1, 2, \ldots, n-1
\end{gathered}
\]
\end{definition}

\begin{definition}[Euler's Formula]
\[
\begin{gathered}
    e^{i \theta} = \cos \theta + i \sin \theta \\
    e^{-i \theta} = \cos  \theta - i \sin \theta \\
    \cos \theta = \frac{1}{2}\left(e^{i \theta}+e^{-i \theta}\right)
    \quad \sin \theta = \frac{1}{2i}\left(e^{i \theta} - e^{-i \theta}\right)
\end{gathered}
\]

\(e\) to a complex number:
\[
e^{a+ib} = e^a e^{ib} = e^a(\cos b + i\sin b)
\]

\end{definition}
Our goal is to:

\begin{enumerate}
\item
  Express \(\cos(nx)\) in terms of \(\sin x, \cos x\).
\item
  Express \(\sin^n(x)\) as a sum of \(\sin x, \cos x\), without
  multiplying them.
\end{enumerate}

\begin{example}
\[
\begin{gathered}
    \cos(5x) = \Re(e^{i5x}) = \Re((e^{ix})^5) \\
    = \Re((\cos x+i\sin x)^5) \\
    (a+b)^5 = a^5 + 5a^4b+10a^3b^2 + 10a^2b^3 + \ldots
\end{gathered}
\]
To simplify our calculation since we are only looking for the real part
of our solution, we can ignore any place where \(\sin\) is raised to an
odd power (since \(i^2 = -1\)). \[
    = \cos^5x-10\cos^x\sin^2x+5\cos x\sin^4x
\] Now for an example in the opposite direction: \[
\begin{gathered}
    \sin^5x = \left(\frac{1}{2i}\right)^4(e^{ix}-e^{-ix})^4 \\
    \frac{1}{16}(e^{i4x}-4e^{i2x} +6 -4 e^{-i2x}+e^{-i4x}) \\
    =\frac{1}{16}(2\cos (4x)-8 \cos (2x)+6)
\end{gathered}
\]
\end{example}
\begin{example}
\[
\begin{gathered}
    a \cos (\omega t) + b\sin(\omega t) \\
    \Re(\underbrace{(a+bi)}_{re^{i \theta}}\underbrace{(\cos(\omega t)- i \sin (\omega t)}_{e^{-i\omega t}}) \\
    = \Re\left(re^{i(\theta - \omega t)}\right)
    = r \cos(\theta - \omega t) = r \cos(\omega t - \theta)\\
    =\sqrt{a^2 + b^2} \cos\left(\omega t -\tan^{-1}\left(\frac{a}{b}\right)(+ \pi)\right)
\end{gathered}
\]
\end{example}

\section{What is \texorpdfstring{$\ln(a+bi)$}{ln(a+bi)}?}

\(\ln z\) is the solution to the equation
\(e^\omega=z \to e^u \cdot e^{iv}=a+bi = re^{i\theta}\).

\[
\begin{gathered}
    u = \ln r = \ln|z| \\
    v= \theta +2\pi k
\end{gathered}
\]

Conclusion: \[
\ln(z) = \ln|z|+i(\arg(z)+2\pi k)
\]

\begin{example}
\[
\begin{gathered}
    \ln(- \sqrt 3 + i)\\
    -\sqrt 3 + i =2 \cis\left(\underbrace{\tan^{-1}\left(-\frac{1}{\sqrt 3}\right)}_{-\frac{\pi}{6}}+\pi\right) \\
    =2 \cis\left(\frac{5\pi}{6}\right) \\
    \ln(-\sqrt 3+i) = \ln 2 + i \left(\frac{5\pi}{6} + 2 \pi k\right)
\end{gathered}
\]
\end{example}

\section{Solving Complex Equations}

\[
\begin{gathered}
    z^4+z^3+z^2+z +1 = 0 \;\backslash : z^2 \\
    z^2+z +1 + \frac{1}{z} + \frac{1}{z^2} = 0 \\
    t = z + \frac{1}{z} \quad t^2 = z^2 +2 + \frac{1}{z^2} \\
    t^2 + t - 1= 0 \\
    t_{1,2} = \frac{-1 \pm \sqrt 5}{2} \\
    \Downarrow \\
    2z^2 -(-1 + \sqrt 5) + 2 = 0
\end{gathered}
\]

\(z\) can be found given that we know how to find the square root of
complex numbers. \[
    z_{1,2} = \frac{-1 + \sqrt 5 \pm \sqrt{(-1+\sqrt 5)^2-16}}{4}
\]

\[
\begin{gathered}
z^4+z^3+z^2+z+1 = 0 \\
(z-1)(z^4+z^3+z^2+z+1)=0 \\
z^5 = 1 = \boxed{1 \cdot e^{i \cdot 0}}
\end{gathered}
\]

\section{Fundamental Theorem of Algebra}

All polynomials can be factored into a product of linear elements in the
complex world. \[
\begin{gathered}
x^4 -1 = (x^2-1)(x^2+1) \\
=(x-1)(x+1)(x-i)(x+i)\tag{1}
\end{gathered}
\]

\[
\begin{gathered}
\tag{2}
x^3-3x^2+2=(x-1)(x^2-2x-2) \\
x_{1,2}=\frac{2\pm2\sqrt{3}}{2} = 1\pm\sqrt 3 \\ \Downarrow \\
=(x-1)(x-(1+\sqrt 3))(x-(1-(1-\sqrt 3)))
\end{gathered}
\]

\[
\begin{gathered}
\tag{3}\\
x^2 + 6  + 9=(x+3)^2 \\
-3 \text{ is the root (double root)}
\end{gathered}
\]

\[
\begin{gathered}
    x^4+2x^2+1 = (x^2+1)^2 \quad (\text{division over the real numbers}) \\
    (x-i)^2(x+i)^2 \quad (\text{division over the complex numbers}) \\
    \pm i \text{ are each double roots} \tag{4}
\end{gathered}
\]
\begin{claim}
\[
p(x)a_nx^n+a_{n-1}x^{x-1}+\ldots+a_1x+a_0
\]

Polynomials with real coefficients if \(z\) is a root of \(p(x)\) then
\(\overline z\) is also a root of \(p(x)\).

\end{claim}
\begin{example}
\[
\begin{gathered}
x^3+3x^2+4x+2 \\
\pm1, \pm 2 = (x+1)(x^2+2x+2) \\
x_{1,2} = \frac{-2 \pm \sqrt{2^2{-8}}}{2} = \boxed{-1 \pm i}
\end{gathered}
\]

\end{example}
\begin{proof}
Given \(a_nz^n+ \ldots + a_0=0\), we have to prove: \[
a_n(\overline z)^n + a_{n-1}(\overline z)^{n-1}+ \ldots = 0
\]

\[
\begin{gathered}
    \text{Reminder:} \\
    \overline{z_1z_2} = \overline z_1 \overline z_2 \quad \overline{z_1+z_2} = \overline z_1 + \overline z_2 \\
    \Downarrow \\
    (\overline{z^n}) = (\overline z)^n
\end{gathered}
\]

Given: \(a_nz^n+a_{a-1}z^{n-1}+\ldots = 0\): \[
\begin{gathered}
\overline{a_nz^n+a_{a-1}z^{n-1}+\ldots = 0} = \overline 0 \\ \Downarrow \\
\overline{a_nz^n} + \ldots + \overline a_n = 0 \\
\overline{a_n}\overline{z^n} + \ldots + \overline a_n = 0 \\
a_n(\overline z)^n + \ldots + a_n = 0 \\
\implies \overline z \text{ is a root of the polynomial.}
\end{gathered}
\] Note on \(\ln(z)\): There are infinite solutions because
\(\omega = \ln |z| + i(\arg z+2\pi k)\) where \(k \in \mathbb{N}\).

\end{proof}
\begin{example}
\(x^4+x^2+1\) can be factored over the real numbers and over the complex
numbers.

\[
\begin{gathered}
t = x^2 \to t^2 + t + 1 \\
t_{1,2} = -\frac{1}{2} \pm i \frac{\sqrt 3}{2} \\
z_1 = \pm \sqrt{t} = \pm \sqrt{e^{i+\frac{2 \pi}{3}}} = \pm e^{i\frac{\pi}{3}} \\
z_2 = \pm \sqrt{e^{\frac{4 \pi}{3}i}} = \pm e^{i\frac{2 \pi}{3}} \\\\
(x-e^{i \frac{\pi}{3}})(x+e^{i \frac{\pi}{3}})(x-e^{i\frac{2 \pi}{3}})(x+e^{i\frac{2 \pi}{3}}) \\ \\
e^{i \frac{\pi}{3}} = \cis \frac{\pi}{3} = \frac{1}{2}+ \frac{\sqrt 3}{2}i \implies \\
\pm e^{i\frac{2 \pi}{3}} = \pm\left(-\frac{1}{2}+\frac{\sqrt 3}{2}i\right) \\
=(x^2+x+1)(x^2+x+1)
\end{gathered}
\]

\[
\begin{gathered}
(x-z)(x-\overline z) \\
= x^2 - (\underbrace{z + \overline z}_{\text{real}}) x + \underbrace{z \cdot
\overline z}_{\text{real}}
\end{gathered}
\]

All real polynomials can be factored into real linear or double roots.

\end{example}
\begin{example}
\[
\begin{gathered}
(\underbrace{2+i}_{\theta_1})(\underbrace{3+i}_{\theta_2}) = 5+ 2i +3i + i^2 = \underbrace{5+5i}_{\theta_1 + \theta_2} \\
\theta_1 =\tan^{-1} \left(\frac{1}{2}\right) \quad \theta_2 = \tan^{-1}
\left(\frac{1}{3}\right) \\
\theta_ 1+ \theta_2 = \frac{\pi}{4}
\end{gathered}
\]

\end{example}
\end{document}
