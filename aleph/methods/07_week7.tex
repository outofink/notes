\documentclass[00_complete]{subfiles}

\title{Mathematical Methods}
\author{Moshe Krumbein}
\date{Fall 2021}

\begin{document}
\Chapter{Indefinite Integrals}{7}

\section{Introduction}

\begin{definition}
    We say that $F(x)$ is a \emph{primitive function (anti-derivative)} of
    $f(x)$ if:
    $$F'(x)=f(x)$$
    We also symbolize:
    $$\boxed{\int f(x)\dd{x} = F(x) + c}$$
\end{definition}

\begin{theorem}
    If $F(x)$ is an \emph{anti-derivative} of $f$, then for all $c$, $F(x)+c$ is also
    a \emph{anti-derivative} of $f$.
\end{theorem}

\begin{theorem}
    If $F$ and $G$ are \emph{anti-derivatives} of $f$ for a given interval, then for
    that same interval $F-G=c$.
\end{theorem}

\begin{example}
    $$
    \begin{gathered}
        f(x)=\frac{1}{x} \\ \\
        F(x)=\begin{cases}
            \ln x & x>0 \\
            \ln(-x) & x<0
        \end{cases} \\
        F(x) \text{ is an anti-derivative of } f(x) \\ \\
        G(x) = \begin{cases}
            \ln x + 1 & x>0 \\
            \ln (-x) + 3 & x<0
        \end{cases} \\
        G(x) \text{ is an anti-derivative of } f(x) \\ \\
    \end{gathered}
    $$
    We see here that although $G-F$ isn't always the same constant, it is
    always a constant.
    $$\int \frac{1}{x} = \ln |x| + c$$
\end{example}

\section{Rules of Integration}

Integral of a sum is the same of the integral:
\begin{equation}
    \int f + g = \int f + \int g
\end{equation}
Integral of composed functions:
\begin{equation}
    \int f(g(x))g'(x) = F(g(x)) + c
\end{equation}
Integration by parts:
\begin{equation}
\begin{gathered}
    \int u(x)v'(x)\dd{x} = u(x)v(x) - \int u'(x)v(x)\dd{x} \\
    \text{or more simply:} \\
    \int uv' = uv - \int u'v
\end{gathered}
\end{equation}
\begin{example}
    $$
    \begin{gathered}
        \int f(ax+b)\dd{x} = \frac{1}{a} \int a\cdot f(ax+b)\dd{x} =
        \frac{1}{a} \int f(t)\dd{t}
        \\
        \frac{1}{a}F(t)+c = \frac{F(ax+b)}{a}+c
    \end{gathered}
    $$
\end{example}
\section{Examples of Integration by Substitution}
\setcounter{example}{0}
\begin{example}
    $$\int \frac{\dd{x}}{1+(ax)^2}=\frac{\tan^{-1}(ax)}{a}+c$$
\end{example}
\begin{example}
    $$
    \int \frac{\dd{x}}{a^2+x^2}
    =\int\frac{1}{a^2}\cdot\frac{\dd{x}}{1+\left(\frac{x}{a}\right)^2}
    =\frac{1}{a^2}\cdot\frac{\tan^{-1}\left(\frac{x}{a}\right)}{\frac{1}{a}}+c
    =\frac{\tan^{-1}\left(\frac{x}{a}\right)}{a}+c
    $$
\end{example}
\begin{example}
    $$
    \int \frac{\dd{x}}{\sqrt{a^2+x^2}}
    =\int\frac{1}{\sqrt{a^2}}\cdot\frac{\dd{x}}{\sqrt{1+\left(\frac{x}{a}\right)^2}}
    =\frac{1}{a}\cdot\frac{\sin^{-1}\left(\frac{x}{a}\right)}{\frac{1}{a}}+c
    =\sin^{-1}\left(\frac{x}{a}\right)+c
    $$
\end{example}
\begin{example}
    $$
    \int \tan x \dd{x}
    =\int \frac{\sin x}{\cos x}\dd{x}
    =-\ln|\cos x|+c
    $$
\end{example}
\begin{example}
    $$
    \int 5x\cdot e^{x^2}\dd{x}
    = \frac{5}{2}\int e^t \dd{t}
    =\frac{5}{2}e^t+c
    =\frac{5}{2}e^{x^2}+c
    $$
\end{example}
In General:
\begin{gather}
    \int f'(x)e^{f(x)}\dd{x}=e^{f(x)}+c \\
    \int f'(x)\cos(f(x))\dd{x} = \sin(f(x))+c \\
    \int \frac{f'(x)}{f(x)}\dd{x} = \ln|f(x)|+c
\end{gather}
\begin{example}
    $$
    \int \frac{2x+1}{x^2+x+17}\dd{x}
    =\ln\underbrace{(x^2+x+17)}_{\text{always positive}} +\,c
    $$
\end{example}
Completing the square:
\begin{example}
    $$
    \int \frac{8}{x^2+10x}\dd{x}
    =\int \frac{8}{(x-5)^2-25}\dd{x}
    =-\frac{8}{5}\tanh^{-1}\left(\frac{x+5}{5}\right)+c
    $$
\end{example}
\begin{example}
    $$
    \int \frac{8}{4x^2+4x+7}\dd{x}
    =\int \frac{8}{(2x+1)^2+6}\dd{x}
    =\frac{8\tan^{-1}\left(\frac{2x+1}{\sqrt 6}\right)}{2\sqrt 6}+c
    $$
\end{example}
\begin{example}
    $$
    \int \frac{e^{\tan^{-1}(x)}}{1+x^2}\dd{x}
    =\int e^t \dd{t}
    =e^t+c
    =e^{\tan^{-1}(x)}+c
    $$
\end{example}
\begin{example}
    $$
    \int \frac{\dd{x}}{\sqrt{1-3x-2x^2}}
    =\frac{1}{\sqrt 2}\int
    \frac{\dd{x}}{\sqrt{\frac{17}{16}-\left(x+\frac{3}{4}\right)^2}}
    =\frac{1}{\sqrt
    2}\sin^{-1}\left(\frac{x+\frac{3}{4}}{\sqrt{\frac{17}{16}}}\right)+c
    $$
\end{example}
\section{Examples for Integration by Parts}
\setcounter{example}{0}
\begin{example}
    $$
    \int xe^{3x}\dd{x}
    =\frac{xe^{3x}}{3}-\int \frac{e^{3x}}{3}\dd{x}
    =\frac{xe^{3x}}{3}-\frac{e^{3x}}{9}+c
    $$
\end{example}
\begin{example}
    $$
    \int \ln x \dd{x}
    =x\ln x - \int \frac{1}{x}\cdot x \dd{x} = x\ln x - x +c
    $$
\end{example}
\begin{example}
    $$
    \int \tan^{-1}x\dd{x}
    =x\tan^{-1}x - \int \frac{x}{1+x^2} \dd{x}
    =x\tan^{-1}x - \frac{1}{2}\ln(1+x^2)+c
    $$
\end{example}
\setcounter{example}{0}
Continuing from last week:
\begin{example}
    $$
        \int x^n \ln x \dd{x} =
        \frac{x^{n+1}}{n+1}\cdot \ln x - \frac{1}{1+n} \int x^n \dd{x} =
        \frac{x^{n+1}}{n+1}\cdot \ln x - \frac{x^{n+1}}{(1+n)^2} + c
    $$
\end{example}
\begin{example}
    $$
    \begin{gathered}
        \underbrace{\int x^ne^x \dd{x}}_{I_n} = x^ne^x - \underbrace{\int
        x^{n-1}e^x\dd{x}}_{I_{n-1}} \\
        I_n = x^ne^x -n\cdot I_{n-1} \quad I_0=e^x+c \\
        \frac{I_n}{n!}=\frac{x^ne^x}{n!}-\frac{I_{n-1}}{(n-1)!}
    \end{gathered}
    $$
\end{example}
\begin{example}
    $$
        \int (\ln x)^n \dd{x} =
        x(\ln x)^n - n \int (\ln x)^{n-1} \dd{x}
    $$
    And we can continue similar to how we  continued in the previous example
    using $I_n$ and $I_{n-1}$. Alternatively:
    $$
        \int (\ln x)^n \dd{x} =
        \int t^ne^t\dd{t} \qquad \ln x = t
    $$
    Which is identical to our previous example.
\end{example}
\begin{example}
    $$
    \begin{gathered}
        I_n=\int (\sin x)^n \dd{x} \\
        I_n=-(\sin x)^{n-1}\cos x +(n+1) \int (\sin x)^{n-2}(1-\sin^2 x) \dd{x}
        \\
        I_n=\frac{n-1}{n}I_{n-2}-\frac{1}{n}(\cos x)(\sin x)^{n-2} \\
        I_0 = x+c \qquad I_1=-\cos x+c
    \end{gathered}
    $$
    Alternatively, if $n$ is odd:
    $$
    \begin{gathered}
        \int \sin^{11} x \dd{x} = \int \sin x \sin^{10} x \\
        (t=\cos x, \dd{t}=-\sin x \dd{x}) \\
        \int (1-t^2)^5 \dd{t}
    \end{gathered}
    $$
    If $n$ is even:
    $$
    \begin{gathered}
        \sin^2 \alpha = \frac{1-\cos 2\alpha}{2}\\
        \int \sin^6 x \dd{x} = \int (\sin^2 x)^3 \dd{x} =
        \int \left(\frac{1-\cos 2 x}{x}\right)^3\dd{x} \\
        = \frac{1}{8} \int (1-3\cos 2x+\underbrace{3\cos^22x}_{\frac{1+\cos
        4x}{2}}-\underbrace{\cos^32x}_{\text{Like our odd example}}) \dd{x}
    \end{gathered}
    $$
\end{example}
\begin{example}
    $$\int \frac{r(t)}{\sqrt{1-t^2}}\dd{t}$$
    Where $r(t)$ is a \emph{rational function}. With substitution:
    $$\int r(\sin \theta) \dd{\theta}$$
    Which is often much easier to solve.
\end{example}
\section{Trigonometric Substitutions}
\subsection{Universal Substitution}
With integrals of rational polynomials containing $\sin x$ or $\cos x$, i.e.:
$$\int \frac{1+\cos \theta}{2+\sin \theta}\dd{\theta}\quad \int
\frac{\cos^3\theta}{1+2\sin \theta} \dd{\theta}$$
We can use the following \emph{universal substitution}:
$$
\begin{gathered}
    t=\tan \left(\frac{\theta}{2}\right) \to \dd{t} =
    \frac{1}{2}(1+t^2)\dd{\theta} \\
    \dd{\theta} = \frac{2}{1+t^2}\dd{t} \quad \tan \theta = \frac{2t}{1-t^2} \\
    \sin \theta = \frac{2t}{1+t^2} \quad \cos \theta = \frac{1-t^2}{1+t^2}
\end{gathered}
$$
\subsection{Special Case Substitutions}
\begin{enumerate}
    \item $f(\pi -\theta) = -f(\theta)$

        We can substitute $t=\sin x$.
    \item $f(-\theta) = -f(\theta)$

        We can substitute $t=\cos x$.
    \item $f(\pi+ \theta) = f(\theta)$

        We can substitute $t=\tan x$.
\end{enumerate}
\section{Integral of Rational Functions}
\begin{enumerate}
    \item Sometimes was divide the polynomials to simplify our integral such
        that the degree of the polynomial in the numerator is lower than the
        degree of the polynomial in the denominator.
    \item
        $$\int \frac{f'}{f} = \ln |f|+c$$
    \item
        $$\int \frac{\dd{x}}{ax+b} = \frac{\ln |ax+b|}{a} + c$$
        $$\int \frac{\dd{x}}{a^2\pm x} = \begin{cases}
            \frac{\tan^{-1}\left(\frac{x}{a}\right)}{a} +c\\
            \frac{\tanh^{-1}\left(\frac{x}{a}\right)}{a} +c\\
        \end{cases}$$
        This works for all fraction that you can complete the square.
    \item Partial Fractions
        $$\frac{1}{(x-a)(x-b)}=\frac{A}{x-a}+\frac{B}{x-b}$$
\end{enumerate}
\paragraph{Important Note:}
If we have a denominator of the form $(ax+b)^n$, we split into partial
fractions in the following way:
$$\frac{p_n(x)}{(ax+b)^n}=\frac{c_1}{ax+b}+\frac{c_2}{(ax+b)^2}+\dots+\frac{c_n}{(ax+b)^n}$$
\begin{example}
    $$
    \begin{gathered}
        \int \frac{x+5}{x^3-x^2-x+1}\dd{x} \\
        \int \frac{x+5}{(x-1)^2(x+1)}= \int
        \left(\frac{A}{x-1}+\frac{B}{(x-1)^2}+\frac{C}{x+1}\right)\dd{x} \\
        =A\ln|x-1|-\frac{B}{x-1}+C\ln|x+1| + c
    \end{gathered}
    $$
\end{example}
\section{Integrating Rational Functions Using Complex Numbers}
\setcounter{example}{0}
\begin{reminder}[Complex Numbers]
    $$
    \begin{gathered}
        \ln(z) = |z| + \arg(z)i \\
        z + \overline z = 2\Re(z) \\
        z - \overline z = 2i\Im(z) \\
        \tan^{-1}\left(\frac{1}{x}\right) = \frac{\pi}{2}-\tan^{-1}(x)
    \end{gathered}
    $$
\end{reminder}
\begin{example}
    $$
    \begin{gathered}
        \int \frac{x+1}{x^4+x^2}\dd{x} \\
        x^4+x^2 = x^2(x^2+1) = x^2(x-i)(x+i) \\
        \frac{x+1}{x^4+x^2}=\frac{A}{x}+\frac{B}{x^2}+\frac{C}{x+i}+\frac{D}{x-i}
        \\
        A=1,\; B=1,\; C=\frac{-1-i}{2},\; D=\frac{-1+i}{2} \\
        \frac{x+1}{x^4} = \int \left( \frac{1}{x} + \frac{1}{x^2} +
            \frac{-1-i}{2(x+i)}+\frac{-1+i}{2(x-i)} \right) \dd{x} \\
            =\ln|x| - \frac{1}{x} +
            \underbrace{\left(\frac{-1-i}{2}\right)\ln(x+i)}_{z}
            +\underbrace{\left(\frac{-1+i}{2}\right)\ln(x-i)}_{\overline z}+c
            \\
            \ln(x+i) = \ln \sqrt{x^2+1}+\tan^{-1}\left(\frac{1}{x}\right)i \\
            z+\overline z=2\Re\left(\frac{1}{2}(-1-i)\left(\ln
                \sqrt{x^2+1}+\tan^{-1}\left(\frac{1}{x}\right)i\right)\right)
                \\
            = -\ln \sqrt{x^2+1}+ \tan^{-1}\left(\frac{1}{x}\right)
            =-\frac{1}{2}\ln(x^2+1)-\tan^{-1}x+\frac{\pi}{2} \\
            \int \frac{x+1}{x^4+x^2}\dd{x} = \ln|x|-\frac{1}{x} -
            \frac{1}{2}\ln(x^2+1)-\tan^{-1}x+c
    \end{gathered}
    $$
\end{example}
\begin{example}
    $$\int \frac{x+1}{x^4+2x^2+1}\dd{x}$$
    Using real numbers:
    $$
    \begin{gathered}
        \frac{1}{2} \int \frac{2x}{(x^2+1)^2}\dd{x}+\int
        \frac{1}{(x^2+1)^2}\dd{x} \\
        x=\tan t, \quad \dd{x} = \frac{1}{\cos^2t}\dd{t} \\
        =-\frac{\frac{1}{2}}{(x^2+1)^2}+\int
        \frac{\frac{1}{\cos^2t}}{\left(\frac{1}{\cos^2t}\right)^2}\dd{t} \\
        \int \cos^2t\dd{x}=\int \frac{1+\cos 2t}{2}\dd{t} =
        \frac{1}{2}\underbrace{t}_{\tan^{-1}x}-\frac{1}{4}\underbrace{\sin 2t}_{?}+c\\
        \dots
    \end{gathered}
    $$
    In essence, using real numbers here is fairly difficult.

    However, using imaginary numbers:
    $$
    \begin{gathered}
        \int \frac{x+1}{(x+i)^2(x-i)^2} \dd{x}=
        \int \left(\frac{\frac{i}{4}}{x+i}+
        \frac{\frac{1}{4}(-1+i)}{(x+i)^2}+
        \frac{-\frac{i}{4}}{x-i}-
        \frac{\frac{1}{4}(1+i)}{(x-i)^2} \right)\dd{x} \\
        =\underbrace{\frac{i}{4}\ln(x+i)}_{z}+
        \underbrace{\left(-\frac{i}{4}\right)\ln(x-i)}_{\overline z}
        \underbrace{-\frac{\frac{1}{4}(-1+i)}{x+i}}_{+\overline \omega}
        +\underbrace{\frac{\frac{1}{4}(1+i)}{x-i}}_{\omega}+c \\
        z+ \overline z =
        2\Re(z)=\Re\left(\frac{i}{2}\left(\frac{1}{2}\ln(x^2+1)
        +i\underbrace{\left(-\tan^{-1}x+\frac{\pi}{2}\right)}_{\tan^{-1}\left(\frac{1}{x}\right)}\right)\right)
        \\
        =\frac{1}{2}\left(\tan^{-1}x+\frac{\pi}{2}\right) \\
        \omega + \overline \omega = 2 \Re(\omega) =
        \Re\left(\frac{\frac{1}{2}(1+i)}{x-i}\right) =
        \frac{1}{2}\Re\left(\frac{(1+i)(x+i)}{(x-i)(x+i)}\right) =
        \frac{1}{2}\frac{x-1}{x^2+1} \\
        \int \frac{x+1}{x^4+2x^2+1}\dd{x} =
        \frac{1}{2}\tan^{-1}x+\frac{x-1}{2(x^2+1)}+c
    \end{gathered}
    $$
\end{example}
\end{document}
