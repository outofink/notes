\documentclass[00_complete]{subfiles}

%\documentclass[12pt]{report}
\usepackage[utf8]{inputenc}
\usepackage{amsmath,amssymb,amsthm,gensymb,parskip,graphicx,footmisc,csquotes,enumerate,datetime2}
\usepackage[]{libertinus}
\usepackage[breaklinks]{hyperref}
\hypersetup{
  pdfauthor={Moshe Krumbein},
  colorlinks=true,
  linkcolor={black},
  filecolor={black},
  citecolor={black}, %blue
  urlcolor={black}, %blue
}
\usepackage[top=30mm,bottom=30mm,left=30mm,right=30mm]{geometry}
%\setlength{\emergencystretch}{2em} % prevent overfull lines
\providecommand{\tightlist}{%
\setlength{\itemsep}{0pt}\setlength{\parskip}{0pt}}

\renewcommand\qedsymbol{$\blacksquare$}

\theoremstyle{definition}
\newtheorem*{definition}{Definition}
\newtheorem*{theorem}{Theorem}
\newtheorem*{axiom}{Axiom}
\newtheorem*{lemma}{Lemma}

\theoremstyle{remark}
\newtheorem*{note}{Note}
\newtheorem*{symbols}{Symbol}
\newtheorem{example}{Example}[section]
\newtheorem*{claim}{Claim}
\newtheorem*{conclusion}{Conclusion}
\newtheorem*{reminder}{Reminder}

\usepackage{fancyhdr}
\usepackage[italicdiff]{physics}
\MakeOuterQuote{"}

\renewcommand{\chaptermark}[1]{\markboth{#1}{}}

\pagestyle{fancy}

\setlength{\headheight}{14.5pt}
\addtolength{\topmargin}{-2.5pt}

\fancyhf{}
\rhead{Moshe Krumbein}
\lhead{\chaptermark}
\cfoot{\thepage}
\fancyhead[R]{\chaptername~\thechapter}
\fancyhead[L]{\mbox{\leftmark}}

\usepackage[Rejne]{fncychap}
\usepackage{titling}

\makeatletter
\renewcommand{\@chapapp}{\vspace*{-100pt}\huge\thetitle}
\makeatother

\makeatletter
\newcommand{\subtitle}[1]{%
  {\center\vspace*{-60pt}%
  \linespread{1.1}\Large\scshape#1%
  \par\nobreak\vspace*{35pt}}
}
\makeatother

\newcommand{\Chapter}[2]{
    \def\n{#2}
    \setcounter{chapter}{\the\numexpr\n-1}
    \chapter{#1}
    \subtitle{\theauthor~- \thedate}
}

\DeclareMathOperator{\Ima}{Im}
\DeclareMathOperator{\Id}{Id}
\DeclareMathOperator{\cis}{cis}

\newcommand{\Mod}[1]{\ (\mathrm{mod}\ #1)}
\newcommand{\st}[0]{\;\mathrm{s.t.}\;}

\title{Mathematical Methods II}
\author{Moshe Krumbein}
\date{Spring 2022}

\begin{document}
\Chapter{Vector Analysis}{1}

\section{Review}
$$\int_{a}^{b}f'=[f]_a^b$$

Given an $n$\textsuperscript{th} dimensional function ($n=2,3$):

\begin{itemize}
    \item Scalar Field:
        $$f: \underbrace{D}_{\subset \mathbb{R}^n} \to \mathbb{R}$$
    \item Vector Field:
        $$\underline f: D \to \mathbb{R}^n$$
\end{itemize}

$$(D_{\underline x}\underline f)(\dd{\underline x})=(\dd{\underline
f(\underline x)})$$

\begin{gather*}
    f: \mathbb{R}^2 \to \mathbb{R} \quad f(x,y) \\
    \underline D = \begin{pmatrix}
        \pdv{f}{x} & \pdv{f}{y}
    \end{pmatrix} \\
    \underline \grad D = \begin{pmatrix}
        \pdv{f}{x} \\ \pdv{f}{y}
    \end{pmatrix}
\end{gather*}
\begin{gather*}
    f: \mathbb{R}^3 \to \mathbb{R} \quad f(x,y,z) \\
    (\underline D f)^T = \underline \grad f = \begin{pmatrix}
        \pdv{f}{x} \\ \pdv{f}{y} \\ \pdv{f}{z}
    \end{pmatrix}
\end{gather*}

\subsection{Three Dimensional Vector Field}
\begin{gather*}
    \underline f: \mathbb{R}^3 \to \mathbb{R}^3 \\
    \underline D f = \begin{pmatrix}
        \pdv{f_1}{x} & \pdv{f_2}{x} & \pdv{f_3}{x} \\
        \pdv{f_1}{y} & \pdv{f_2}{y} & \pdv{f_3}{y} \\
        \pdv{f_1}{z} & \pdv{f_2}{z} & \pdv{f_3}{z} \\
    \end{pmatrix} \\
    \Div \underline f = \div \underline f = \pdv{f_1}{x} +
    \pdv{f_2}{y} + \pdv{f_3}{z} \\
    \Curl \underline f = \curl \underline f = \begin{pmatrix}
        \pdv{f_3}{y} - \pdv{f_2}{z} \\
        \pdv{f_1}{z} - \pdv{f_3}{x} \\
        \pdv{f_2}{x} - \pdv{f_1}{y}
    \end{pmatrix}
\end{gather*}

\begin{gather*}
    \int_{a}^{b}f(x)\dd{x} \\
    \sum_{r=1}^{N}f(x_r)(x_r-x_{r-1})
\end{gather*}
\begin{gather*}
    \iint_Df \dd{A} \\
    \sum_{\text{parts}}\underline f \cdot \dd{r} \\
    \int_{a}^{b}\underline f (\underline r(t)) \cdot \underline r(t) \dd{t}
\end{gather*}
\begin{gather*}
    \iint_\Sigma = \iint_D \underline f(x(u,v),y(u,v),z(u,v)) \cdot
    \left(\pdv{f}{u} \times \pdv{f}{v}\right) \dd{u}\dd{v}
\end{gather*}
\begin{gather*}
    \iiint_R f \dd{V} \\
    \dd{S} = \|\pdv{f}{u}\times\pdv{f}{v}\|\dd{u}\dd{v} \\
    \underline{\hat N} = \frac{\pdv{f}{u}\times\pdv{f}{v}}{\|\pdv{f}{u}\times\pdv{f}{v}\|}
\end{gather*}
\begin{example}
\begin{gather*}
    \underline f(x,y) = \binom{y}{x} \quad \underline r(t)= \binom{t^2}{t^3}
    \quad 0 \leq t \leq 1 \\
    \int_c \underline f \cdot \dd{\underline r} =
    \int_{0}^{1}\binom{t^3}{t^2}\cdot\binom{2t}{3t^2}\dd{t}=
    \int_{0}^{1}5t^3\dd{t} = \left[t^3\right]_0^1=1 \\
    \int_cy\dd{x}+x\dd{y} = \int_cf_1\dd{x} + f_2\dd{y} =
    \int_c\dd(xy)=\left[xy\right]_A^B=1
\end{gather*}
\end{example}
\section{Meaning of \texorpdfstring{$\Grad$,$\;\Div$,$\;\Curl$}{grad, div, curl}}
\subsection{\texorpdfstring{$\Grad$}{grad}}
$$\grad f = \binom{\pdv{f}{x}}{\pdv{f}{y}}$$
\subsection{\texorpdfstring{$\Curl$}{curl}}
Essentially the "angular velocity".
$$\underline v = \underline \omega \times \underline r$$
\section{Integral Theorems}
\subsection{One Dimensional}
$$[\phi]_a^b=\int_{a}^{b}\phi'$$
\subsection{Two Dimensional}
$$[\phi]_A^B=\int_c\grad\phi \cdot \dd{r}$$
\subsubsection{Green's Theorem}
$$\oint_{\partial D}\dd{x}+Q\dd{y}=\iint_D\left(\pdv{Q}{x}-\pdv{P}{y}\right)\dd{x}\dd{y}$$

\subsection{Three Dimensional}
$$[\phi]_A^B=\int_c\grad\phi \cdot \dd{r}$$

\subsubsection{Stokes' Theorem}
$$\oint_{\partial \Sigma}\underline f \dd{s}=\iint_\Sigma(\curl \underline
f)\dd{S}$$

\subsubsection{Gauss's Theorem}
$$\oiint_{\partial R} \underline f \dd{\underline S} = \iiint_R (\div
\underline f)\dd{V}$$
\subsection{First Type}
\subsubsection{One Dimensional}
$$[f]_a^b=\int_{a}^{b}f'$$
\subsubsection{n-Dimensional}
$$[\phi]_A^B=\int_c\grad\phi\cdot \dd{r}$$
Suppose $\underline r = \underline r(t)$:
\begin{gather*}
    \int_c\grad \phi \cdot \dd{r} = \int_{a}^{b}\underbrace{(\grad
    \phi)(\underline r(t))}_{\grad phi}\cdot\underbrace{\underline{\dot
r}(t)\dd{t}}_{\dd{\underline r}} \\
= \int_{a}^{b}\frac{\dd}{\dd{t}}(\phi(\underline r(t)))\dd{t} =
[\phi(\underline r (t))]_a^b = \phi((\underline r(b))-\phi(\underline r(a))) \\
=\phi(B)-\phi(A)=[\phi]_A^B
\end{gather*}
\subsection{Second Type}
\subsubsection{Green's Theorem}
Where the edge is $\partial D$ and it is a closed path and with a
anticlockwise orientation:
$$\boxed{\oint_{\partial D}P\dd{x}+Q\dd{y} = \iint_D
\left(\pdv{Q}{x}-\pdv{P}{y}\right)\dd{A}}$$
\begin{proof}
    It's enough to prove:
    \begin{gather*}
        \oint_{\partial D}P \dd{x} \iint_D-\pdv{P}{y}\dd{A} \\
        \oint_{\partial D}Q \dd{y}= \iint_D\pdv{Q}{x}\dd{A}
    \end{gather*}
When $D$ is a \emph{simple region}:
$$D=\{a\leq x \leq b, c(x) \leq y \leq d(x)\}$$
\begin{gather*}
    \gamma_1:\underline r(t)=\binom{t}{c(t)}_{a \leq t \leq b} \\
    \gamma_2:\underline r(t)=\binom{t}{d(t)}_{a \leq t \leq b} \\
    \partial D = \gamma_1 \cup (-\gamma_2)
\end{gather*}
\begin{gather*}
    \oint_{\partial D}P \dd{x}=
    \int_{\gamma_1}P\underbrace{\dd{x}}_{\binom{P}{0}\dd{\underline r}} -
    \int_{\gamma_2}P\dd{x} \\
    =\int_{a}^{b}P(t,c(t))\dd{t}-\int_{a}^{b}P(t,d(t))\dd{t} \\
    \vdots
\end{gather*}
Where $D$ is not a \emph{simple region} it can be split into \emph{simple
regions}.
\end{proof}
Essentially, we see that Stokes', Green's and Gauss's theorems are all just
implementations of the first fundamental theorem of calculus.

$$\oint_{\partial D}\binom{P}{Q}\cdot \dd{\underline r}$$

is the \emph{circulation} of $\binom{P}{Q}$ on $\partial D$.
$$\frac{\oint_{\partial D}\binom{P}{Q}\cdot \dd{\underline r}}{\text{area over
$D$}} = \frac{\iint_D\pdv{Q}{x}-\pdv{P}{y}\dd{A}}{\text{area over
$D$}}=\text{average of $\pdv{Q}{x}-\pdv{P}{y}$ over $D$}$$
In other words, this is the 2D rotor (curl) over $D$.
$$\frac{1}{2}\oint_{\partial D}x\dd{y}-y\dd{x}=\text{area of $D$}$$

\section{Scalar and Vector Potential}

Suppose that $\ul f$ is an $n$-th dimensional vector field ($n=2,3$).
$$\ul f: D \to \mathbb{R}^n$$
Scalar potential is a function $V$ on $D$, such that:
$$\grad V =-\ul f$$
If $\ul f$ is a \emph{conservative} field ($\iff$):
\begin{itemize}
    \item $\int_\gamma \ul f \cdot \dd{\ul r}$ is not dependent on $\gamma$
    \item $\oint_\gamma \ul f \cdot \dd{\ul r}=0$
\end{itemize}
$$\curl \ul f = \ul 0 \Leftarrow \text{$\ul f$ is a \emph{conservative
field} ($n=3$)} \iff \oint_\gamma \ul f \cdot \dd{\ul r} =0$$
\begin{note}
    If we know that $D$ is \emph{simply connected} then we know that:
    $$\curl \ul f = \ul 0 \iff \ul f \text{ is a \emph{conservative field}}$$
\end{note}
\section{Finding Scalar Potential from a Vector Field \texorpdfstring{\ul f}{f}}
\begin{gather*}
    \int_\gamma - \grad V \cdot \dd{\ul r} = V(A)-V(B) \\
    \ul f = V (\ul r_0) - V(\ul r) \\
    \implies \boxed{V(\ul r)= \underbrace{V(\ul r_0)}_{\text{constant}}- \int_\gamma \ul f \cdot \dd{\ul r}}
\end{gather*}
\begin{note}
    If $V$ is a scalar potential of $\ul f$, then $V+c$ is also a scalar
    potential of $\ul f$.
\end{note}

If $\ul f$ is a vector field, $D=\mathbb{R}^3$, $\curl \ul f= \ul 0$ $\implies$
there exists a scalar potential for $\ul f$:
$$V=-\int_{0}^{x}f_1(t,0,0)\dd{t}-\int_{0}^{y}f_2(x,t,0)\dd{t}-\int_{0}^{z}f_3(x,y,t)\dd{t}$$
\begin{note}
    Finding a scalar potential for $f$ is equivalent to writing the
    differential as an \emph{exact differential}.
    \begin{gather*}
        f=\begin{pmatrix}
            f_1 \\f_2 \\f_3
        \end{pmatrix} \quad f_1\dd{x}+f_2\dd{y}+f_3\dd{z} \\
        \dd{V} = \pdv{V}{x}\dd{x} + \pdv{V}{y}\dd{y} + \pdv{V}{z}\dd{z}
    \end{gather*}
\end{note}
\end{document}
