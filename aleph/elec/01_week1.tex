\documentclass[00_complete]{subfiles}

%\documentclass[12pt]{report}
\usepackage[utf8]{inputenc}
\usepackage{amsmath,amssymb,amsthm,gensymb,parskip,graphicx,footmisc,csquotes,enumerate,datetime2}
\usepackage[]{libertinus}
\usepackage[breaklinks]{hyperref}
\hypersetup{
  pdfauthor={Moshe Krumbein},
  colorlinks=true,
  linkcolor={black},
  filecolor={black},
  citecolor={black}, %blue
  urlcolor={black}, %blue
}
\usepackage[top=30mm,bottom=30mm,left=30mm,right=30mm]{geometry}
%\setlength{\emergencystretch}{2em} % prevent overfull lines
\providecommand{\tightlist}{%
\setlength{\itemsep}{0pt}\setlength{\parskip}{0pt}}

\renewcommand\qedsymbol{$\blacksquare$}

\theoremstyle{definition}
\newtheorem*{definition}{Definition}
\newtheorem*{theorem}{Theorem}
\newtheorem*{axiom}{Axiom}
\newtheorem*{lemma}{Lemma}

\theoremstyle{remark}
\newtheorem*{note}{Note}
\newtheorem*{symbols}{Symbol}
\newtheorem{example}{Example}[section]
\newtheorem*{claim}{Claim}
\newtheorem*{conclusion}{Conclusion}
\newtheorem*{reminder}{Reminder}

\usepackage{fancyhdr}
\usepackage[italicdiff]{physics}
\MakeOuterQuote{"}

\renewcommand{\chaptermark}[1]{\markboth{#1}{}}

\pagestyle{fancy}

\setlength{\headheight}{14.5pt}
\addtolength{\topmargin}{-2.5pt}

\fancyhf{}
\rhead{Moshe Krumbein}
\lhead{\chaptermark}
\cfoot{\thepage}
\fancyhead[R]{\chaptername~\thechapter}
\fancyhead[L]{\mbox{\leftmark}}

\usepackage[Rejne]{fncychap}
\usepackage{titling}

\makeatletter
\renewcommand{\@chapapp}{\vspace*{-100pt}\huge\thetitle}
\makeatother

\makeatletter
\newcommand{\subtitle}[1]{%
  {\center\vspace*{-60pt}%
  \linespread{1.1}\Large\scshape#1%
  \par\nobreak\vspace*{35pt}}
}
\makeatother

\newcommand{\Chapter}[2]{
    \def\n{#2}
    \setcounter{chapter}{\the\numexpr\n-1}
    \chapter{#1}
    \subtitle{\theauthor~- \thedate}
}

\DeclareMathOperator{\Ima}{Im}
\DeclareMathOperator{\Id}{Id}
\DeclareMathOperator{\cis}{cis}

\newcommand{\Mod}[1]{\ (\mathrm{mod}\ #1)}
\newcommand{\st}[0]{\;\mathrm{s.t.}\;}

\title{Electricity and Magnetism}
\author{Moshe Krumbein}
\date{Spring 2022}

\begin{document}
\Chapter{Introduction}{1}

\begin{definition}[Operator]
An \emph{operator} is an mathematical object which takes a \emph{function} and
returns a \emph{function}.

For example, the \emph{derivative}: $\frac{\dd{f}}{\dd{x}}$.
\begin{note}
    This is not a fraction!
\end{note}
\end{definition}
The most common \emph{operator} is electromagnatism is the \emph{nabla}
($\grad$):
$$\vec \grad =\frac{\partial}{\partial x} \hat \imath + \frac{\partial}{\partial
y} \hat \jmath + \frac{\partial}{\partial z} \hat k$$
\section{Gradient}

$$\vec f(x,y,z)= \grad \phi = \pdv{\phi}{x}\hat \imath + \pdv{\phi}{y}\hat
\jmath + \pdv{\phi}{z} \hat k = \Grad(\phi)$$

What can we use the \emph{gradient} for?

\subsection{Directional derivative}
The \emph{derivative} of the function in the certain direction.
$$\pdv{\phi}{\hat n}=\grad \phi \cdot \hat n$$
Where $\hat n$ is the unit vector in the desired direction.
\subsubsection{Conservative Fields}
$$\phi(\vec r_2)-\phi(\vec r_1)=\int_\gamma \grad \phi \cdot \dd{\vec r}$$

\section{Divergence}
$$\vec f(x,y,z)=f_x(x,y,z)\hat \imath + f_y(x,y,z)\hat \jmath+f_z(x,y,z)\hat k$$
$$\Div(\vec f)= \div \vec f =
\frac{\dd{f_x}}{\dd{x}}+\frac{\dd{f_y}}{\dd{y}}+\frac{\dd{f_z}}{\dd{z}}$$
It's much easier to work in cases that are symmetrical (cylinders and infinite
planes).
$$\vec f(x,y,z)=xy\hat x + yz\hat y + xz\hat z \quad \div \vec f = y+z+x$$
\subsubsection{Integration on Surfaces}

Suppose we have a vector field $\vec f (x,y,z)$ and a surface defined by
$s(x,y,z)=c$ (a sphere).
$$\iint\limits_S \vec f(x,y,z)\dd{\vec s} \quad \dd{\vec s}=\dd{s} \cdot \hat n$$
\begin{note}
    We define the direction to be \emph{outward}.
\end{note}
$$\dd{s}=R^2\sin(\theta)\dd{\theta}\dd{\phi}\hat r \quad 0 \leq \theta \leq
\pi, \quad 0 \leq \phi \leq 2\pi$$
$$\vec f (\vec r)=\vec r$$
$$\int \vec f (\vec r)\dd{\vec r} = \oiint \hat r R^2 \sin
(\theta)\dd{\phi}\dd{\theta}\cdot \hat r = R^2
\oiint\sin(\theta)\dd{\phi}\dd{\theta}= 4 \pi R^2$$
\subsection{Gauss's Law}
$$\oiint\limits_{S(V)}\vec f \cdot \dd{\vec s}= \iiint\limits_V(\div \vec
f) \cdot \dd{V}$$
Where $S(V)$ is the area which encloses the $V$.
\section{Curl}
\begin{gather*}
    f(x,y,z)=f_x\hat x + f_y \hat y + f_z\hat z \\
    \Curl(f)=\curl f = \left(\pdv{f_z}{y}-\pdv{f_y}{z}\right)\hat x
    + \left(\pdv{f_x}{z}-\pdv{f_z}{x}\right)\hat y
    + \left(\pdv{f_y}{x}-\pdv{f_x}{y}\right)\hat z
\end{gather*}
\subsection{Stokes' Law}
$$\oint\limits_{l(A)}\vec f \cdot \dd{\vec l} = \int\limits_A(\curl \vec
f)\dd{\vec s} \quad \dd{\vec s}=\dd{s}\cdot \hat n$$
\begin{note}
    We define the direction using the \emph{right hand rule}.
\end{note}
\section{What is electricity?}
$$e=1.602\cdot 10^{-19}\;\mathrm{C}$$
There exists \emph{conservation of charge}.
\subsection{Coulomb's Law}
\begin{definition}[Coulomb's Law]
    \emph{Coulomb's Law} defines the vector (magnitude and direction) of the
    force between two charges in space given charges $q_1, q_2$ at distances
    $r_1, r_2$ respectively.
    \begin{gather*}
        \vec r_{1,2}=\vec r_1 - \vec r_2 \quad \hat r_{1,2}=\frac{\vec r_1 -
        \vec r_2}{\|\vec r_1 - \vec r_2\|} \\
        \vec F_{1,2}= \frac{kq_1q_2}{\|\vec r_{1,2}\|^2}\hat r_{1,2}
    \end{gather*}
    Where $k=8.99\cdot 10^9 \frac{\mathrm{N}\cdot \mathrm{m}^2}{\mathrm{C}^2}$.
    Sometimes we define $k$ in terms of $\varepsilon_0$ (dielectric constant in a
    vacuum):
    $$k=\frac{1}{4\pi\varepsilon_0} \quad \varepsilon_0=8.8542\cdot
    10^{12}\frac{\mathrm{C}^2}{\mathrm{N}\cdot\mathrm{m}^2}$$
    More simply, we say:
    $$\vec F = \frac{kq_1q_2}{r^2}\hat r$$
\end{definition}
\section{Superposition Principle}

\begin{definition}[Superposition Principle]
    Given a system of charges, the force on a single particle is the
    \emph{vector sum} of all of the forces in the system on that particle.
\end{definition}
In order to more simple use the superposition principle, we can use:
$$\frac{\hat r_{1,2}}{\|\vec r_{1,2}\|^2}=\frac{\vec r_{1,2}}{\|\vec r_{1,2}\|^3}$$
$$\vec F_i=\sum_{i\neq j}\frac{kq_iq_j}{\|\vec r_i-\vec r_j\|}(\vec r_i - \vec
r_j)$$
\subsection{Work and Energy}
\begin{gather*}
    \vec F_{2,1}=k\frac{q_1q_2}{|r'_2-r_1|^2}\hat r'_{2,1} \\
W = - \int_\infty^{r_2} \vec F_{2,1}\cdot \dd{r'_2}
\end{gather*}
\begin{note}
    The work being done is \emph{conservative}, which means it is not dependent
    on the path taken.
\end{note}
Doing a change a variables:
\begin{gather*}
    \overline r' = \overline r'_2 - \overline r_1 \\
    |\overline r_2' - \overline r_1| = |r'| \\
    \dd{r'}=\dd{r_2'}
\end{gather*}
Back to finding the work:
\begin{gather*}
    W = -\int_{\infty}^{\overline r_2 - \overline
    r_1}k\frac{q_1q_2}{|r'|^2}\dd{r'}  =
    k\frac{q_1q_2}{|r'|}\Bigr|_{\infty}^{\overline r_2-\overline r_1} =
    \frac{kq_1q_2}{|\overline r_2 - \overline r_1|}
\end{gather*}
The about of work required to bring $q_2$ from infinity to distance $r$ from
$r_1$ is equal to the amount of work $q_1$ needs to push $q_2$ from $r$ to
infinity.

If we wanted to bring in a third particle $q_3$:
\begin{gather*}
    U_{1,2,3}= \frac{kq_1q_2}{|\overline r_2 -\overline r_1|} +
    \frac{kq_1q_3}{|\overline r_3 - \overline r_1|} + \frac{kq_2q_3}{|\overline
    r_3 - \overline r_2|} \\
    U=\sum_{i}\sum_{j<i}\frac{kq_iq_j}{|\overline r_i-\overline r_j|} =
    \frac{1}{2}\sum_i\sum_{i \neq j}\frac{kq_iq_j}{|\overline r_i - \overline
    r_j|}
\end{gather*}
\section{Electric Fields}

Given a fixed charge $q_0$ and a free charge $q_1$:
\begin{gather*}
    \overline F_{0,1}=k\frac{q_0q_1}{|\vec r_0-\vec r_1|^2}\hat r_{0,1} \\
    \vec E(\vec r_0)=\frac{\vec F(\vec r_0)}{q_0}=k\frac{q_1}{|\vec r_0 - \vec
    r_1|^2}\hat r_{0,1} \left[\frac{\mathrm N}{\mathrm C}\right]
\end{gather*}
Where $\vec E$ represents the electric field and $\vec r_0$ is the location at
which we are measuring the field.

For an arbitrary number of points in the field:
\begin{gather*}
    \vec E(\vec r_0)=\sum_i\frac{kq_i}{|\vec r_0-\vec r_i|^2}\hat r_{0,i}
\end{gather*}
And assuming there's one large charge $Q$:
\begin{gather*}
    \vec E(\vec r)=k\frac{Q}{r^2}\hat r
\end{gather*}
\subsection{Characteristics}
\begin{itemize}
    \item $\oint \vec E \cdot \dd{ \vec \ell} = 0$
    \item $\curl \vec E = 0$
\end{itemize}
\section{Flux}
\begin{definition}[Flux]
\begin{gather*}
    \phi = EA
\end{gather*}
Where $\phi$ is \emph{flux}, $E$ is the electric field, and $A$ is the area.
\end{definition}
If the field and the area are perpendicular, then we can say:
\begin{gather*}
    \phi = \vec E \cdot \vec A = |E||A|\cos\theta \\
    \phi = \int \vec E \cdot \dd{\vec s}
\end{gather*}
\section{Charge Densities}
\begin{definition}[Volume Charge Density]
\begin{gather*}
    \rho \left[\frac{\mathrm{C}}{\mathrm{m^3}}\right] \implies q = \int_V \rho
    \dd{V}
\end{gather*}
\end{definition}
\begin{definition}[Surface Charge Density]
\begin{gather*}
    \sigma \left[\frac{\mathrm{C}}{\mathrm{m^2}}\right] \implies q = \int_S
    \sigma \dd{S}
\end{gather*}
\end{definition}
\begin{definition}[Linear Charge Density]
\begin{gather*}
    \lambda \left[\frac{\mathrm{C}}{\mathrm{m}}\right] \implies q = \int_\ell
    \lambda \dd{\ell}
\end{gather*}
\end{definition}
\section{Electric Field}
Per \emph{Coulomb's Law}, we can build a "force field" by a single charge $q_1$:
$$\vec F(\vec r_0)= \frac{kq_0q_1}{|\vec r_0-\vec r_1|^2}\hat r_{0,1}$$
Where $q_0$ is a \emph{test charge} at any point in space.
From this, we can build an \emph{electric field}:
$$\vec E(\vec r_0) =\frac{\vec F(\vec r_0)}{q_0}= \frac{kq_1}{|\vec r_0-\vec r_1|^2}\hat r_{0,1}$$
Or in general:
$$\vec E(\vec r_0) =\frac{\vec F(\vec r_0)}{q_0}= \sum_i\frac{kq_i}{|\vec r_0-\vec
r_i|^2}\hat r_{0,i}$$
\begin{example}
    Suppose we have a charged two-dimensional disk on the $xy$-plane with
    radius $r$ and \emph{charge density} $\sigma$ (constant). What in the
    \emph{electric field} on the $z$-axis?
    \begin{gather*}
        \vec E(z)=\int\frac{k\sigma \dd{s'}}{|\vec r - \vec r'|^2}\hat r_{0,'}
        \quad \hat r_{0,'} = \frac{\vec r -\vec r'}{|\vec r - \vec r'|} \implies \\
        \vec E(z)=\int k\frac{\sigma(\vec r- \vec r')\dd{s'}}{|\vec r -\vec r'|^3}
        \\
        \vec r (0,0,z) \quad \vec r' (r'\cos(\phi'),r'\sin(\phi'),0) \\
        0 \leq  r' \leq R \quad 0 > \phi' \leq 2 \pi \\
        \vec r -\vec r' = (-r'\cos(\phi'),-r'\sin(\phi'),z) \\
        |\vec r- \vec r'|^3=(r'^2+z^2)^\frac{3}{2} \\
        \dd{s'}=r'\dd{r'}\dd{\phi'} \\
        \vec E(z)=\int_0^{2\pi}\int_{0}^{R} \frac{k\sigma}
        {(r'^2+z^2)^\frac{3}{2}}\cdot(-r'\cos{\phi'},-r'\sin{\phi'},z)\cdot
        r' \dd{r'}\dd{\phi'} \\
        =2 \pi k \sigma z \int_{0}^{R}\frac{r' \dd{r'}}{(r'^2+z^2)^\frac{3}{2}}
        \\
        \vec E(z)=2 \pi k \sigma z \hat z \cdot
        \frac{-1}{\sqrt{r'^2+z^2}}\biggr|_0^R = \left(2\pi k\sigma -
        \frac{2\pi k \sigma z}{\sqrt{R^2+z^2}}\right)\hat z \\
        \boxed{\vec E (z \to 0)=2\pi k \sigma \hat z}
    \end{gather*}
    Which is the \emph{electric field} for an infinite plane.
    \begin{gather*}
        z\gg R \\
        \vec E(z) \approx 2 \pi k \sigma \left( 1
        \left(-\frac{R^2}{2z^2}\right)\right)\hat z = \frac{\pi
        R^2k\sigma}{z^2}\hat z = \frac{kq}{z^2}\hat z
    \end{gather*}
    Which is reminiscent of \emph{Coulomb's law}, which makes sense
    intuitively, for something with a finite radius from very far away appears
    and behaves point-like.
\end{example}
\section{Gauss's Law}
Where $\Phi$ is the \emph{electric flux} through a closed surface $S$:
\begin{gather*}
    \Phi = \oint \vec E \cdot \dd{\vec S} = \frac{q}{\varepsilon_0} = 4 \pi k
    \cdot q
\end{gather*}
This expresses that in order to find the \emph{flux} on a closed surface,
we only need to know how much charge in within the surface, regardless of the
shape of the surface and the charges outside the surface.
$$\boxed{\oint \vec E \dd{\vec S} = \frac{q}{\varepsilon_0}} \quad \text{(Integral
Form)}$$

\emph{Gauss's law} is useful because it allows us to choose whatever surface
we would like, namely a sphere, and to take advantage of its symmetries, in
order to simplify our calculations.

\subsection{Charged sphere without thickness of radius a}
\begin{gather*}
    \vec E(r) = \begin{cases}
        0 & r < a \\
        \frac{kq}{r^2}\hat r & a > a \quad q = 4 \pi a^2 \sigma
    \end{cases}
\end{gather*}

\subsection{Differential form of Gauss's Law}

\begin{gather*}\oiint \vec E \dd{\vec S} = \frac{q}{\varepsilon_0} =
    \frac{1}{\varepsilon_0}\iiint \rho \dd{V}
\end{gather*}
We can used \emph{Gauss's theorem} (\emph{divergence theorem}), which states:
$$\oiint_{S(\sigma)} \vec E \dd{s} =\iiint_\sigma \div \vec E \dd{V}$$
which provides:
\begin{gather*}
    \int \div \vec E \dd{V} = \int \frac{\rho}{\varepsilon_0} \dd{V} \\
    \implies \boxed{\div \vec E = \frac{\rho}{\varepsilon_0}}
\end{gather*}
\end{document}
