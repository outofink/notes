\documentclass[00_complete]{subfiles}

%\documentclass[12pt]{report}
\usepackage[utf8]{inputenc}
\usepackage{amsmath,amssymb,amsthm,gensymb,parskip,graphicx,footmisc,csquotes,enumerate,datetime2}
\usepackage[]{libertinus}
\usepackage[breaklinks]{hyperref}
\hypersetup{
  pdfauthor={Moshe Krumbein},
  colorlinks=true,
  linkcolor={black},
  filecolor={black},
  citecolor={black}, %blue
  urlcolor={black}, %blue
}
\usepackage[top=30mm,bottom=30mm,left=30mm,right=30mm]{geometry}
%\setlength{\emergencystretch}{2em} % prevent overfull lines
\providecommand{\tightlist}{%
\setlength{\itemsep}{0pt}\setlength{\parskip}{0pt}}

\renewcommand\qedsymbol{$\blacksquare$}

\theoremstyle{definition}
\newtheorem*{definition}{Definition}
\newtheorem*{theorem}{Theorem}
\newtheorem*{axiom}{Axiom}
\newtheorem*{lemma}{Lemma}

\theoremstyle{remark}
\newtheorem*{note}{Note}
\newtheorem*{symbols}{Symbol}
\newtheorem{example}{Example}[section]
\newtheorem*{claim}{Claim}
\newtheorem*{conclusion}{Conclusion}
\newtheorem*{reminder}{Reminder}

\usepackage{fancyhdr}
\usepackage[italicdiff]{physics}
\MakeOuterQuote{"}

\renewcommand{\chaptermark}[1]{\markboth{#1}{}}

\pagestyle{fancy}

\setlength{\headheight}{14.5pt}
\addtolength{\topmargin}{-2.5pt}

\fancyhf{}
\rhead{Moshe Krumbein}
\lhead{\chaptermark}
\cfoot{\thepage}
\fancyhead[R]{\chaptername~\thechapter}
\fancyhead[L]{\mbox{\leftmark}}

\usepackage[Rejne]{fncychap}
\usepackage{titling}

\makeatletter
\renewcommand{\@chapapp}{\vspace*{-100pt}\huge\thetitle}
\makeatother

\makeatletter
\newcommand{\subtitle}[1]{%
  {\center\vspace*{-60pt}%
  \linespread{1.1}\Large\scshape#1%
  \par\nobreak\vspace*{35pt}}
}
\makeatother

\newcommand{\Chapter}[2]{
    \def\n{#2}
    \setcounter{chapter}{\the\numexpr\n-1}
    \chapter{#1}
    \subtitle{\theauthor~- \thedate}
}

\DeclareMathOperator{\Ima}{Im}
\DeclareMathOperator{\Id}{Id}
\DeclareMathOperator{\cis}{cis}

\newcommand{\Mod}[1]{\ (\mathrm{mod}\ #1)}
\newcommand{\st}[0]{\;\mathrm{s.t.}\;}

\title{Discrete Mathematics}
\author{Moshe Krumbein}
\date{Fall 2021}

\begin{document}
\Chapter{Counting Problems}{5}

\section{Introduction}

Essentially, there are four basic counting problems:
\begin{table}[ht]
\centering
{\renewcommand{\arraystretch}{1.2}% for the vertical padding
\begin{tabular}{ccc}
 \hline
& Repeats & No Repeats \\
 \hline
    Order is important & ?&? \\
    Order is unimportant & ?&? \\
 \hline
\end{tabular}}
\caption{Types of counting problems}
\end{table}

\section{Order is important}
\subsection{Repeats}
When order is important but there are repeats, we can simply express the
number of ways to place $n$ different balls in $k$ places as:
$$n^k$$
\subsection{No Repeats}

If $n=k$, we get the number of permutations on $[n]$, which is $n!$.
\begin{gather*}
    D=\{f:[k]\to[n] \mid f \text{ is injective}, \forall i \in [k]: f(i)=n\} (n
    \notin Imf) \\
    |D|= \frac{(n-1)!}{(n-1-k)!}
\end{gather*}
\begin{gather*}
    E=\{f:[k]\to[n] \mid f \text{ is injective}, \exists! i \in [k]: f(i)=n\}
    \\
    \forall i \in[k]: E_i = \{f \in E \mid f(i) = n \} \\
    |E_i| =\frac{(n-1)!}{n-1-(k-1)} = \frac{(n-1)!}{(n-k)!} \implies |E| = \frac{k(n-1)!}{(n-k)!}
    = \frac{n!}{(n-k)!} - \frac{(n-1)!}{(n-1-k)!}
\end{gather*}

Simply put, the number of ways to place $n$ distinct balls in $k$ places
without repeats is:
$$\frac{n!}{(n-k)!}$$

\section{Order is unimportant, without repeats}

Instead of dealing with sequences, we're dealing with sets.

In essence, when order is unimportant, we just divide the answer where order is
important by $k!$:
$$\frac{n!}{k!(n-k)!}$$

This is also known as the \emph{binomial coefficient}, symbolized as:
$$\binom{n}{k}$$

\subsection{Characteristics of binomial coefficients}
\begin{enumerate}
    \item $\displaystyle \binom{n}{0} = 1, \binom{n}{n} = 1, \binom{n}{1} = n = \binom{n}{n-1}$
    \item $\displaystyle \binom{n}{k} \in \mathbb{N} \quad (n \in \mathbb{N}\cup\{0\}, 0 \leq k \leq
   n)$
    \item Symmetry: $\displaystyle \binom{n}{k} = \binom{n}{n-k}$

\begin{symbols}
\begin{gather*}
[n] = \{1,2, \dots, n\}, \binom{[n]}{k} = \{A \subseteq [n] \mid |A| = k \} \\
\left|\binom{[n]}{k}\right| = \binom{n}{k}
\end{gather*}
\end{symbols}

\begin{proof}
\begin{gather*}
    f : \binom{[n]}{k} \to \binom{[n]}{n-k} \\
    A \in \binom{[n]}{k}, \quad f(A)=([n]\setminus A) \in \binom{[n]}{n-k} \\
    f(f(A)) = A \quad B \in \binom{[n]}{n-k}, f(f(B))=B
\end{gather*}
\end{proof}

Algebraically:
\begin{gather*}
\binom{n}{n-k} = \frac{n!}{(n-k)!(n-(n-k))!} = \frac{n!}{k!(n-k)!} = \binom{n}{k} \\
\binom{n}{k}=\binom{n}{n-k}
\end{gather*}


Back to characteristics:
\item Pascal's rule:
$$\binom{n-1}{k} + \binom{n-1}{k-1} = \binom{n}{k}$$
\end{enumerate}

\section{Order is unimportant, with repeats}

\subsection{Multinomial Coefficients}

If we had to order $k_1$ balls of the color $1$, $k_2$ balls of color $2$,
$\dots k_m$ balls of color $m$, how many ways can we sort them (where the order
of balls of the same color does not matter)?
$$
    n = \sum_{i=1}^m k_i, \quad \frac{n!}{k_1!k_2!\dots k_m!}
    = \binom{n}{k_1,k_2,\dots,k_m} -
    \text{\emph{multinomial coefficient}}
$$

As a generalization:
$$\frac{(a+b+c)!}{a!\;b!\;c!} =\binom{a+b+c}{a,b,c} = \binom{a+b+c}{c}\cdot\binom{a+b}{b}\cdot\binom{a}{a}$$

If we apply Pascal's rule to multinomial coefficients, we receive the following
generalization:
$$\binom{a+b+c}{a,b,c} = \binom{(a+b+c)-1}{a-1,b,c}+\binom{(a+b+c)-1}{a,b-1,c}+\binom{(a+b+c)-1}{a,b,c-1}$$

\subsection{Different ways to ask the same question}

\begin{enumerate}
\item How many ways can we distribute $k$ items to $n$ people?
\item How many ways can we pick $k$ items from $n$ different piles?
\item How many solutions (whole, positive) to the following equation:
$$x_1+x_2+\dots+x_n=k$$
\end{enumerate}

A helpful way to view this problem is by asking how many ways can we place
$n-1$ dividers among between $k$ places? (which really means if dividers take
up space then how many ways to place $n-1$ items over $k + (n-1)$ places?)
$$\binom{k+(n-1)}{n-1} = \boxed{\binom{n+k-1}{k}}$$

\section{Conclusion}

\begin{table}[ht]
\centering
{\renewcommand{\arraystretch}{1.2}% for the vertical padding
\begin{tabular}{ccc}
 \hline
& Repeats & No Repeats \\
 \hline
 Order is important & $n^k$ & $\dfrac{n!}{(n-k)!}$ \\
 Order is unimportant & $\displaystyle\binom{n+k-1}{k}$ & $\displaystyle\binom{n}{k}=\frac{n!}{k!(n-k)!}$ \\
 \hline
\end{tabular}}
\caption{Our final table}
\end{table}
\end{document}
