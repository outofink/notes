\documentclass[00_complete]{subfiles}

%\documentclass[12pt]{report}
\usepackage[utf8]{inputenc}
\usepackage{amsmath,amssymb,amsthm,gensymb,parskip,graphicx,footmisc,csquotes,enumerate,datetime2}
\usepackage[]{libertinus}
\usepackage[breaklinks]{hyperref}
\hypersetup{
  pdfauthor={Moshe Krumbein},
  colorlinks=true,
  linkcolor={black},
  filecolor={black},
  citecolor={black}, %blue
  urlcolor={black}, %blue
}
\usepackage[top=30mm,bottom=30mm,left=30mm,right=30mm]{geometry}
%\setlength{\emergencystretch}{2em} % prevent overfull lines
\providecommand{\tightlist}{%
\setlength{\itemsep}{0pt}\setlength{\parskip}{0pt}}

\renewcommand\qedsymbol{$\blacksquare$}

\theoremstyle{definition}
\newtheorem*{definition}{Definition}
\newtheorem*{theorem}{Theorem}
\newtheorem*{axiom}{Axiom}
\newtheorem*{lemma}{Lemma}

\theoremstyle{remark}
\newtheorem*{note}{Note}
\newtheorem*{symbols}{Symbol}
\newtheorem{example}{Example}[section]
\newtheorem*{claim}{Claim}
\newtheorem*{conclusion}{Conclusion}
\newtheorem*{reminder}{Reminder}

\usepackage{fancyhdr}
\usepackage[italicdiff]{physics}
\MakeOuterQuote{"}

\renewcommand{\chaptermark}[1]{\markboth{#1}{}}

\pagestyle{fancy}

\setlength{\headheight}{14.5pt}
\addtolength{\topmargin}{-2.5pt}

\fancyhf{}
\rhead{Moshe Krumbein}
\lhead{\chaptermark}
\cfoot{\thepage}
\fancyhead[R]{\chaptername~\thechapter}
\fancyhead[L]{\mbox{\leftmark}}

\usepackage[Rejne]{fncychap}
\usepackage{titling}

\makeatletter
\renewcommand{\@chapapp}{\vspace*{-100pt}\huge\thetitle}
\makeatother

\makeatletter
\newcommand{\subtitle}[1]{%
  {\center\vspace*{-60pt}%
  \linespread{1.1}\Large\scshape#1%
  \par\nobreak\vspace*{35pt}}
}
\makeatother

\newcommand{\Chapter}[2]{
    \def\n{#2}
    \setcounter{chapter}{\the\numexpr\n-1}
    \chapter{#1}
    \subtitle{\theauthor~- \thedate}
}

\DeclareMathOperator{\Ima}{Im}
\DeclareMathOperator{\Id}{Id}
\DeclareMathOperator{\cis}{cis}

\newcommand{\Mod}[1]{\ (\mathrm{mod}\ #1)}
\newcommand{\st}[0]{\;\mathrm{s.t.}\;}

\title{Discrete Mathematics}
\author{Moshe Krumbein}
\date{Fall 2021}

\begin{document}
\Chapter{Introduction to Set Theory and Logic}{1}

\section{Sets}
\begin{definition}[Discrete Sets]
A \emph{discrete set} is a set that has a distinct, individualized parts.

Discrete sets may be finite or infinite (but countable).
\end{definition}

\begin{definition}[Sets]
A \emph{set} is a collection of objects.

A set of objects of does not have to contain objects of only one type.

A property of sets is that a set $A$ has a \emph{binary relation} between
itself and an object $o$. Either $A$ contains or does not contain $o$.
\end{definition}

\begin{example}
\begin{gather*}
    A=\{1,4,7,8\} \\
    B=\{a,b,c,d\} \\
    C=\{x,y,z\} \\
    \\
    1 \in A \\
    2 \notin A
\end{gather*}
\end{example}

Additionally, a set that has multiple members of the same element is
equivalent to a set that has one one of the element, due to the property of
\emph{binary relation}.

$$A = \{1,4,7,8\} = \{1,1,4,4,4,7,8\}$$

\begin{definition}[Size of a Sets]
    Number of (by definition, distinct) elements in a set.
    $$|A| = 4$$
\end{definition}

\begin{definition}[Conditional Set]
$$\{ x \;|\; \text{such that } x \}$$
$$B = \{a \in A \;|\; a < 5 \}$$
\begin{example}
All of the whole numbers from 1 to 1000.
$$\{x \;|\; 1<x<1000, \text{ such that $x$ is whole }\}$$
\end{example}

Although the following set is small in size, it is easier to express it in
the following fashion instead of explicitly:
$$\{x \;|\; x^5-4x^3+7x^2-11x+\sqrt 2=0\}$$
\end{definition}

\begin{symbols}[Quantifiers]
\begin{gather*}
    \forall = \text{for all} \\
    \exists = \text{exists}
\end{gather*}

\end{symbols}

\subsection{Containment (Subset)}
\begin{definition}[Containment]
$B$ contains $A$ when all elements in $B$ are also in $A$.
$$B \subseteq A$$
Specifically if $A$ and $B$ are distinct:
$$B \subsetneq A$$
If two sets are \emph{equivalent}: ($A \subseteq B, B \subseteq A$)
$$A=B$$
\end{definition}

\begin{example}
\begin{gather*}
    A = \{1,3,4, \{1\}, \{1,2\}\} \\
    |A| = 5 \\
    1 \in A \\
    \{1\} \in A \\
    \{1\} \subseteq A \\
    \text{ (because of the first element)} \\
    \{\{1\}\} \subseteq A \\
    \text{(because of the forth element)} \\
    \{1,2\} \in A \\
    \{1,2\} \nsubseteq A \\
    \{1\} \subseteq A \\
    \{1\} \subsetneq A \\
\end{gather*}
\end{example}

\begin{note}
$A \in B$: the \emph{element} $A$ is in \emph{set} $B$.
$A \subseteq B$: the contents of \emph{set} $A$ are all in \emph{set} $B$.
\end{note}

\subsection{Special Sets}

\begin{align}
    \emptyset  =& \{\} \tag{\text{empty set}} \\
    \mathbb{N} =& \{1,2,3,\mathellipsis\} \tag{\text{natural numbers}} \\
    \mathbb{Z} =& \{\mathellipsis,-2,-1,0,1,2,3,\mathellipsis\}
    \tag{\text{whole numbers}} \\
    \mathbb{Q} =& \left\{ \frac{m}{n} \;|\; m,n \in \mathbb{Z}, n \neq 0
    \right\}\tag{\text{rational numbers}} \\
        \mathbb{R} \quad& \tag{\text{real numbers}}
\end{align}
$$
    \emptyset \subsetneq
    \mathbb{N} \subsetneq
    \mathbb{Z} \subsetneq
    \mathbb{Q} \subsetneq
    \mathbb{R}
$$

\subsection{Binary operations on sets}
\begin{align}
    A \cup B =& \{a \;|\; a \in B \text{ or } a \in A \} \tag{\text{Union}} \\
    A \cap B =& \{a \;|\; a \in B \text{ and } a \in A \}
    \tag{\text{Intersection}} \\
    A \setminus B =& \{a \;|\; a \notin B \text{ and } a \in A \}
    \tag{\text{Difference}} \\
    A \triangle B =& (A \cup B) \setminus (A \cap B) \tag{\text{Symmetrical
    Difference}}
\end{align}

\subsection{Size of union of sets}

$$|A \cup B| \overset{?}{=} |A| + |B|$$

\begin{definition}[Disjoint Sets]
$A$ and $B$ are \emph{disjoint sets} if $A \cap B = \emptyset$.

If $A$ and $B$ are disjoint sets, then:
$$|A \cup B| = |A| + |B|$$
\end{definition}
In general:
$$|A \cup B| = |A| + |B|- |A \cap B|$$

When $A, B, C$ are \emph{pairwise disjointed}:
$$|A \cup B \cup C| = |A| + |B| + |C|$$

\subsection{Algebraic Properties}

\begin{gather*}
    \text{Commutative property:} \\
    A \cup B = B \cup A \\
    A \cap B = B \cap A \\
    \text{Associative property:} \\
    (A \cup B) \cup C = A \cup (B \cup C) \\
    (A \cap B) \cap C = A \cap (B \cap C) \\
    \text{Distributive property:} \\
    A \cup (B \cap C) = (A \cup B) \cap (A \cup C) \\
    A \cap (B \cup C) = (A \cap B) \cup (A \cap C) \\
\end{gather*}

Given the sets $A_1, A_2. \mathellipsis, A_n$:
\begin{gather*}
    [n] = \{1,2,3, \mathellipsis, n\}, n \in \mathbb{N},\quad |[n]|=n \\
    \bigcup_{i=1}^{n}A_i=A_1\cup A_2 \cup \mathellipsis \cup A_n = \{ x \;|\;
    \exists \; i \in [n] : x \in A_i \} \\
    \bigcap_{i=1}^{n}A_i=A_1\cap A_2 \cap \mathellipsis \cap A_n = \{ x \;|\;
    \forall \; i \in [n] : x \in A_i \}
\end{gather*}

\begin{definition}[Compliment]
For set $U$ where $A \subseteq U$:
$$A^c = U \setminus A$$
\end{definition}

\begin{definition}[De Morgan's Laws]
Given sets $A,B$:
\begin{gather*}
(A \cup B)^c=A^c \cap B^c \\
(A \cap B)^c=A^c \cup B^c
\end{gather*}
\end{definition}

\section{Logic and Logical Relations}

\begin{axiom}
Every claim is either \emph{true} and \emph{false}: $\{T,F\}$
\begin{gather*}
    1+1=2 \to T \\
    1+1=0 \to F
\end{gather*}
\end{axiom}

\subsection{Unary Operations - Negation}

\begin{table}[h!]
\centering
{
\begin{tabular}{cc}
 \hline
 $P$ & $\neg P$ \\
 \hline
 T & F \\
 F & T \\
 \hline
\end{tabular}}
\caption{Negation}
\end{table}

\subsection{Binary Operations}

\begin{table}[ht!]
\centering
{
\begin{tabular}{cc|cccc}
 \hline
 $P$ & $Q$ & $P \lor Q$ & $P \land Q$ & $P \to Q$ & $P\leftrightarrow Q$ \\
 \hline
 T & T & T & T & T & T\\
 T & F & T & F & F & F\\
 F & T & T & F & T & F\\
 F & F & F & F & T & T\\
 \hline
\end{tabular}}
\caption{Binary Operations}
\end{table}

\begin{definition}[Complex Statements]
Combination of logical relations in one statement.
\begin{example}
    $$\neg P \to Q$$
\end{example}
\end{definition}

\subsubsection{Relation between $\cup, \cap$ and $\lor, \land$}
\begin{gather*}
    P : x \in A \quad Q: x \in B \\
    P \lor Q : x \in A \cup B \\
    P \land Q: x \in A \cap B
\end{gather*}

\begin{definition}[Logical Equivalency]
If two complex statements are made up of the same statements, then they are
\emph{logically equivalent}. ($\iff$)
\end{definition}

\begin{gather*}
    1<2 \quad x<2 \\
    \text{We'll symbolize the statement $P$ with the variable $x$ for $P(x)$} \\
    \text{Example: }P: (\forall \; x \in \mathbb{N} \quad x<2)
\end{gather*}

$$\forall \; x \in A \quad P(x)$$
is true if every $P(x)$ is true, and will be false if even one $P(x)$ is false.

$$\exists \; x \in A \quad P(x)$$
will be true if even one $P(x)$ is true, and will be false if all $P(x)$ is false.

Given $P(x), Q(x)$, the following logical statements can be constructed:
\begin{gather*}
    \forall \; x \in A \quad \forall \; y \in B \quad P(x) \land Q(y) \\
    \exists \; x \in A \quad \exists \; y \in B \quad P(x) \land Q(y) \\
    \exists \; x \in A \quad \forall \; y \in B \quad P(x) \land Q(y) \\
    \forall \; x \in A \quad \exists \; y \in B \quad P(x) \land Q(y) \\
\end{gather*}

\section{Series}

\begin{definition}
Given two sets $A,B$:
$$A \times B = \{(a,b) \mid a \in A, b \in B\}$$
\begin{note}
The order of $A, B$ does matter.
\end{note}
\end{definition}

If $A, B$ are \emph{finite}:

$$|A \times B| = |A| \cdot |B|$$

\begin{definition}[Plane]
$$\mathbb{R} \times \mathbb{R} = \mathbb{R}^2 = \{(x,y) \mid x, y \in \mathbb{R}\}$$
\end{definition}

If $A_1, \mathellipsis, A_n$ are sets:

$$\X_{i=1}^n = A_1 \times \mathellipsis \times A_n = \{(a_1, \mathellipsis,
a_n) \mid \forall \; i \in [n], a_i \in A_i\}$$

If $A_1, \ldots A_n$ are finite:

$$\left|\X_{i=1}^n A_i\right| = \prod_{i=1}^n |A_i|$$

\begin{definition}[Characteristics]
\begin{gather*}
    A \times \emptyset = \emptyset \\
    A \times A = A^2 \\
    \underbrace{A_1 \times \ldots \times A_n}_{n\text{ times}} = A^n
\end{gather*}
\end{definition}

\end{document}
