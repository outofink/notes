\documentclass[00_complete]{subfiles}

%\documentclass[12pt]{report}
\usepackage[utf8]{inputenc}
\usepackage{amsmath,amssymb,amsthm,gensymb,parskip,graphicx,footmisc,csquotes,enumerate,datetime2}
\usepackage[]{libertinus}
\usepackage[breaklinks]{hyperref}
\hypersetup{
  pdfauthor={Moshe Krumbein},
  colorlinks=true,
  linkcolor={black},
  filecolor={black},
  citecolor={black}, %blue
  urlcolor={black}, %blue
}
\usepackage[top=30mm,bottom=30mm,left=30mm,right=30mm]{geometry}
%\setlength{\emergencystretch}{2em} % prevent overfull lines
\providecommand{\tightlist}{%
\setlength{\itemsep}{0pt}\setlength{\parskip}{0pt}}

\renewcommand\qedsymbol{$\blacksquare$}

\theoremstyle{definition}
\newtheorem*{definition}{Definition}
\newtheorem*{theorem}{Theorem}
\newtheorem*{axiom}{Axiom}
\newtheorem*{lemma}{Lemma}

\theoremstyle{remark}
\newtheorem*{note}{Note}
\newtheorem*{symbols}{Symbol}
\newtheorem{example}{Example}[section]
\newtheorem*{claim}{Claim}
\newtheorem*{conclusion}{Conclusion}
\newtheorem*{reminder}{Reminder}

\usepackage{fancyhdr}
\usepackage[italicdiff]{physics}
\MakeOuterQuote{"}

\renewcommand{\chaptermark}[1]{\markboth{#1}{}}

\pagestyle{fancy}

\setlength{\headheight}{14.5pt}
\addtolength{\topmargin}{-2.5pt}

\fancyhf{}
\rhead{Moshe Krumbein}
\lhead{\chaptermark}
\cfoot{\thepage}
\fancyhead[R]{\chaptername~\thechapter}
\fancyhead[L]{\mbox{\leftmark}}

\usepackage[Rejne]{fncychap}
\usepackage{titling}

\makeatletter
\renewcommand{\@chapapp}{\vspace*{-100pt}\huge\thetitle}
\makeatother

\makeatletter
\newcommand{\subtitle}[1]{%
  {\center\vspace*{-60pt}%
  \linespread{1.1}\Large\scshape#1%
  \par\nobreak\vspace*{35pt}}
}
\makeatother

\newcommand{\Chapter}[2]{
    \def\n{#2}
    \setcounter{chapter}{\the\numexpr\n-1}
    \chapter{#1}
    \subtitle{\theauthor~- \thedate}
}

\DeclareMathOperator{\Ima}{Im}
\DeclareMathOperator{\Id}{Id}
\DeclareMathOperator{\cis}{cis}

\newcommand{\Mod}[1]{\ (\mathrm{mod}\ #1)}
\newcommand{\st}[0]{\;\mathrm{s.t.}\;}

\title{Discrete Mathematics}
\author{Moshe Krumbein}
\date{Fall 2021}

\begin{document}
\Chapter{Functions and Permutations}{2}

\section{Functions}

Given two sets $A, B$, function $f: A \mapsto B$, all $a \in A$ maps to $b \in B$.

$A$ is called the \emph{domain} and $B$ is called the \emph{range}.
$$f(a)=b$$

$a$ is \emph{a source} of $b$, and $b$ is \emph{the image} of $a$.

For $c \subset a$, the image of $c$ is:

$$f(C) = \left\{b \in B \mid a \in C : f(a) = b \right\} \subseteq B$$
$$=\{f(a) \mid a \in C\}$$

\begin{example}
\begin{enumerate}
Given $f: A \mapsto B$:

$$\Im f = f(A) \subseteq B$$

Given $D \subseteq B$, the source of $D$:

$$ f^{-1}(D) = \{a \in A \mid f(a) \in D\}$$

\item
$$A \subseteq U$$
$$\mathds{1}: U \mapsto \{0,1\} (\chi)$$

$$\forall \; a \in U, \mathds{1}_A(a) = \begin{cases}
    1 \quad a \in A \\
    0 \quad a \notin A
\end{cases}$$

\item
$$
\begin{gathered}
    A = \mathbb{Z}^k \quad B = \mathbb{Z} \\
    f((a_1, \ldots, a_n)) = a_3
\end{gathered}
$$

\item
$$
\begin{gathered}
    \text{Given: }\{(a_1, \ldots, a_n)\} \subseteq \mathbb{Z}^k \\
    f: \underbrace{[n]}_{\{1,2,\ldots,n\}} \mapsto \mathbb{Z} \\
    \forall \; i \in [n] : f(i)= a_i
\end{gathered}
$$

\end{enumerate}
\end{example}

\begin{claim}
Given: $f: A \mapsto B$, $D_1, D_2 \subseteq B$:
$$f^{-1}(D_1 \cap D_2) = f^{-1}(D_1) \cap f^{-1}(D_2)$$
$$a \in f^{-1}(D_1 \cap D_2) \iff f(a) \in D_1 \cap D_2$$
Given: $C_1, C_2 \subseteq A$:
$$f(C_1 \cap C_2) \neq f(C_1) \cap f(C_2)$$
\end{claim}

\subsubsection{Characteristics of Functions}
Given $f: A \mapsto B$:

$f$ is \emph{injective} (one-to-one) if:

$$\forall \; a_1,a_2 \in A : a_1 \neq a_2 \; \exists \; f(a_1) \neq f(a_2)$$
$$\forall \; a_1,a_2 \in A \quad f(a_1) = f(a_2) \implies a_1 = a_2$$

$f$ is \emph{surjective} (onto) if:

$$\Im f = B$$

If $f$ is injective: $|A| \leq |B|$.

If $f$ is surjective: $|A| \geq |B|$.

If $f$ is both injective and surjective, it is \emph{bijective} (invertible) and: $|A| = |B|$.
If we have two finite sets $A, B$, function $f: A \mapsto B$, and $C
\subseteq A, D \subseteq B$:
\begin{enumerate}
    \item $|f(c)| \leq |C|$
    \item $f$ is injective:
     \begin{enumerate}
        \item $|f(C)|= |C|$
        \item $|f^{-1}(D)| = |D|$
     \end{enumerate}
    \item $f$ is surjective:
    \begin{enumerate}
        \item $|f^{-1}(D)| \geq |D|$
    \end{enumerate}
    \item If $f$ is injective and $|A|=|B|$, then it is also surjective.
    \item If $f$ is surjective and $|A|=|B|$, then we've learned nothing new.
\end{enumerate}

\subsection{Composing Functions}

Given sets $A, B, C$:

$$f: A \mapsto B, \quad g: B \mapsto C$$
$$g \circ f: A \mapsto C: \forall \; a \in A:$$
$$(g \circ f)(a) = g(f(a))$$

\subsection{Identity function}

$$\Id_A: A \mapsto A$$
$$f \circ \mathrm{Id}_A: A \mapsto B$$
$$f \circ \mathrm{Id}_A: f$$
$$\mathrm{Id}_B \circ f: A \mapsto B$$
$$\mathrm{Id}_B \circ f: f$$

\subsection{Inverse function}

Given $f: A \mapsto B, g: B \mapsto C$:

$g$ is the inverse of $f$ if:

$$g \circ f = \mathrm{Id}_A$$
$$f \circ g = \mathrm{Id}_B$$

If the previous is true, then $g = f^{-1}$.

$n \in \mathbb{N}, [n] = \{1,2,\ldots, n\}$
\subsection{Examples}
\begin{example}
\begin{enumerate}
Given $f:A \mapsto B$, $C_1, C_2 \subseteq A$
\item If $C_1 \subseteq C_2$, then $f(C_1) \subseteq f(C_2)$.

Yes.

Suppose $a \in f(C_1) \implies b \in C_1$ such that $f(b)=a$. Since we also
know $b \in C_2$, therefore $f(b) \subseteq f(C_2) \implies f(C_1) \subseteq f(C_2)$.

\item If $f(C_1) \subseteq f(C_2)$, then $f(C_1) \subseteq f(C_2)$.

No.

We define $A=\{1,2\}, B=\{u\}$, $f(1)=f(2)=a$.

item If $f$ is injective, and $f(C_1) \subseteq f(C_2)$ ,then $C_1 \subseteq
C_2$.

Suppose $c \in C_1, c \notin C_2$.

$f(c) \notin f(C_2)$
\end{enumerate}
\end{example}

\begin{example}
    \begin{enumerate}
Let $f: A \mapsto B, g: B \mapsto C$. Prove that if $g \circ f$ is injective
then $f$ is also injective.

Suppose $a_1, a_2 \in A, a_1 \neq a_2$.

$$g \circ f(a_1) \neq g \circ f(a_2)$$
$$f(a_1) \neq f(a_2)$$
    \end{enumerate}
\end{example}

\begin{example}
\begin{enumerate}
Let $f: A \mapsto B, g: B \mapsto A, h: B \mapsto A$

 \item $g \circ f = \mathrm{Id}_A$ and $h \circ f = \mathrm{Id}_A$, then $g=h$.

False

\item $f \circ g = \mathrm{Id}_B$ and $h \circ f = \mathrm{Id}_A$, then $g=h$.

True
\end{enumerate}
\end{example}

\section{Permutations}

A \emph{permutation} is a (possible) rearrangement of objects. For example,
there are 6 permutations of the letters $a, b, c$:

$$abc, acb, bac, bca, cab, cba$$

Permutation on $[n]$ is $f: [n] \mapsto [n]$, for example $\mathrm{Id}_{[n]}$

$f: A \mapsto A$
$$f^2 = f \circ f: A \mapsto A$$
$$f^m = \underbrace{f \circ f \circ f \ldots}_{m \text{ times}}$$

If $f \circ g$ is bijective, it is a permutation.

$$\sigma^6 = \mathrm{Id}_{\sigma}$$
$$\sigma^{100} = \mathrm{Id}_{\sigma}$$


\begin{definition}[Permutation]
Permutation of $[n]$ is $f: [n] \mapsto [n]$ is surjective and injective.

We symbolize the \emph{collection} of permutations of $[n]$: $S_n$

If $f, g \in S_n$, then $g \circ f: [n] \mapsto [n]$.

If $g \circ f$ is \emph{one-to-one}, then we see it's also \emph{onto} $g \circ
f \in S_n$.

\end{definition}

\begin{claim}
If $f: A \to B$ is \emph{one-to-one} and $g: B \to C$ is also
\emph{one-to-one}, then $g \circ f A \to C$ is also \emph{one-to-one}.
\end{claim}

\begin{proof}
Suppose $a_1, a_2 \in A$ such that $(g \circ f)(a_1)=(g \circ f)(a_2)$ (which
means $a_1=a_2$).

Given $g(f(a_1))=g(f(a_2))$ is \emph{one-to-one} $\implies f(a_1)=f(a_2)$. $f$
is also \emph{one-to-one} $\implies a_1=a_2$.
\end{proof}

\begin{conclusion}
Conclusion: $g \circ f$ is \emph{one-to-one}.
\end{conclusion}

\begin{definition}
Let $f: [n] \to [n]$ be the permutation of $f$, the minimal number of
permutations $k$ such that $f_k=\Id_{[n]}$.

We see that $k$ is the \emph{lowest common multiple} of the lengths of the
sub-permutations.
\end{definition}
\end{document}
