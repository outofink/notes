\documentclass[00_complete]{subfiles}

%\documentclass[12pt]{report}
\usepackage[utf8]{inputenc}
\usepackage{amsmath,amssymb,amsthm,gensymb,parskip,graphicx,footmisc,csquotes,enumerate,datetime2}
\usepackage[]{libertinus}
\usepackage[breaklinks]{hyperref}
\hypersetup{
  pdfauthor={Moshe Krumbein},
  colorlinks=true,
  linkcolor={black},
  filecolor={black},
  citecolor={black}, %blue
  urlcolor={black}, %blue
}
\usepackage[top=30mm,bottom=30mm,left=30mm,right=30mm]{geometry}
%\setlength{\emergencystretch}{2em} % prevent overfull lines
\providecommand{\tightlist}{%
\setlength{\itemsep}{0pt}\setlength{\parskip}{0pt}}

\renewcommand\qedsymbol{$\blacksquare$}

\theoremstyle{definition}
\newtheorem*{definition}{Definition}
\newtheorem*{theorem}{Theorem}
\newtheorem*{axiom}{Axiom}
\newtheorem*{lemma}{Lemma}

\theoremstyle{remark}
\newtheorem*{note}{Note}
\newtheorem*{symbols}{Symbol}
\newtheorem{example}{Example}[section]
\newtheorem*{claim}{Claim}
\newtheorem*{conclusion}{Conclusion}
\newtheorem*{reminder}{Reminder}

\usepackage{fancyhdr}
\usepackage[italicdiff]{physics}
\MakeOuterQuote{"}

\renewcommand{\chaptermark}[1]{\markboth{#1}{}}

\pagestyle{fancy}

\setlength{\headheight}{14.5pt}
\addtolength{\topmargin}{-2.5pt}

\fancyhf{}
\rhead{Moshe Krumbein}
\lhead{\chaptermark}
\cfoot{\thepage}
\fancyhead[R]{\chaptername~\thechapter}
\fancyhead[L]{\mbox{\leftmark}}

\usepackage[Rejne]{fncychap}
\usepackage{titling}

\makeatletter
\renewcommand{\@chapapp}{\vspace*{-100pt}\huge\thetitle}
\makeatother

\makeatletter
\newcommand{\subtitle}[1]{%
  {\center\vspace*{-60pt}%
  \linespread{1.1}\Large\scshape#1%
  \par\nobreak\vspace*{35pt}}
}
\makeatother

\newcommand{\Chapter}[2]{
    \def\n{#2}
    \setcounter{chapter}{\the\numexpr\n-1}
    \chapter{#1}
    \subtitle{\theauthor~- \thedate}
}

\DeclareMathOperator{\Ima}{Im}
\DeclareMathOperator{\Id}{Id}
\DeclareMathOperator{\cis}{cis}

\newcommand{\Mod}[1]{\ (\mathrm{mod}\ #1)}
\newcommand{\st}[0]{\;\mathrm{s.t.}\;}

\title{Discrete Mathematics}
\author{Moshe Krumbein}
\date{Fall 2021}

\begin{document}
\Chapter{Cardinality}{12}

\section{Introduction}
\begin{reminder}
    For \emph{finite} sets, $A,B$:
    $$|A|=|B| \iff \exists f:A\to B \text{ that is \emph{bijective}}$$
\end{reminder}
\begin{definition}[Equinumerosity]
    $A, B$ have the same \emph{cardinality} (\emph{equinumerous}) if there
    exists an $f:A\to B$ that is \emph{bijective}. We symbolize this as $A\sim
    B$.
\end{definition}
\section{Characteristics}
\begin{enumerate}
    \item For all $A$: $A \sim A$
    $$\Id_A:A\to A \text{ is \emph{bijective}}$$

    \item For all $A,B$: if $A \sim B \to B \sim A$.
    $$f:A \to B \quad f^{-1}: B \to A$$
\begin{claim}
    Given sets $A,B,C$, $f:A\to B, g:B \to C$:
    \begin{itemize}
        \item If $f,g$ are \emph{injective}, then $g \circ f: A \to C$ is
            \emph{injective}
        \item If $f,g$ are \emph{surjective}, then $g \circ f:A \to C$ is
            \emph{surjective}
    \end{itemize}
\end{claim}
    \item Given sets $A,B,C$:
        $$A \sim B, B \sim C \implies A \sim C$$
\end{enumerate}
\section{Countable Sets}
\begin{definition}[Countable Sets]
    A set $A$ is called \emph{countable} if $\mathbb{N} \sim A$. In other
    words, there exists a function $f: \mathbb{N} \to A$ that is
    \emph{bijective}. In this case we symbolize $|A|=\aleph_0$.
\end{definition}
\begin{note}
    $A$ is \emph{countable} if there is a \emph{sequence} which contains all
    the elements of $A$ exactly once.
\end{note}
\subsection{Hilbert's Hotel}
    There are a countably infinite number of rooms such that each room is
    labeled with a natural number.

    Suppose all the rooms in Hilbert's hotel are full.

    A new guest arrives and wants a room. To accommodate for him, we can move
    each occupant to the room immediately to his right and put our new guest in
    first room which is unoccupied.

    Suppose a bus arrives containing an countably infinite number of guests. To
    accommodate for them, we can move each occupant to his room number times
    two, and place all the new guests in the now unoccupied odd-numbered rooms.

    Suppose $\aleph_0$ buses arrive with $\aleph_0$ guests each. To accommodate
    for all the new guests we can move each occupant in room $i$ to the room
    $2^i$. For the first bus, we can send the $i$-th person the $3^i$-th room.
    For each subsequent bus $j$ we send guest $i$ to room number $j$-th prime number to
    the $i$-th power.

    Finally, a single bus arrives with all the numbers between $0$ and $1$.
    Unfortunately, given that there are an uncountably infinite number of new
    guests, we will not able about to make enough room for them.
\subsection{Examples}

Let us explore using more mathematical notion:
\begin{example}
    \begin{gather*}
        A=\mathbb{N}\cup \{0\} \\
        f:\mathbb{N}\to A, \forall n \in \mathbb{N}: f(n)=n-1 \\
        A \sim \mathbb{N} \implies |A|=\aleph_0
    \end{gather*}
\end{example}
\begin{example}
    \begin{gather*}
        A=\mathbb{N}\times\{0\} \quad f:\mathbb{N} \to A\\
        f(n) = \begin{cases}
            (\frac{n}{2},1) & \text{$n$ is even} \\
            (\frac{n+1}{2},0) & \text{$n$ is odd}
        \end{cases}
    \end{gather*}
\end{example}
\begin{example}
    \begin{gather*}
        A=\mathbb{Z} \quad f:\mathbb{N}\to\mathbb{Z} \\
        f(n) = \begin{cases}
            \frac{n}{2} & \text{$n$ is even} \\
            -\frac{n-1}{2} & \text{$n$ is odd}
        \end{cases}
    \end{gather*}
\end{example}
\begin{claim}
    If $A$ is \emph{countably infinite} and $B \subseteq A$, then $B$
    is either \emph{finite} or \emph{countably infinite}.
\end{claim}
\begin{proof}
    Since we know that $A$ is \emph{countably infinite}, there exists
    $f:\mathbb{N}\to A$ that is \emph{bijective}. If $B$ is not \emph{finite},
    we will prove that $B$ is \emph{countably infinite}.

    We define $g:\mathbb{N}\to B$:
    $$g(n)=f(\min\{m\in\mathbb{N} \mid f(m) \in B, \forall i\in [n-1]: g(i)\neq
    f(m)\})$$
    In other words we can write $A$ as $f(1),f(2),f(3),\dots$ and $B$ as
    $f(2),f(4),f(5),\dots$, and therefore:
    \begin{gather*}
        g(1)=f(2) \\ g(2)=f(4) \\ g(3)=f(5) \\ \vdots \\
        g \text{ is \emph{bijective}} \implies \mathbb{N} \sim B \implies
        |B|=\aleph_0
    \end{gather*}
\begin{note}
    We can only define $g(n)$ after we define $g(1),g(2),\dots,g(n-1)$.
\end{note}
\end{proof}
\begin{claim}
    If $A$ is \emph{countably infinite} and $B$ is \emph{finite}, then $A\cup
    B$ is \emph{countably infinite}.
\end{claim}
\begin{note}
    In \emph{Hilbert's hotel} if $k$ new guests arrive we can move all currents
    guests $k$ rooms to the right, thus freeing up rooms $1$ through $k$.
\end{note}
\begin{proof}
    We will split into two cases:
    \begin{enumerate}
        \item Suppose $A\cap B = \emptyset$, and $|B|=k$.

        There exists $f:\mathbb{N}\to A, g:[k]\to B$ that are both
        \emph{bijective}.

        We define $h:\mathbb{N}\to A \cup B$:
        $$h(n)=\begin{cases}
            g(n)   & n \in [k] \\
            f(n-k) & n > k
        \end{cases}$$
        $h$ is \emph{surjective} and \emph{injective} (because $A\cup B =
        \emptyset$).
        \item In general:
        \begin{gather*}
            A \cup B = (A \setminus B) \cup B
        \end{gather*}
        $B$ is \emph{finite} and $A \setminus B \subseteq A$.

        $A \setminus B$ is not \emph{finite} because otherwise $A=(A\setminus
        B)\cup(A\cap B)$ would be a \emph{union} of \emph{finite} sets which is
        \emph{finite}, by contradiction (since we know $A$ is \emph{countably
        infinite}).

        Therefore, $A \setminus B$ is \emph{countably infinite} and we can
        finish with case $1$.
    \end{enumerate}
\end{proof}
\begin{claim}
    If $A,B$ are \emph{countably infinite}, then $A \cup B$ is \emph{countably
    infinite}.
\end{claim}
\begin{proof}
    Here too we will consider two different cases:
    \begin{enumerate}
        \item $A \cap B = \emptyset$:

        There exists $f:\mathbb{N}\to A,g:\mathbb{N}\to B$ which are
        \emph{bijective}. We can define $h: \mathbb{N} \to A \cup B$:
        \begin{gather*}
        h(n)=\begin{cases}
            g(\frac{n}{2}) & \text{$n$ is even} \\
            f(\frac{n+1}{2}) & \text{$n$ is odd} \\
        \end{cases} \implies A \cup B \sim \mathbb{N} \implies |A\cup B| =
        \aleph_0
        \end{gather*}
        \item In general:
        $$A \cup B = (A \setminus B) \cup B$$
        If $A \setminus B$ is \emph{finite} we can use the previous claim to
        finish the proof.

        If $A \setminus B$ is \emph{countably infinite}, since $(A \setminus B)
        \cap B = \emptyset$, we can use the first case to finish the proof.
    \end{enumerate}
\end{proof}
\begin{conclusion}
    There are two conclusions that can be drawn from the aforementioned claims:
    \begin{enumerate}
       \item If sets $A_1,\dots A_n$ are \emph{countably infinite}, then
           $\bigcup_{i=1}^nA_i$ is also \emph{countably infinite}.
       \item If $A_1,\dots A_n$ are sets such that for all $1 \leq i \leq n$:
           $A_i$ is either \emph{finite} or \emph{countably infinite}, then
           $\bigcup_{i=1}^kA_i$ is either \emph{finite} or \emph{countably
           infinite}.
    \end{enumerate}
\end{conclusion}
    What if we take a \emph{countably infinite} number of \emph{countably infinite} sets, such as the
    \emph{sequence} $A_1,A_2,\dots$ such that for all $i \in \mathbb{N}$: $A_i$
    is \emph{countably infinite}? Is $\bigcup_{i=1}^{\infty}A_i$
    \emph{countably infinite}?
\begin{claim}
    $\mathbb{N}\times\mathbb{N}$ is \emph{countably infinite}.
\end{claim}
\begin{proof}
    \begin{gather*}
        f:\mathbb{N}\times\mathbb{N}\to\mathbb{N} \\
        f(m,n)=(1+2+3+\dots+(m+n-2)) + n \\
        \implies \mathbb{N} \sim \mathbb{N}\times\mathbb{N} \implies
        |\mathbb{N}\times\mathbb{N}|=\aleph_0
    \end{gather*}
\end{proof}
\begin{claim}
    Given sets $A_1, A_2,\dots$ such that for all $i$: $A_i$ is either
    \emph{finite} or \emph{countably infinite}, then $\bigcup_{i=1}^\infty A_i$
    is \emph{finite} or \emph{countably infinite}.
\end{claim}
\begin{proof}
    We see that $\bigcup_{i=1}^\infty A_i$ is \emph{equinumerous} to a subset of
    $\mathbb{N}\times\mathbb{N}$.

    Given that all $A_n$ is either \emph{finite} or \emph{countably infinite},
    there exists a function $f_n: \mathbb{N} \to A_n$ that is \emph{bijective}
    or $f_n:[m_n]\to A_n$ that is \emph{bijective} in the case that
    $|A_n|=m_n$.

    Now, we will define for all $n$: $g_n=(f_n)^{-1}$ and $h:
    \bigcup_{i=1}^\infty A_i \to \mathbb{N}\times\mathbb{N}$.

    For all $x \in \bigcup_{i=1}^\infty A_i$, there exists $n \in \mathbb{N}$
    such that $x \in A_n$. We therefore define $h(x)=(n,g_n(x))$. Now all we
    have to do is prove that $h$ is \emph{injective} and \emph{bijective}.

    $h$ is \emph{injective}: If $h(x)=h(y)$, then we can symbolize:
    $$h(x)=(n_1,g_{n_1}(x)) \qquad h(y)=(n_2,g_{n_2}(y))$$
    Therefore, we see that $n_1=n_2$ and $g_{n_1}(x)=g_{n_1}(y)$, but since we
    know that $g_{n_1}$ is \emph{injective} $\implies x=y$.

    We notice that $h:\bigcup_{i=1}^\infty A_1 \to \Im(h)$ is \emph{bijective}
    and therefore $\bigcup A_i \sim \Im(h)$. However, we know that
    $\Im(h)\subseteq \mathbb{N}\times\mathbb{N}$, and that
    $\mathbb{N}\times\mathbb{N}$ is \emph{countably infinite}, and therefore
    $\Im(h)$ is either \emph{finite} or \emph{countably infinite} $\implies
    \bigcup_{i=1}^\infty A_i$ is \emph{finite} or \emph{countably infinite}.
\end{proof}
\begin{note}
    If all $A_i$ are \emph{countably infinite}, then $\bigcup_{i=1}^\infty A_i$
    is also \emph{countably infinite}.
\end{note}
Alternatively, if one of the sets$A_j \in \{A_1,A_2,\dots\}$ is \emph{countably infinite}:
$$A_j \subseteq \bigcup_{i=1}^\infty A_i \implies \bigcup_{i=1}^\infty A_i \sim
\aleph_0$$
\subsection{Cardinality of the Rational Numbers \texorpdfstring{($\mathbb{Q}$)}{}}
\begin{claim}
    $\mathbb{Q}$ is \emph{countably infinite}.
\end{claim}
\begin{proof}
    \begin{gather*}
    \mathbb{Q}=\left\{\frac{m}{n} \;\middle|\; m \in \mathbb{Z}, n \in
    \mathbb{N}\right\} \\
    Q_n=\left\{\frac{m}{n} \;\middle|\; m \in \mathbb{Z}\right\}
    \end{gather*}
    For all $n$, $Q_n$ is \emph{countably infinite} since there exists $f_n:
    \mathbb{Z} \to Q_n$ which is \emph{bijective}:
    $$\forall m \in \mathbb{Z}: f_n(m)=\frac{m}{n}$$
    $$\mathbb{Q}=\bigcup_{n \in \mathbb{N}}Q_n \implies \mathbb{Q} \sim
    \aleph_0$$
\end{proof}
\section{Uncountable Sets}
Is the set of all the numbers between $0$ and $1$ \emph{countable}?
\begin{definition}[Weak Inequality]
    $|A|\leq|B|$ if there exists:
    $$f:A \to B$$
    that is \emph{injective}.
\end{definition}
\begin{definition}[Strong Inequality]
    $|A|<|B|$ if $|A|\leq|B|$ and there does not exist:
    $$g: A \to B$$
    that is \emph{surjective}.
\end{definition}
\begin{claim}
    $|\mathbb{N}|<|(0,1)|$
\end{claim}
\begin{proof}
    This proof is famously known as \emph{Cantor's diagonal argument}.

    We define $f: \mathbb{N} \to (0,1)$:
    $$\forall n \in N: f(n)=\frac{1}{n}\implies f \text{ is
    \emph{injective}}\implies |\mathbb{N}|\leq|(0,1)|$$
    Now, we will prove by contradiction, supposing there does exist $g:
    \mathbb{N} \to (0,1)$ that is \emph{injective}.

    We suppose that every number $(0,1)$ has a unique expression as a
    decimal fraction.

    We define:
    \begin{gather*}
        g(1)=0.a_{11}a_{12}a_{13}a_{14}\dots \\
        g(2)=0.a_{21}a_{22}a_{23}a_{24}\dots \\
        g(3)=0.a_{31}a_{32}a_{33}a_{34}\dots \\
        \vdots
    \end{gather*}
    We want to show that there exists $b=0.b_1b_2b_3b_4\dots \in (0,1)$, such
    that $b \notin \Im (g)$. We define for all $i \in \mathbb{N}$:
    $$b_i=\begin{cases}
        4 & a_{ii}=3 \\
        3 & a_{ii}\neq 3
    \end{cases}$$
    where for all $i \in \mathbb{N}$: $b_i \neq a_{ii}$, and therefore, $b\neq
    g(i)$. In other words, $b$ does not have a source from the domain of $g$,
    which is contrary to the fact that $g$ is \emph{injective}.
\end{proof}
We need a new \emph{equivalence class} for $(0,1)$.
\begin{symbols}
    $$|(0,1)|=\aleph \text{ (or  $\aleph_1$)}$$
\end{symbols}
\begin{itemize} \tightlist
    \item For all $a<b$: $(a,b)$:
    $$f(x):(0,1)\to (a,b) \qquad f(x)=(b-a)x+a$$
    \item $\mathbb{R}$:
    $$f(x):(0,1)\to \mathbb{R} \qquad f(x)=\frac{1-2x}{x(1-x)}$$
    \item $(0, \infty)$:
    $$f(x):\mathbb{R}\to (0,\infty) \qquad f(x)=e^x$$
    \item $|(0,\infty)|=|(1,\infty)|$:
    $$f(x):(0,\infty)\to (1,\infty) \qquad f(x)=x+1$$
    \item $|(1,\infty)|=|(0,1)|$:
    $$f(x):(1,\infty)\to (0,1) \qquad f(x)=\frac{1}{x}$$
\end{itemize}
$$\implies |0,1| = |(1,\infty)|=|\mathbb{R}| = \aleph_1$$
\begin{claim}[Continuum Hypnosis]
    There is no set $A$, such that:
    $$\aleph_0 =|\mathbb{N}|<|A|<|\mathbb{R}|=\aleph_1$$
\end{claim}
\begin{theorem}[Cantor's Theorem]
    For any set $A$: $|A|<\underbrace{|P(A)|}_{2^A}$
\end{theorem}
\begin{proof}
    We define $f:A \to P(A)$: $f(a)=\{a\}$. We can say that there is no
    function $g: A \to P(A)$ that is \emph{injective}. We suppose by
    contradiction that there exists such a function $g$. We define:
    $$B=\{a \in A \mid a \notin g(a)\} \subseteq A$$
    There exists $b \in A$: $g(b)=B$, since $g$ is \emph{injective}:
    \begin{gather*}
        b \in B \implies b \in g(b) \implies b \notin B \\
        b \notin B \implies b \in g(b) \implies b \in B
    \end{gather*}
    Which is a contradiction.
\end{proof}
\end{document}
