\documentclass[00_complete]{subfiles}

%\documentclass[12pt]{report}
\usepackage[utf8]{inputenc}
\usepackage{amsmath,amssymb,amsthm,gensymb,parskip,graphicx,footmisc,csquotes,enumerate,datetime2}
\usepackage[]{libertinus}
\usepackage[breaklinks]{hyperref}
\hypersetup{
  pdfauthor={Moshe Krumbein},
  colorlinks=true,
  linkcolor={black},
  filecolor={black},
  citecolor={black}, %blue
  urlcolor={black}, %blue
}
\usepackage[top=30mm,bottom=30mm,left=30mm,right=30mm]{geometry}
%\setlength{\emergencystretch}{2em} % prevent overfull lines
\providecommand{\tightlist}{%
\setlength{\itemsep}{0pt}\setlength{\parskip}{0pt}}

\renewcommand\qedsymbol{$\blacksquare$}

\theoremstyle{definition}
\newtheorem*{definition}{Definition}
\newtheorem*{theorem}{Theorem}
\newtheorem*{axiom}{Axiom}
\newtheorem*{lemma}{Lemma}

\theoremstyle{remark}
\newtheorem*{note}{Note}
\newtheorem*{symbols}{Symbol}
\newtheorem{example}{Example}[section]
\newtheorem*{claim}{Claim}
\newtheorem*{conclusion}{Conclusion}
\newtheorem*{reminder}{Reminder}

\usepackage{fancyhdr}
\usepackage[italicdiff]{physics}
\MakeOuterQuote{"}

\renewcommand{\chaptermark}[1]{\markboth{#1}{}}

\pagestyle{fancy}

\setlength{\headheight}{14.5pt}
\addtolength{\topmargin}{-2.5pt}

\fancyhf{}
\rhead{Moshe Krumbein}
\lhead{\chaptermark}
\cfoot{\thepage}
\fancyhead[R]{\chaptername~\thechapter}
\fancyhead[L]{\mbox{\leftmark}}

\usepackage[Rejne]{fncychap}
\usepackage{titling}

\makeatletter
\renewcommand{\@chapapp}{\vspace*{-100pt}\huge\thetitle}
\makeatother

\makeatletter
\newcommand{\subtitle}[1]{%
  {\center\vspace*{-60pt}%
  \linespread{1.1}\Large\scshape#1%
  \par\nobreak\vspace*{35pt}}
}
\makeatother

\newcommand{\Chapter}[2]{
    \def\n{#2}
    \setcounter{chapter}{\the\numexpr\n-1}
    \chapter{#1}
    \subtitle{\theauthor~- \thedate}
}

\DeclareMathOperator{\Ima}{Im}
\DeclareMathOperator{\Id}{Id}
\DeclareMathOperator{\cis}{cis}

\newcommand{\Mod}[1]{\ (\mathrm{mod}\ #1)}
\newcommand{\st}[0]{\;\mathrm{s.t.}\;}

\title{Discrete Mathematics}
\author{Moshe Krumbein}
\date{Fall 2021}

\begin{document}
\Chapter{Cardinality}{12}

\section{Introduction}
\begin{reminder}
    For \emph{finite} sets, $A,B$:
    $$|A|=|B| \iff \exists f:A\to B \text{ that is \emph{bijective}}$$
\end{reminder}
\begin{definition}[Equinumerosity]
    $A, B$ have the same \emph{cardinality} (\emph{equinumerous}) if there
    exists an $f:A\to B$ that is \emph{bijective}. We symbolize this as $A\sim
    B$.
\end{definition}
\section{Characteristics}
\begin{enumerate}
    \item For all $A$: $A \sim A$
    $$\Id_A:A\to A \text{ is \emph{bijective}}$$

    \item For all $A,B$, if $A \sim B \to B \sim A$.
    $$f:A \to B \quad f^{-1}: B \to A$$
\begin{claim}
    Given sets $A,B,C$, $f:A\to B, g:B \to C$:
    \begin{itemize}
        \item If $f,g$ are \emph{injective}, then $g \circ f: A \to C$ is
            \emph{injective}
        \item If $f,g$ are \emph{surjective}, then $g \circ f:A \to C$ is
            \emph{surjective}
    \end{itemize}
\end{claim}
    \item Given sets $A,B,C$:
        $$A \sim B, B \sim C \implies A \sim C$$
\end{enumerate}
\section{Countable Sets}
\begin{definition}[Countable Sets]
    A set $A$ is called \emph{countable} if $\mathbb{N} \sim A$. In other
    words, there exists a function $f: \mathbb{N} \to A$ that is
    \emph{bijective}. In this case we symbolize $|A|=\aleph_0$.
\end{definition}
\begin{note}
    $A$ is \emph{countable} if there is a \emph{sequence} which contains all
    the elements of $A$ exactly once.
\end{note}
\subsection{Hilbert's Hotel}
    There are a countably infinite number of rooms such that each room is
    labeled with a natural number.

    Suppose all the rooms in Hilbert's hotel are full.

    A new guest arrives and wants a room. To accommodate for him, we can move
    each occupant to the room immediately to his right and put our new guest in
    first room which is unoccupied.

    Suppose a bus arrives containing an countably infinite number of guests. To
    accommodate for them, we can move each occupant to his room number times
    two, and place all the new guests in the now unoccupied odd-numbered rooms.

    Suppose $\aleph_0$ buses arrive with $\aleph_0$ guests each. To accommodate
    for all the new guests we can move each occupant in room $i$ to the room
    $2^i$. For the first bus, we can send the $i$-th person the $3^i$-th room.
    For each subsequent bus $j$ we send guest $i$ to room number $j$-th prime number to
    the $i$-th power.

    Finally, a single bus arrives with all the numbers between $0$ and $1$.
    Unfortunately, given that there are an uncountably infinite number of new
    guest, we will not able about to make enough room for them.
\end{document}
