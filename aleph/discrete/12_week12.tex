\documentclass[00_complete]{subfiles}

%\documentclass[12pt]{report}
\usepackage[utf8]{inputenc}
\usepackage{amsmath,amssymb,amsthm,gensymb,parskip,graphicx,footmisc,csquotes,enumerate,datetime2}
\usepackage[]{libertinus}
\usepackage[breaklinks]{hyperref}
\hypersetup{
  pdfauthor={Moshe Krumbein},
  colorlinks=true,
  linkcolor={black},
  filecolor={black},
  citecolor={black}, %blue
  urlcolor={black}, %blue
}
\usepackage[top=30mm,bottom=30mm,left=30mm,right=30mm]{geometry}
%\setlength{\emergencystretch}{2em} % prevent overfull lines
\providecommand{\tightlist}{%
\setlength{\itemsep}{0pt}\setlength{\parskip}{0pt}}

\renewcommand\qedsymbol{$\blacksquare$}

\theoremstyle{definition}
\newtheorem*{definition}{Definition}
\newtheorem*{theorem}{Theorem}
\newtheorem*{axiom}{Axiom}
\newtheorem*{lemma}{Lemma}

\theoremstyle{remark}
\newtheorem*{note}{Note}
\newtheorem*{symbols}{Symbol}
\newtheorem{example}{Example}[section]
\newtheorem*{claim}{Claim}
\newtheorem*{conclusion}{Conclusion}
\newtheorem*{reminder}{Reminder}

\usepackage{fancyhdr}
\usepackage[italicdiff]{physics}
\MakeOuterQuote{"}

\renewcommand{\chaptermark}[1]{\markboth{#1}{}}

\pagestyle{fancy}

\setlength{\headheight}{14.5pt}
\addtolength{\topmargin}{-2.5pt}

\fancyhf{}
\rhead{Moshe Krumbein}
\lhead{\chaptermark}
\cfoot{\thepage}
\fancyhead[R]{\chaptername~\thechapter}
\fancyhead[L]{\mbox{\leftmark}}

\usepackage[Rejne]{fncychap}
\usepackage{titling}

\makeatletter
\renewcommand{\@chapapp}{\vspace*{-100pt}\huge\thetitle}
\makeatother

\makeatletter
\newcommand{\subtitle}[1]{%
  {\center\vspace*{-60pt}%
  \linespread{1.1}\Large\scshape#1%
  \par\nobreak\vspace*{35pt}}
}
\makeatother

\newcommand{\Chapter}[2]{
    \def\n{#2}
    \setcounter{chapter}{\the\numexpr\n-1}
    \chapter{#1}
    \subtitle{\theauthor~- \thedate}
}

\DeclareMathOperator{\Ima}{Im}
\DeclareMathOperator{\Id}{Id}
\DeclareMathOperator{\cis}{cis}

\newcommand{\Mod}[1]{\ (\mathrm{mod}\ #1)}
\newcommand{\st}[0]{\;\mathrm{s.t.}\;}

\title{Discrete Mathematics}
\author{Moshe Krumbein}
\date{Fall 2021}

\begin{document}
\Chapter{Cardinality}{12}

\section{Introduction}
\begin{reminder}
    For \emph{finite} sets, $A,B$:
    $$|A|=|B| \iff \exists f:A\to B \text{ that is \emph{bijective}}$$
\end{reminder}
\begin{definition}[Equinumerosity]
    $A, B$ have the same \emph{cardinality} (\emph{equinumerous}) if there
    exists an $f:A\to B$ that is \emph{bijective}. We symbolize this as $A\sim
    B$.
\end{definition}
\section{Characteristics}
\begin{enumerate}
    \item For all $A$: $A \sim A$
    $$\Id_A:A\to A \text{ is \emph{bijective}}$$

    \item For all $A,B$, if $A \sim B \to B \sim A$.
    $$f:A \to B \quad f^{-1}: B \to A$$
\begin{claim}
    Given sets $A,B,C$, $f:A\to B, g:B \to C$:
    \begin{itemize}
        \item If $f,g$ are \emph{injective}, then $g \circ f: A \to C$ is
            \emph{injective}
        \item If $f,g$ are \emph{surjective}, then $g \circ f:A \to C$ is
            \emph{surjective}
    \end{itemize}
\end{claim}
    \item Given sets $A,B,C$:
        $$A \sim B, B \sim C \implies A \sim C$$
\end{enumerate}
\section{Countable Sets}
\begin{definition}[Countable Sets]
    A set $A$ is called \emph{countable} if $\mathbb{N} \sim A$. In other
    words, there exists a function $f: \mathbb{N} \to A$ that is
    \emph{bijective}. In this case we symbolize $|A|=\aleph_0$.
\end{definition}
\begin{note}
    $A$ is \emph{countable} if there is a \emph{sequence} which contains all
    the elements of $A$ exactly once.
\end{note}
\subsection{Hilbert's Hotel}
    There are a countably infinite number of rooms such that each room is
    labeled with a natural number.

    Suppose all the rooms in Hilbert's hotel are full.

    A new guest arrives and wants a room. To accommodate for him, we can move
    each occupant to the room immediately to his right and put our new guest in
    first room which is unoccupied.

    Suppose a bus arrives containing an countably infinite number of guests. To
    accommodate for them, we can move each occupant to his room number times
    two, and place all the new guests in the now unoccupied odd-numbered rooms.

    Suppose $\aleph_0$ buses arrive with $\aleph_0$ guests each. To accommodate
    for all the new guests we can move each occupant in room $i$ to the room
    $2^i$. For the first bus, we can send the $i$-th person the $3^i$-th room.
    For each subsequent bus $j$ we send guest $i$ to room number $j$-th prime number to
    the $i$-th power.

    Finally, a single bus arrives with all the numbers between $0$ and $1$.
    Unfortunately, given that there are an uncountably infinite number of new
    guests, we will not able about to make enough room for them.
\subsection{Examples}

Let us explore using more mathematical notion:
\begin{example}
    \begin{gather*}
    A=\mathbb{N}\cup \{0\} \\
    f:\mathbb{N}\to A, \forall n \in \mathbb{N}: f(n)=n-1 \\
    A \sim \mathbb{N} \implies |A|=\aleph_0
    \end{gather*}
\end{example}
\begin{example}
    \begin{gather*}
        A=\mathbb{N}\times\{0\} \quad f:\mathbb{N} \to A\\
        f(n) = \begin{cases}
            (\frac{n}{2},1) & \text{$n$ is even} \\
            (\frac{n+1}{2},0) & \text{$n$ is odd}
        \end{cases}
    \end{gather*}
\end{example}
\begin{example}
    \begin{gather*}
        A=\mathbb{Z} \quad f:\mathbb{N}\to\mathbb{Z} \\
        f(n) = \begin{cases}
            \frac{n}{2} & \text{$n$ is even} \\
            -\frac{n-1}{2} & \text{$n$ is odd}
        \end{cases}
    \end{gather*}
\end{example}
\begin{claim}
    IF $A$ is \emph{countably infinite} and $B \subseteq A$, then $B$
    is either \emph{finite} or \emph{countably infinite}.
\end{claim}
\begin{proof}
    Since we know that $A$ is \emph{countably infinite}, there exists
    $f:\mathbb{N}\to A$ that is \emph{bijective}. If $B$ is not \emph{finite},
    we will prove that $B$ is \emph{countably infinite}.

    We define $g:\mathbb{N}\to B$:
    $$g(n)=f(\min\{m\in\mathbb{N} \mid f(m) \in B, \forall i\in [n-1]: g(i)\neq
    f(m)\})$$
    In other words we can write $A$ as $f(1),f(2),f(3),\dots$ and $B$ as
    $f(2),f(4),f(5),\dots$, and therefore:
    \begin{gather*}
        g(1)=f(2) \\ g(2)=f(4) \\ g(3)=f(5) \\ \vdots \\
        g \text{ is \emph{bijective}} \implies \mathbb{N} \sim B \implies
        |B|=\aleph_0
    \end{gather*}
\begin{note}
    We can only define $g(n)$ after we define $g(1),g(2),\dots,g(n-1)$.
\end{note}
\end{proof}
\begin{claim}
    If $A$ is \emph{countably infinite} and $B$ is \emph{finite}, then $A\cup
    B$ is \emph{countably infinite}.
\end{claim}
\begin{note}
    In \emph{Hilbert's hotel} if $k$ new guests arrive we can move all currents
    guests $k$ rooms to the right, thus freeing up rooms $1$ through $k$.
\end{note}
\begin{proof}
    We will split into two cases:
    \begin{enumerate}
        \item Suppose $A\cap B = \emptyset$, and $|B|=k$.

        There exists $f:\mathbb{N}\to A, g:[k]\to B$ that are both
        \emph{bijective}.

        We define $h:\mathbb{N}\to A \cup B$:
        $$h(n)=\begin{cases}
            g(n)   & n \in [k] \\
            f(n-k) & n > k
        \end{cases}$$
        $h$ is \emph{surjective} and \emph{injective} (because $A\cup B =
        \emptyset$).
        \item In general:

        \begin{gather*}
            A \cup B = (A \setminus B) \cup B
        \end{gather*}
        $B$ is \emph{finite} and $A \setminus B \subseteq A$.

        $A \setminus B$ is not \emph{finite} because otherwise $A=(A\setminus
        B)\cup(A\cap B)$ would be a \emph{union} of \emph{finite} sets which is
        \emph{finite}, by contraction (since we know $A$ is \emph{countably
        infinite}).

        Therefore, $A \setminus B$ is \emph{countably infinite} and we can
        finish with case $1$.
    \end{enumerate}
\end{proof}
\begin{claim}
    If $A,B$ are \emph{countably infinite}, then $A \cup B$ is \emph{countably
    infinite}.
\end{claim}
\begin{proof}
    Here too we will consider two different cases:
    \begin{enumerate}
        \item $A \cap B = \emptyset$:

        There exists $f:\mathbb{N}\to A,g:\mathbb{N}\to B$ which are
        \emph{bijective}. We can define $h: \mathbb{N} \to A \cup B$:
        \begin{gather*}
        h(n)=\begin{cases}
            g(\frac{n}{2}) & \text{$n$ is even} \\
            f(\frac{n+1}{2}) & \text{$n$ is odd} \\
        \end{cases} \implies A \cup B \sim \mathbb{N} \implies |A\cup B| =
        \aleph_0
        \end{gather*}
        \item In general:
        $$A \cup B = (A \setminus B) \cup B$$
        If $A \setminus B$ is \emph{finite} we can use the previous claim to
        finish the proof.

        If $A \setminus B$ is \emph{countably infinite}, since $(A \setminus B)
        \cap B = \emptyset$, we can use the first case to finish the proof.
    \end{enumerate}
\end{proof}
\begin{conclusion}
    There are two conclusions that can be drawn from the aforementioned claims:
    \begin{enumerate}
       \item If sets $A_1,\dots A_n$ are \emph{countably infinite}, then
           $\bigcup_{i=1}^nA_i$ is also \emph{countably infinite}.
       \item If $A_1,\dots A_n$ are sets such that for all $1 \leq i \leq n$:
           $A_i$ is either \emph{finite} or \emph{countably infinite}, then
           $\bigcup_{i=1}^kA_i$ is either \emph{finite} or \emph{countably
           infinite}.
    \end{enumerate}
\end{conclusion}
    What we take an infinite number \emph{countably infinite} sets, such as the
    \emph{sequence} $A_1,A_2,\dots$ such that for all $i \in \mathbb{N}$: $A_i$
    is \emph{countably infinite}? Is $\bigcup_{i=1}^{\infty}A_i$
    \emph{countably infinite}?
\begin{claim}
    $\mathbb{N}\times\mathbb{N}$ is \emph{countably infinite}.
\end{claim}
\begin{proof}
    \begin{gather*}
        f:\mathbb{N}\times\mathbb{N}\to\mathbb{N} \\
        f(m,n)=(1+2+3+\dots+(m+n-2)) + n \\
        \implies \mathbb{N} \sim \mathbb{N}\times\mathbb{N} \implies
        |\mathbb{N}\times\mathbb{N}|=\aleph_0
    \end{gather*}
\end{proof}
\end{document}
