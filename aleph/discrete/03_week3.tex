\documentclass[00_complete]{subfiles}

%\documentclass[12pt]{report}
\usepackage[utf8]{inputenc}
\usepackage{amsmath,amssymb,amsthm,gensymb,parskip,graphicx,footmisc,csquotes,enumerate,datetime2}
\usepackage[]{libertinus}
\usepackage[breaklinks]{hyperref}
\hypersetup{
  pdfauthor={Moshe Krumbein},
  colorlinks=true,
  linkcolor={black},
  filecolor={black},
  citecolor={black}, %blue
  urlcolor={black}, %blue
}
\usepackage[top=30mm,bottom=30mm,left=30mm,right=30mm]{geometry}
%\setlength{\emergencystretch}{2em} % prevent overfull lines
\providecommand{\tightlist}{%
\setlength{\itemsep}{0pt}\setlength{\parskip}{0pt}}

\renewcommand\qedsymbol{$\blacksquare$}

\theoremstyle{definition}
\newtheorem*{definition}{Definition}
\newtheorem*{theorem}{Theorem}
\newtheorem*{axiom}{Axiom}
\newtheorem*{lemma}{Lemma}

\theoremstyle{remark}
\newtheorem*{note}{Note}
\newtheorem*{symbols}{Symbol}
\newtheorem{example}{Example}[section]
\newtheorem*{claim}{Claim}
\newtheorem*{conclusion}{Conclusion}
\newtheorem*{reminder}{Reminder}

\usepackage{fancyhdr}
\usepackage[italicdiff]{physics}
\MakeOuterQuote{"}

\renewcommand{\chaptermark}[1]{\markboth{#1}{}}

\pagestyle{fancy}

\setlength{\headheight}{14.5pt}
\addtolength{\topmargin}{-2.5pt}

\fancyhf{}
\rhead{Moshe Krumbein}
\lhead{\chaptermark}
\cfoot{\thepage}
\fancyhead[R]{\chaptername~\thechapter}
\fancyhead[L]{\mbox{\leftmark}}

\usepackage[Rejne]{fncychap}
\usepackage{titling}

\makeatletter
\renewcommand{\@chapapp}{\vspace*{-100pt}\huge\thetitle}
\makeatother

\makeatletter
\newcommand{\subtitle}[1]{%
  {\center\vspace*{-60pt}%
  \linespread{1.1}\Large\scshape#1%
  \par\nobreak\vspace*{35pt}}
}
\makeatother

\newcommand{\Chapter}[2]{
    \def\n{#2}
    \setcounter{chapter}{\the\numexpr\n-1}
    \chapter{#1}
    \subtitle{\theauthor~- \thedate}
}

\DeclareMathOperator{\Ima}{Im}
\DeclareMathOperator{\Id}{Id}
\DeclareMathOperator{\cis}{cis}

\newcommand{\Mod}[1]{\ (\mathrm{mod}\ #1)}
\newcommand{\st}[0]{\;\mathrm{s.t.}\;}

\title{Discrete Mathematics}
\author{Moshe Krumbein}
\date{Fall 2021}

\begin{document}
\Chapter{Binary Relations and Equivalence Relations}{3}

\section{Binary Relations}
\begin{definition}[Binary Relation]
Given two sets $A, B$, all subsets $R \subseteq A \times B$ is called "binary
relations" from $A$ to $B$. If $(a,b) \in R$ we symbolize relations as $aRb$.
\end{definition}

\begin{example}
$$A=B=\mathbb{R} \quad R=\{(x,x^2) \mid x \in A \}$$
\end{example}

\begin{definition}
If $A=B$, then $R \subseteq A \times A$ is called a \emph{relation on $A$}.
\end{definition}

\begin{example}
$A=\{2,3,4,5,8,12\}, R=\{(a,b) \mid a, b \in A, a | b \}$
$$R=\{(2,4), (2,8),(2,12),(2,2),(3,3),(3,12),\ldots \}$$
\end{example}

\begin{definition}
\emph{Empty relation}: $R=\emptyset$
\end{definition}
\begin{definition}
\emph{Full (universal) relation}: $R = A \times A$
\end{definition}
\begin{definition}[Power Set]
Given set $A$, we'll define the \emph{power set} as being all the subsets of $A$:
$$2^A = P(A) = \{B \mid B \subseteq A \}$$
$$|P(A)| = 2^{|A|}$$
\end{definition}
\begin{example}
$B = P(\{1,2,3\})$:
$$R=\{(A_1,A_2) \mid A_1, A_2 \in B, A_1 \subseteq A_2 \}$$
\end{example}

\subsection{Characteristics of Binary Relations}
\begin{enumerate}
    \item $R$ is \emph{reflexive} if for all $a \in A: (a,a) \in R$
    \item $R$ is \emph{irreflexive} if for all $a \in A: (a,a) \notin R$
    \item $R$ is \emph{symmetric} if for all $a,b \in A: (a,b) \in R \implies (b,a) \in R$
    \item $R$ is \emph{antisymmetric} if for all $a,b \in A: (a,b), (b,a) \in R \implies
   a=b$
    \item $R$ is \emph{transitive} if for all $a,b,c \in A: (a,b), (b,c) \in R \implies
   (a,c) \in R$
\end{enumerate}
$A \neq \emptyset$ and not limited in size can have a relation that is
reflexive, symmetric and antisymmetric: $R =\{(a,a) \mid a \in A \}$.
\begin{example}
$$
\begin{gathered}
    A \in \mathbb{Z} \\
    a|x-y \iff xRy \\
    (x,y), (y,z) \in R \implies (x,z) \in R \\
    (x,y) \in R \implies \exists t_1 \in \mathbb{Z}, x-y=2t_1 \\
    (y,z) \in R \implies \exists t_2 \in \mathbb{Z}, y-z=2t_2 \\
    x-z = x-y + y-z = 2t_1 + 2t_2 = 2(t_1 + t_2) \implies (x,z) \in R
\end{gathered}
$$
\end{example}
\section{Equivalence relations}
\begin{definition}
$R$ is a \emph{equivalence relation} on $A$ if $R$ is \emph{reflexive}, \emph{symmetric}, and
\emph{transitive}.
\end{definition}

\begin{example}
$$A = \{1,2,3,4,5\}$$
\begin{enumerate}
    \item $R = \{(a,a) \mid a \in A \}$ - Identity relation
    \item Everything is related to itself, and $1, 2$ relate to each other
    \item Everything is related to itself, and $1, 2, 3$ relate to each other
\end{enumerate}

We can see that there are closed groups (\emph{closures}) within each of the relations: 1 has
five closed groups, 2 has four, and 3 has three.
\end{example}

\begin{symbols}
If $R$ is a equivalence relation on $A$ we sometimes symbolize $(x,y) \in R$ or
$x \sim y$
\end{symbols}

\begin{definition}
If $R$ is a equivalence relation on $A$ then for all $a \in A$ we will define
\emph{equivalence closure}:

$$[a]_R = \{b \in A \mid (a,b) \in R$$

Given group $A$ the closure of $A$ is the collection $\{A_i\}_{i \in I}$:
\begin{enumerate}
\item $\forall i \in I, A_i \neq \emptyset$
\item $\forall i,j \in I, A_i \neq A_j \implies A_i \cap A_j = \emptyset$
\item $\displaystyle \bigcup_{i \in I}A_i=A$
\end{enumerate}
\end{definition}

\begin{example}
$$
\begin{gathered}
    A = \mathbb{R} = \{(-\infty, -1), [-1,1], (1,\infty)\} \\
    B = \mathbb{N} = \{\{1,3,5,7,\ldots\},\{2,4,6,8,\ldots\}\}
\end{gathered}
$$
\end{example}

\begin{claim}
If $R$ is a equivalence relation on $A$ then the collection of all the
equivalence closures:

$$\{[a]_R \mid a \in A \}$$

is a \emph{partition} of $A$.
\end{claim}

\begin{symbols}
\emph{Equivalence closures}: $A/R$
$$A/R  = \{[a]_R \mid a \in A\}$$

$D$ is the \emph{representative set} if:
$$\forall a \in A: |D \cap [a]_R| = 1$$
\end{symbols}

\begin{example}[Identity relation]
$$
\begin{gathered}
    \forall a \in A: [a]_R = \{a\} \\
    A/R=\{\{a\}\}_{a \in A} \\
    D=A
\end{gathered}
$$
\end{example}
\begin{example}[Universal Relation]
$$
\begin{gathered}
    R = A \times A \\
    A/R = \{A\} \\
    \forall a \in A: D = \{a\}
\end{gathered}
$$
\end{example}
\begin{example}
$$
\begin{gathered}
    A  = \mathbb{Z} \\
    R = \{(x,y) \mid x,y \in A, 2 | x-y \} \\
    [0]_R = \{-4,-2,0,2,4,6,\ldots\} \\
    [1]_R = \{-3,-1,1,3,5,\ldots\} \\
    A/R = \{[0]_R, [1]_R \} \\
    D = \{0,1\} \\
    \text{ (or any two even/odd numbers)}
\end{gathered}
$$
\end{example}
\section{Binary Relations}

\begin{definition}
\(R \subseteq A \times A\)

\emph{Relations} are made up of pair from set \(A\) that fulfill certain
parameters.

\emph{Equivalence Relation}: A binary relation that is
\emph{reflexive}
\footnote{\(R\) is \emph{reflexive} if \(\forall \; a \in A: (a,a) \in
R\)\label{reflexive}},
\emph{symmetric}
\footnote{\(R\) is \emph{symmetric} if \(\forall \; a,b \in A: (a,b) \in R
\implies (b,a) \in R\)\label{symmetric}},
and \emph{transitive}
\footnote{\(R\) is \emph{transitive} if \(\forall \; a,b,c \in A: (a,b), (b,c)
\in R \implies (a,c) \in R\) \label{transitive}}.

\end{definition}
\begin{example}
\[A = \mathbb{Z} \times \mathbb{Z} \setminus \{0\}\]
\[R=\{(m_1,n_1),(m_2,n_2) \mid m_1n_2=m_2n_1\}\]
\end{example}

To prove that this relation is an equivalence relation, we have to prove
that it is \emph{reflexive}, \emph{symmetric} and \emph{transitive}.

\end{document}
