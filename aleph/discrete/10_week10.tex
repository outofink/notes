\documentclass[00_complete]{subfiles}

%\documentclass[12pt]{report}
\usepackage[utf8]{inputenc}
\usepackage{amsmath,amssymb,amsthm,gensymb,parskip,graphicx,footmisc,csquotes,enumerate,datetime2}
\usepackage[]{libertinus}
\usepackage[breaklinks]{hyperref}
\hypersetup{
  pdfauthor={Moshe Krumbein},
  colorlinks=true,
  linkcolor={black},
  filecolor={black},
  citecolor={black}, %blue
  urlcolor={black}, %blue
}
\usepackage[top=30mm,bottom=30mm,left=30mm,right=30mm]{geometry}
%\setlength{\emergencystretch}{2em} % prevent overfull lines
\providecommand{\tightlist}{%
\setlength{\itemsep}{0pt}\setlength{\parskip}{0pt}}

\renewcommand\qedsymbol{$\blacksquare$}

\theoremstyle{definition}
\newtheorem*{definition}{Definition}
\newtheorem*{theorem}{Theorem}
\newtheorem*{axiom}{Axiom}
\newtheorem*{lemma}{Lemma}

\theoremstyle{remark}
\newtheorem*{note}{Note}
\newtheorem*{symbols}{Symbol}
\newtheorem{example}{Example}[section]
\newtheorem*{claim}{Claim}
\newtheorem*{conclusion}{Conclusion}
\newtheorem*{reminder}{Reminder}

\usepackage{fancyhdr}
\usepackage[italicdiff]{physics}
\MakeOuterQuote{"}

\renewcommand{\chaptermark}[1]{\markboth{#1}{}}

\pagestyle{fancy}

\setlength{\headheight}{14.5pt}
\addtolength{\topmargin}{-2.5pt}

\fancyhf{}
\rhead{Moshe Krumbein}
\lhead{\chaptermark}
\cfoot{\thepage}
\fancyhead[R]{\chaptername~\thechapter}
\fancyhead[L]{\mbox{\leftmark}}

\usepackage[Rejne]{fncychap}
\usepackage{titling}

\makeatletter
\renewcommand{\@chapapp}{\vspace*{-100pt}\huge\thetitle}
\makeatother

\makeatletter
\newcommand{\subtitle}[1]{%
  {\center\vspace*{-60pt}%
  \linespread{1.1}\Large\scshape#1%
  \par\nobreak\vspace*{35pt}}
}
\makeatother

\newcommand{\Chapter}[2]{
    \def\n{#2}
    \setcounter{chapter}{\the\numexpr\n-1}
    \chapter{#1}
    \subtitle{\theauthor~- \thedate}
}

\DeclareMathOperator{\Ima}{Im}
\DeclareMathOperator{\Id}{Id}
\DeclareMathOperator{\cis}{cis}

\newcommand{\Mod}[1]{\ (\mathrm{mod}\ #1)}
\newcommand{\st}[0]{\;\mathrm{s.t.}\;}

\title{Discrete Mathematics}
\author{Moshe Krumbein}
\date{Fall 2021}

\begin{document}
\setcounter{chapter}{9}

\chapter{Asymptotic Growth}
\subtitle{\theauthor~- \thedate}

\section{Big O Notation}
\begin{definition}
    The sequence ($a_n$) is almost always positive if there exists $N_0\in
    \mathbb{N}$ such that $n>N_0$, $a_n>0$
\end{definition}
\begin{definition}
    Given sequences $(a_n)(b_n)$, they are almost always positive itself:
    $$a_n=O(b_n)$$
    $((a_n)=O((b_n)))$ if there exists $N \in \mathbb{N}$, $0 <c \in
    \mathbb{R}$ such that $n \leq N$: $a_n \leq cb_n$.
\end{definition}
\begin{example}
    $$a_n=7n+5 \qquad b_n=n^2$$
    $$\underbrace{7n+5}_{a_n} \leq 7n+5n = 12n \leq 12n^2$$
    We define $N=1$, $c=12$. For all $n\geq 1$:
    $$a_n \leq 12b_n$$
\end{example}
\subsection{Characteristics}
\begin{enumerate}
    \item If there exists $N \in \mathbb{N}$ such that for all $n \geq N$: $a_n
        \leq b_n$, then $a_n = O(b_n)$
    \item If $a_n = O(b_n)$ and $0 < k \in \mathbb{R}$, then $ka_n=O(b_n)$
    \item If $a_n=O(b_n)$, $a'_n=O(b_n)$, then $a_n+a'_n=O(b_n)$
    \item If $a_n=o(b_n)$, $b_n=O(d_n)$, then $a_n = O(d_n)$
\end{enumerate}
\section{Claims about Asymptotic Growth}
\begin{claim}
    Suppose $k,l \in \mathbb{R}, k<n$, then:
    \begin{enumerate}
        \item $(n^k) = O(n^l)$
        \item $(n^l) \neq O(n^k)$
    \end{enumerate}
\end{claim}
\begin{proof}
    \begin{enumerate}
        \item For all $n \in \mathbb{N}, n^k \leq n^l$. $\checkmark$
        \item Suppose $n \geq N$: $n^l \leq cn^k \implies n^{l-k}\leq c$. If we
            define $n>c^\frac{1}{l-k}$ then:
            $$c \geq n^{l-k}\geq \left(c^{\frac{1}{l-k}}\right)^{l-k} = c$$
            Which is a contradiction. $\checkmark$
    \end{enumerate}
\end{proof}
\begin{claim}
    For $n \in \mathbb{N}$ we define
    $b_n=\underbrace{n(n-1)\dots(n-k+1)}_{\frac{n!}{(n-k)!}}$:
    \begin{enumerate}[I.]
        \item $n^k = O(n_n)$
        \item $b_n = O(n^k)$
    \end{enumerate}
\end{claim}
\begin{proof}
    \begin{enumerate}[I.]
        \item We need to prove $n^k = O(b_n)$:
            $$
            \begin{gathered}
                \frac{n}{2} \leq n-k<n-k+1 \iff \\
                n\leq 2n-2k \iff \\ \boxed{2k\leq n}
            \end{gathered} \quad \vline \quad
            \begin{gathered}
                n^k=2^k\left(\frac{n}{2}\right)^k \\
                \frac{n}{2} \leq n-k+1 \\
                \frac{n}{2}\leq n-k \\ \vdots \\
                \frac{n}{2} \leq n
            \end{gathered}
            $$
            Therefore:
            $$n^k\left(\frac{n}{2}\right)^k \leq 2^kn(n-1)\dots(n-k+1) = 2^kb_n$$
            We choose $c=2^k, N=2k$, and therefore:
            $$n^k\leq cb_n$$
        \item $b_n= \underbrace{n(n-1)\dots(n-k+1)}_{k \text{ terms}} \leq n^k$

            ($c=1, N=1 \implies b_n=O(a_n)$)
    \end{enumerate}
\end{proof}
\begin{conclusion}
    From the claim and the characteristics:
    $$
    \begin{gathered}
        n^k = O\binom{n}{k} \qquad \binom{n}{k} = O(n^k) \\
        b_n = O\binom{n}{k} \qquad \binom{n}{k} = O(b_n)
    \end{gathered}
    $$
\end{conclusion}
\begin{claim}
    Given $k \in \mathbb{N}$:
\begin{enumerate}[I.]
    \item $n^k=O(2^k)$
    \item $2^k \neq O(n^k)$
\end{enumerate}
\begin{proof}
    \begin{enumerate}[I.]
    \item \begin{gather*}
        2^n = \binom{n}{0} + \binom{n}{1} + \dots + \binom{n}{n} \implies
        \forall n \geq k: \binom{n}{k}\leq 2^n \\
        \implies N=k, c=1: \binom{n}{k}=O(2^n), n^k=O\binom{n}{k}
    \end{gather*}
    \begin{conclusion}
        \begin{gather*}
            n^k=O(2^n) \tag{\checkmark}
        \end{gather*}
    \end{conclusion}
    \item Proof by contradiction:

    There exists $k_o \in \mathbb{N}$ such that: $2^n = O(n^{k_0})$.

    We already know that $n^{k_0+1}=O(2^n)$. Using the property of
    transitivity: $n^{k_0+1}=O(n^{k_0})$, but since $k_0+1>k_0$, this
    contradicts our first claim.
    \end{enumerate}
\end{proof}
\begin{note}
    We can generalize this for all $\alpha>1$:
\begin{enumerate}[I.]
    \item $n^k=O(\alpha^k)$
    \item $2^k \neq O(\alpha^k)$
\end{enumerate}
\end{note}
\end{claim}
\section{Big \texorpdfstring{$\Theta$}{Theta} Notation}
\begin{definition}[Big $\Theta$ Notation]
   Given sequences $(a_n)(b_n)$, that are almost always positive, we define:
   $$a_n = \Theta(b_n)$$
   If:
   $$a_n=O(b_n) \quad b_n=O(a_n)$$
\end{definition}
\begin{example}
    $$n^k=\Theta\binom{n}{k}$$
\end{example}
\begin{claim}
    $\Theta$ is an equivalence relation of the collection of all of the
    sequences that are almost always positive.
    \begin{proof}
        \begin{enumerate}[I.]
            \item Reflexivity:
            \begin{gather*}
                \forall n \quad a_n \leq a_n \implies a_n = O(a_n) \\
                a_n = \Theta(a_n) \tag{\checkmark}
            \end{gather*}
            \item Symmetry:

                If $a_n \Theta(b_n)$, then $a_n = O(b_n)$ and $b_n=O(a_n)$,
                which implies that: $b_n = \Theta(a_n)$.
                \begin{gather*}
                    \tag{\checkmark}
                \end{gather*}
            \item Transitivity:

                Given $a_n=\Theta(b_n)$ and $b_n=\Theta(c_n)$:
                \begin{align*}
                    a_n=\Theta(b_n) \to \quad& a_n=O(b_n), a_n=O(a_n) \\
                    b_n=\Theta(c_n) \to \quad& b_n=O(c_n), c_n=O(b_n)
                \end{align*}
                \begin{gather*}
                    \implies a_n=\Theta(c_n) \tag{\checkmark}
                \end{gather*}
        \end{enumerate}
    \end{proof}
\end{claim}
\begin{definition}[Polynomial Growth]
    Given a sequence $(a_n)$ which is almost always positive, we say that for
    $(a_n)$ has a \emph{polynomial growth rate} if there exists $k \in
    \mathbb{N}$ such that $a_n = \Theta(n^k)$.

    In this case we say that $(a_n)$ has polynomial growth of order $k$.
\end{definition}
\begin{example}
    $(2^n)$ does not have a polynomial growth rate.
\end{example}
\begin{claim}
    Given:
    \[
        p(n)=n^k + \sum_{i=0}^{k-1}t_in^i
    \]
    Therefore $(p(n))$ has a polynomial growth rate of order $k$.
    \begin{proof}
        $p_n = O(n^k)$.
        \begin{gather*}
            p(n)=n^k + \sum_{i=0}^{k-1}t_in^i \leq \left|n^k +
            \sum_{i}^{k-1}t_in^i\right| \leq n^k + \sum_{i=0}^{k-1}|t_i|n^i \\
            \leq n^k + \sum_{i=0}^{k-1}|t_i|n^k =
            n^k\underbrace{\left(1+\sum_{i=0}^{k-1}|t_i|\right)}_{c}
        \end{gather*}
        $p(n) = O(n^k)$.

        Now to prove $n^k = O(p(n))$:
        \begin{gather*}
            p(n)=n^k + t_{k-1}n^{k-1}+k_{k-2}n^{k-2}+\dots+t_1n +t_0 \\
            =n^k\underbrace{\left(1 +
            \frac{t_{k-1}}{n}+\frac{t_{k-2}}{n^2}+\dots+\frac{t_0}{n^k}\right)}
            _{k+1 \text{ addends}}
        \end{gather*}
        For a large enough $n$, $p(n)>\frac{1}{2}$, and therefore
        $p(n)>\frac{1}{2}n^k \implies n^k < 2p(n)$.

        Therefore, $n^k=O(p(n)) \implies p(n) = \Theta(n^k)$.
    \end{proof}
\end{claim}
\section{Examples}
\begin{example}
    $a_n$ is the number of ways to place 100 distinct balls to $n$ cells.
    $$a_n=n^{100}$$
    $a_n$ has polynomial growth rate of order 100.
\end{example}
\begin{example}
    $a_n$ is the number of ways to place 100 distinct balls to $n$ cells such
    that in each cell there contains at most one ball.

    For all $n \geq 100$, $a_n>0$, and therefore $(a_n)$ is almost always
    positive.
    $$a_n=\frac{n!}{(n-100)!}$$
    $a_n$ has polynomial growth rate of order 100.
\end{example}
\begin{example}
    $a_n$ is the number of ways to place $n$ distinct balls to 100 cells.

    $$a_n=100^n$$
    $a_n$ does not have an polynomial growth rate.
\end{example}
\begin{example}
    $a_n$ is the number of ways to place $n$ distinct balls to 100 cells such
    that in each cell there contains at most one ball.

    For all $n > 100$, $a_n=0$, which makes it not a sequence which is
    almost always positive.
\end{example}
\begin{example}
    $a_n$ is the number of ways to place 100 identical balls to $n$ cells such
    that in each cell there contains at most one ball.

    For all $n \geq 100$, $a_n>0$, and therefore $(a_n)$ is almost always
    positive.
    $$a_n=\binom{n}{100}$$
    $a_n$ has polynomial growth rate of order 100.
\end{example}
\begin{example}
    $a_n$ is the number of ways to place 100 identical balls to $n$ cells.
    $$a_n=\binom{n+99}{100}$$
    $a_n$ has polynomial growth rate of order 100.
\end{example}
\begin{example}
    $a_n$ is the number of ways to place $n$ identical balls to 100 cells.
    $$a_n=\binom{n+99}{99}$$
    $a_n$ has polynomial growth rate of order 99.
\end{example}
\begin{example}
    Find the order of polynomial growth $k$ of $a_n$ where $a_n$ is the number of
    ways to place 100 distinct balls to $n$ cells such that:
    \begin{enumerate}[a.]
        \item In the first cell there are exactly 4 balls
        $$a_n = \binom{100}{4}(n-1)^{96}$$
        $k=96$.
        \item In the first cell there is at most 4 balls
        \begin{gather*}
            a_n \leq \text{total number of ways to place $100$ balls in $n$
            cells} = n^{100} \\
            a_n = O(n^{100}) \tag{I} \\
            a_n \geq \text{total number of ways to place $100$ balls in $n-1$
            cells} = (n-1)^{100} \\
            n^{100} = O(a_n) \tag{II} \\
            a_n=\Theta(n^{100}), k=100
        \end{gather*}
        \item There exists a cell that has exactly 4 balls
        \begin{gather*}
        a_n \geq \text{number of ways such that one cell has 4 balls and all other
        have at least one} \\
        =\underbrace{\binom{n}{1}\binom{100}{4}(n-1)(n-2)\dots(n-96)}_{\text{polynomial
        of degree 97}} \implies n^{97} = O(a_n) \tag{I} \\
        \end{gather*}
        We define $A_k$ to be the number of ways such that in the $k$-th cell
        there are exactly 4 balls:
        \begin{gather*}
            A_k=\binom{100}{4}(n-1)^{96} \\
            a_n = \left|\bigcup_{k=1}^{n} A_k \right| \leq
            \left|\sum_{k=1}^{n}A_k\right| = n\binom{100}{4}(n-1)^{96} \implies
            a_n =O(n^{97}) \\
            a_n =\Theta(n^{97}), k=97
        \end{gather*}
        \item There exists a cell that has at least 4 balls
        $$c \leq d \implies n^{97} = O(a_n)$$
        $A_k$ - all the ways such that in the $k$-th cell there are at least 4
        balls:
        \begin{gather*}
            |A_k|=\binom{100}{4}n^{96} \\
            a_n=\left|\bigcup_{k=1}^nA_k\right|\leq\sum_{k=1}^{n}|A_k|
            = n\binom{100}{n}n^{96} \implies a_n = O(n^{97}) \\
            a_n = \Theta(n^{97}), k=97
        \end{gather*}
    \end{enumerate}
\end{example}
\end{document}
